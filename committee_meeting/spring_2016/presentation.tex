\documentclass{beamer} 
\usepackage{amsmath,amsthm}
\usepackage{graphicx,microtype,parskip}
\usepackage{caption,subcaption,multirow}
\usepackage{attrib}

\frenchspacing

\usetheme{default}
\usecolortheme{whale}

\setbeamertemplate{navigation symbols}{}

\setbeamercolor{title}{fg=blue,bg=white}

\setbeamercolor{block title}{fg=white,bg=gray}
\setbeamercolor{block body}{fg=black,bg=lightgray}

\setbeamercolor{block title alerted}{fg=white,bg=darkgray}
\setbeamercolor{block body alerted}{fg=black,bg=lightgray}

\AtBeginSection[]
{
  \begin{frame}
    \tableofcontents[currentsection]
  \end{frame}
}

\title{}
\author{}
\institute{}
\date{}


\begin{document}

\begin{frame}
  \tableofcontents
\end{frame}


\section{How macroecology affects macroevolution: the interplay between extinction intensity and trait-dependent extinction in brachiopods.}

\begin{frame}
  \begin{block}{History}
    \begin{itemize}
      \item presented at GSA 2015
      \item rejected from \textit{Evolution}
        \begin{itemize}
          \item encouraged resubmit
          \item audience issues
          \item difficult and transformative reviews
        \end{itemize}
    \end{itemize}
  \end{block}
\end{frame}

% biggest criticism was that i completely ignored effect of sampling
%   i measure sampling as the simple gap statistic
%   use it as a covariate of duration
%   this is the estimate the biasing effect of sampling
%     NOT correct duration for sampling
%     record is too coarse for a lot of methods
%     just a different way of doing it
%       i would use the new Wang approach for sampling instead of simple CLS type model
% effect of sampling probability on duration
%   (imupted) gap statistic
%   some limit pushing, but use beta regression to help estimate

% results
%   table and over-all probabilities 
%   environmental preference
%   change over time
%   covariance matrix


\section{Taxon occurrence as a function of both emergent biological traits and its environmental context}

% Theoretical underpinning
%   change in regional species pool over time
%   
%   connection to macroecology
%   connection to macroevolution

% modified fourth-corner type problem
%   presence_{i,t} | individual-level_{i,t}, group-level_{t}
%   assumes effect of covariates constant for all t
% no individual X group covariates
%   instead of group-level covariates included as part of multi-level model
%   keeping in mind group-level covariates depend on t

% after \citep{Smits2015}
%   decrease in extinction risk over time
%     notice the pattern at the paleogene-neogene boundary
%   higher extinction risk in arboreal taxa
%     strong effect
%     what is variation in effect? paleogene vs neogene
%       is the transition/neogene effect so strong to appear constant?
%     driven by the environmental change?

% Vallesian crisis in Europe
%   faunal shift at Paleogene-Neogene boundary
%   do we see this in NA in terms of ecotypes?

\begin{frame}
  \frametitle{Analysis of Cenozoic mammal fossil record for NA}
  \begin{columns}
    \begin{column}{0.5\textwidth}
      individual-level (genus)
      \begin{itemize}
        \item intercept term, \\varying by time
        \item locomotor type/category
          \begin{itemize}
            \item arboreal, digitigrade, plantigrade, unguligrade, fossorial, scansorial
          \end{itemize}
        \item dietary type/category
          \begin{itemize}
            \item carnivore, herbivore, insectivore, omnivore
          \end{itemize}
        \item body size \\(rescaled log body mass)
        \item phylogenetic effect
      \end{itemize}
    \end{column}
    \begin{column}{0.5\textwidth}
      group-level (2 My time unit)
      \begin{itemize}
        \item intercept
        \item isotope record
          \begin{itemize}
            \item mean and interquartile range of rescaled value
          \end{itemize}
        \item temperature record
          \begin{itemize}
            \item mean and interquartile range of rescaled value
          \end{itemize}
        \item plant community phase following Graham
      \end{itemize}
    \end{column}
  \end{columns}
\end{frame}


\begin{frame}
  \begin{block}{Model of taxon occurrence}
    \begin{itemize}
      \item response is p/a of genus in NA at time \(t\)
        \begin{itemize}
          \item Bernoulli variable 
          \item probability is (observation prob) times (``true'' presence)
        \end{itemize}
      \item observation probability is effect of sampling/fossil record
        %\begin{itemize}
          %\item preservation as logistic regression with relative abundance as covariate
        %\end{itemize}
      \item true presence is multi-level logistic regression
        \begin{itemize}
          \item individual- and group-level
        \end{itemize}
      \item break-point model is the eventual goal
    \end{itemize}
  \end{block}
\end{frame}

\begin{frame}
  \begin{block}{Posterior predictive checks}
    \begin{itemize}
      \item simulate fossil record given only \(y_{\_t = 1}\) and \(\theta\)
        \begin{itemize}
          \item where \(\theta\) is the set of all parameters
        \end{itemize}
      \item equivalent to leave-one-out cross-validation for time series?
        \begin{itemize}
          \item Bayesian statement is \(p(\tilde{y}_{t + 1} | y_{t} \theta)\)
        \end{itemize}
      \item ROC as measure of performance
    \end{itemize}
  \end{block}
\end{frame}


\section{Other projects}

\begin{frame}
  \begin{block}{How cryptic is cryptic diversity? Machine learning approaches to classifying morphological variation in the Pacific Pond Turtle (\textit{Emys marmorata})}
    \begin{itemize}
      \item estimate which species classification is best supported by morphology
        \begin{itemize}
          \item multiple machine learning approaches
          \item focus on one turtle species complex
          \item results compared against results from two other turtle datasets
          \item comparison of in- and out-of-sample model performance
        \end{itemize}
      \item collaboration with Ken, Jim Parham, and Bryan Stuart
      \item submitted to then rejected from Systematic Biology
      \item resubmitted soon
    \end{itemize}
  \end{block}
\end{frame}

% bivalve naming rate w/ stewart
\begin{frame}
  \begin{block}{Modeling the rate at which new species are named.}
    \begin{itemize}
      \item collaboration with Stewart Edie; he's lead
      \item I developed the statistical model
        \begin{itemize}
          \item zero-inflated Poisson model 
          \item both Bernoulli and Poisson distributions are time series models
          \item response is the number of species named per publication \\per year for each biogeographic province
          \item increasing, decreasing, or level?
        \end{itemize}
      \item draft phase
      \item targets seem to be PNAS or Systematic Biology
    \end{itemize}
  \end{block}
\end{frame}


\section{Moving forward}

\begin{frame}
  \begin{block}{Research statement}
    \begin{itemize}
      \item Intersection of macroevolution and macroecology.
      \item Quantitative approaches to understanding global and regional patterns of biodiversity.
      \item Paleontological data.
      \item Non-taxon specific; with emphasis on mammals.
      \item More like an ecologist-modeler than an evolution-modeler
        \begin{itemize}
          \item No one-model to fit them all; tailor-made models for question.
          \item This contrasts with the field.
        \end{itemize}
    \end{itemize}
  \end{block}
\end{frame}


\end{document}
