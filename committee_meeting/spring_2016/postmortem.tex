\documentclass{article}

\usepackage{amsmath, amsthm}
\usepackage{setspace}
\usepackage{microtype, parskip}
\usepackage[comma,sort&compress]{natbib}
\usepackage{lineno}
\usepackage{docmute}
\usepackage{caption, subcaption, multirow, morefloats, rotating}
\usepackage{wrapfig}

\frenchspacing

\title{Postmortem of Peter Smits' Spring 2016 Committee Meeting}
\date{}

\begin{document}
\linenumbers
\modulolinenumbers[2]

\maketitle

% introduction
Peter Smits' Spring quarter committee meeting occurred Thursday 12 May, 2016. Michael Foote, Kenneth Angielczyk, Graham Slater, and David Polly were present. Richard Ree was out of the country and thus couldn't attend; arrangements have been made for an independent discussion between Peter and Richard at a later date. In total, the meeting lasted two hours.

This meeting was broadly structured into three parts: dissertation projects, collaboration projects, and moving forward (i.e. defense and post-docs). Slides from the structuring presentation and an academic timeline were presented but electronically and physically to all members of the committee.

Currently, Peter's dissertation is taking the form of three chapters: 1) effect of traits on extinction risk in Cenozoic mammals of North America, 2) effects of traits on extinction risk in post-Cambrian Paleozoic brachiopods, and 3) analysis of changes in the demographic structure of Cenozoic mammals of North American wrt emergent species traits and environmental context. Chapter 1 of Peter's dissertation was published last year, so approximately two chapters remain. 

Chapter 2 was in review as of Fall quarter's committee meeting. Since then it was rejected, revised, and resubmitted. As of this quarter's committee meeting, Chapter 2 was again under review.

Chapter 3 was premiered at Fall quater's committee meeting and had since undergone a great deal of statistical and conceptual revisions. We reviewed the current issues facing this project, primarily the extreme computational cost to fitting the full model using full stochastic sampling Bayes. The full model takes into account the effect of imperfect observation but requires a lot of parameters. The two major pieces of advice were given were simplfy the model by removing the latent-discrete state, try approxmate Bayesian methods to get an approximation of the posterior. Both will be tested.

The other two projects reviewed were Peter's collaboration projects. The first of these was on comparing different classification schemes to see which is best supported by the data. This project is in collaboration with Ken. The other project discussed was a collaboration between Peter and Stewart Edie on modeling the rate at which new species are identified while taking into account biogeographic differences and taxonomic effort. The latter is much closer to be finished than the former, though both could be submitted this year.

Finally, we discussed Peter's ``research program'' and post-doctoral opportunities. Peter needs to work on developing a coherent idea of a ``research program'' in terms of what type of work/questions he wants to be known for. Specifically, he should cast himself as an independent thinker and not just someone who cleans up other's messes (``don't make a career of cleaning up after the elephants''). Additionally, while collaborating is fine, it was suggested that Peter try and take a stronger intellectual lead in the collaborations instead of acting as an analysis mercenary.

As with Fall quarter's meeting, Peter's committee predicted him finishing Spring/Summer of 2017. Peter is expected to submit post-doctoral applications in the fall of 2016.


Addition from Michael Foote: [Peter's] committee [is] quite pleased with your work and satisfied that you are on track to complete a dissertation that will be a real credit to CEB and the U of C. 

\end{document}
