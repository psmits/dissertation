\documentclass{article}

\usepackage{amsmath, amsthm}
\usepackage{setspace}
\usepackage{microtype, parskip}
\usepackage[comma,sort&compress]{natbib}
\usepackage{lineno}
\usepackage{docmute}
\usepackage{caption, subcaption, multirow, morefloats, rotating}
\usepackage{wrapfig}

\frenchspacing

\title{Postmortem of Peter Smits' Fall 2016 Committee Meeting}
\date{}

\begin{document}
\linenumbers
\modulolinenumbers[2]

\maketitle

Peter Smits' Fall 2016 committee meeting took place November 8th. Michael Foote, Kenneth Angiecylzk, Graham Slater, Rick Ree, and P. David Polly (via Skype) were all present. There were three major objectives to this meeting: postdoctoral applications and post-PhD plans, the state of chapter 2, and the state of chapter 3. The first chapter of my dissertation was published in \textit{Proceedings of the National Academy} in 2015. In total, the meeting lasted about an hour and half. 

Prior to this meeting, Peter had submitted two competitive postdoctoral applications: Miller Fellowship at Berkeley with Charles Marshall, and the Killam Fellowship at University of British Columbia with Mathew Pennell. The comment from the committee was that Peter should continue to apply for both postdoctoral positions and jobs. Following this meeting, Peter contacted Felisa Smith (University of New Mexico) to follow up on a soft-offer presented at GSA.

Chapter 2 of Peter's dissertation, which is a study of extinction risk in post-Cambrian Paleozoic brachiopods, was rejected for the second time from \textit{Evolution} this Summer. The paper was rejected for both being overly technical and domain specific. The committee and Peter discussed ways of translating the paper to a more general and less technical audience. The major barriers are differences in approach, cross-domain translation of parameters, and changing the emphasis away from brachiopods and to the more meaningful results. In general, the committee supports revising and submitting to \textit{The American Naturalist} though Peter would need to consider how to accomplish this carefully. Additionally, revising this chapter and getting it published is secondary to finishing chapter 3 and the dissertation.

The final part of the committee meeting was a discussion of the current results and remaining aspects of the third chapter of Peter's dissertation. This chapter concerns the relationship between changes in the North American mammal species pool and the ecologies of those species. The preliminary results of this analysis were presented at the Geological Society of America meeting in September earlier this year. The remaining aspects of this project include some model improvement, detailed analysis of posterior estimates, and writing up the manuscript. Given these challenges, the committee was confident in Peter's ability to finish this chapter within the next 4-6 months. 

At the conclusion of the meeting the committee agreed that Peter should aim to finish his dissertation and graduate in Spring quarter. Kenneth Anegielzyck also requested that Peter work on a think- or perspective-piece on his approach to modelling macroevolutionary questions. This perspective piece will form aspects of the introduction and conclusion of Peter's dissertation.


\end{document}
