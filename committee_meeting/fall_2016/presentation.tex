\documentclass{beamer} 
\usepackage{amsmath,amsthm}
\usepackage{graphicx,microtype,parskip}
\usepackage{caption,subcaption,multirow}
\usepackage{attrib}

\frenchspacing

\usetheme{default}
\usecolortheme{whale}

\setbeamertemplate{navigation symbols}{}
\setbeamertemplate{footline}[frame number]

\setbeamercolor{title}{fg=blue,bg=white}

\setbeamercolor{block title}{fg=white,bg=gray}
\setbeamercolor{block body}{fg=black,bg=lightgray}

\setbeamercolor{block title alerted}{fg=white,bg=darkgray}
\setbeamercolor{block body alerted}{fg=black,bg=lightgray}


\AtBeginSection[]
{
  \begin{frame}
    \tableofcontents[currentsection]
  \end{frame}
}

\title{Fall 2016 committee meeting for Peter Smits}
\author{}
\institute{}
\date{}


\begin{document}

\begin{frame}
  \maketitle
\end{frame}

\begin{frame}
  \tableofcontents
\end{frame}


\section{Post-doc applications}
\begin{frame}
  \frametitle{What I've applied for}
  \begin{itemize}
    \item project idea: models of both within- and across-species continuous trait evolution
    \item Miller Fellowship at Berkeley with Charles Marshall
    \item Killam Fellowship at UBC with Matt Pennell
  \end{itemize}
\end{frame}


\section{Chapter 2: brachiopod survival}
\begin{frame}
  \frametitle{Revisions following rejection}
  \begin{itemize}
    \item reviews considered text to be both too ``paleontological'' and technical
    \item new target is \textit{American Naturalist}.
    \item conceptual figures (e.g. behaviour Wiebull distribution, model structure)
    \item text emphasis on fitness, selection interpretations
  \end{itemize}
\end{frame}


\section{Chapter 3: mammal species pool}

\begin{frame}
  \begin{block}{Note}
    Correction to Chapter 1 issued this Summer. No changes to conclusions, results.
  \end{block}
\end{frame}

\begin{frame}
  \begin{alertblock}{Question}
    When are certain ecologies/ecotypes enriched or depleted?
  \end{alertblock}
\end{frame}

\begin{frame}
  \frametitle{Differences in extinction risk}
  \begin{center}
    \includegraphics[height=0.8\textheight,width=\textwidth,keepaspectratio=true]{figure/smits_2015_results}
  \end{center}

  \attrib{\footnotesize{Smits, 2015, \em{PNAS}}}
\end{frame}

\begin{frame}
  \frametitle{Species pool concept}
  \begin{center}
    \includegraphics[height=0.8\textheight,width=\textwidth,keepaspectratio=true]{figure/schemske_pool}
  \end{center}

  \attrib{\footnotesize{Mittelbach and Schemske, 2015, \em{TREE}}}
\end{frame}

\begin{frame}
  \frametitle{Eco-cube and ecotypes}
  \begin{center}
    \includegraphics[height=0.8\textheight,width=\textwidth,keepaspectratio=true]{figure/ecocube}
  \end{center}

  \attrib{\footnotesize{Bambach \em{et al.}, 2007, \em{Palaeontology}}}
\end{frame}

\begin{frame}
  \frametitle{The fourth-corner problem}
  \begin{center}
    \includegraphics[height=0.8\textheight,width=\textwidth,keepaspectratio=true]{figure/warton_fourth_corner}
  \end{center}

  \attrib{\footnotesize{Brown \em{et al.}, 2014, \em{Methods Ecol. Evol.}}}
\end{frame}

\begin{frame}
  \frametitle{Covariates of interest}
  \begin{columns}
    \begin{column}{0.5\textwidth}
      individual-level \\(species i at time unit t)
      \begin{itemize}
        \item log-odds of occurrence probability at time t
        \item effect of locomotor type
          \begin{itemize}
            \item arboreal, digitigrade, plantigrade, unguligrade, fossorial, scansorial
          \end{itemize}
        \item effect of dietary type
          \begin{itemize}
            \item carnivore, herbivore, insectivore, omnivore
          \end{itemize}
        \item effect body size \\(rescaled log body mass)
      \end{itemize}
    \end{column}
    \begin{column}{0.5\textwidth}
      group-level (2 My time unit t)
      \begin{itemize}
        \item overall mean of log-odds of occurrence probability
        \item temperature record based on Mg/Ca estimates
          \begin{itemize}
            \item mean and interquartile range of rescaled value
          \end{itemize}
        \item plant community phase following Graham 2011
      \end{itemize}
    \end{column}
  \end{columns}
\end{frame}

\begin{frame}
  \frametitle{Model of taxon occurrence}
  \begin{itemize}
    \item response is p/a of genus in NA at time \(t\)
      \begin{itemize}
        \item Bernoulli variable 
        \item probability is (observation prob) times (``true'' presence)
      \end{itemize}
    \item observation probability is effect of sampling/fossil record
      \begin{itemize}
        \item basic model does not model sampling
      \end{itemize}
    \item the latent discrete ``true'' presence modeled as a \\multi-level logistic regression
      \begin{itemize}
        \item individual- and group-level
      \end{itemize}
  \end{itemize}
\end{frame}

\begin{frame}
  \frametitle{Paleo-fourth corner model}
  \begin{center}
    \includegraphics[height=0.8\textheight,width=\textwidth,keepaspectratio=true]{figure/paleo_fourth_corner}
  \end{center}
\end{frame}

\begin{frame}
  \frametitle{Model and sampling statement definition}
  \footnotesize{
    \begin{align*}
      y_{i,t} &\sim \text{Bernoulli}(\rho_{t} z_{i,t}) \\
      \text{logit}(\rho_{t}) &\sim \mathcal{N}(\rho^{'}, \sigma_{\rho}) \\
      z_{i,t} &\sim \text{Bernoulli}(\theta_{i, t}) \\
      \text{logit}(\theta_{i, t}) &= z_{i,t-1} (X_{i} \beta_{t\_}) + (\prod_{k = 1}^{t-1} 1 - z_{i,k}) (X_{i} \beta_{t\_}) \\
      \beta_{t} &\sim \text{MVN}(\mu, \Sigma) \\
    \end{align*}
  }
  \scriptsize{Note: Product term ensures taxon-loss is permanent. Implementation in Stan marginalizes over all possible (range-through) values of \(z\) instead of estimating the discrete parameters. I also use a noncentered parameterization of the hierarchical effects for better posterior sampling behavior. This presentation excludes final (hyper)priors.}
\end{frame}

\begin{frame}
  \frametitle{Parameter estimation and inference}
  \begin{columns}
    \begin{column}{0.45\textwidth}
      \begin{itemize}
        \item full HMC/MCMC slow
        \item Automatic Differentiation Variational Inference (ADVI) 
          \begin{itemize}
            \item approximate Bayesian inference
            \item assumes posterior is Gaussian, no correlation between parameters
            \item true Bayesian posterior
          \end{itemize}
      \end{itemize}
    \end{column}
    \begin{column}{0.55\textwidth}
      \includegraphics[height=0.9\textheight,width=\textwidth,keepaspectratio=true]{figure/stan_logo}
    \end{column}
  \end{columns}
\end{frame}

\begin{frame}
  \frametitle{Posterior predictive performance}
  \begin{center}
    \includegraphics[height=\textheight,width=\textwidth,keepaspectratio=true]{figure/pred_occ}
  \end{center}
\end{frame}

\begin{frame}
  \frametitle{Effect of mass on log-odds of occurrence}
  \begin{columns}
    \begin{column}{0.5\textwidth}
      Basic model
      \vspace*{0.05\textheight}

      \includegraphics[height=\textheight,width=\textwidth,keepaspectratio=true]{figure/mass_eff_basic}
    \end{column}
    \begin{column}{0.5\textwidth}
      Full model
      \vspace*{0.05\textheight}

      \includegraphics[height=\textheight,width=\textwidth,keepaspectratio=true]{figure/mass_eff_full}
    \end{column}
  \end{columns}
\end{frame}

\begin{frame}
  \frametitle{Probability occurrence is of ecotype (basic model)}
  \begin{center}
    \includegraphics[height=0.8\textheight,width=\textwidth,keepaspectratio=true]{figure/cept_occur_prob_basic}
  \end{center}
\end{frame}

\begin{frame}
  \frametitle{Probability occurrence is of ecotype (full model)}
  \begin{center}
    \includegraphics[height=0.8\textheight,width=\textwidth,keepaspectratio=true]{figure/cept_occur_prob_full}
  \end{center}
\end{frame}

\begin{frame}
  \frametitle{Group-level effects (plant phase, climate)}
  \begin{center}
    \includegraphics[height=0.8\textheight,width=\textwidth,keepaspectratio=true]{figure/gamma_est_full}
  \end{center}
\end{frame}

\begin{frame}
  \begin{block}{Concerns and conclusions}
    \begin{itemize}
      \item basic and full models have similar results until Neogene
      \item posterior predictive simulations disimilar to observed; poor model adequacy
        \begin{itemize}
          \item previous work has \emph{never} evaluated model adequacy
          \item second-order Markov process?
          \item full posterior inference?
        \end{itemize}
      \item decreasing ability to discern arboreal taxa over time (absence/increased rarity)
      \item increase in scansorial taxa over time
      \item increase in herbivorous taxa over time
      \item plant phase has small, idiosyncratic effects
    \end{itemize}
  \end{block}
\end{frame}

\begin{frame}
  \begin{block}{What's left with this project?}
    \begin{itemize}
      \item full posterior inference
        \begin{itemize}
          \item extremely slow b/c latent variables
          \item \textit{Folk Theory of Statistical Computing}?
        \end{itemize}
      \item improve model structure
        \begin{itemize}
          \item preservation as function of body size
          \item second-order Markov process?
        \end{itemize}
      \item write manuscript
        \begin{itemize}
          \item no strong conclusions yet
          \item journal target?
        \end{itemize}
    \end{itemize}
  \end{block}
\end{frame}


\section{Finishing}
\begin{frame}
  \begin{huge}
    See ``Timeline''
  \end{huge}
\end{frame}

\begin{frame}
  \frametitle{Options}
  \begin{itemize}
    \item Finish in the Spring or finish in the Summer.
    \item Idea
      \begin{itemize}
        \item Turn in, defend at end of Spring quarter.
        \item Technically finish Summer quarter.
        \item Unsure of details; need to contact Chair/Carolyn.
      \end{itemize}
  \end{itemize}
\end{frame}

\end{document}
