\documentclass{article}

\usepackage{amsmath, amsthm}
\usepackage{setspace}
\usepackage{microtype, parskip}
\usepackage[comma,sort&compress]{natbib}
\usepackage{lineno}
\usepackage{docmute}
\usepackage{caption, subcaption, multirow, morefloats, rotating}
\usepackage{wrapfig}

\frenchspacing

\title{Postmortem of Peter Smits' Fall 2015 Committee Meeting}
\date{}

\begin{document}
\linenumbers
\modulolinenumbers[2]

\maketitle

% general overview
%   systematic run through of past, present, future
This committee meeting was structured via a presentation separated into three parts: past, present, and future. Kenneth Angieczyk, Michael Foote, Rick Ree, and Graham Slater were present. David Polly, while originally scheduled to attend, ended up having an unexpected conflict. Peter Smits will be meeting with him separately to go over the material covered in the committee meeting. Graham Slater is a new addition to the committee which needs to be made official with CEB, but that shouldn't be an issue.

% past projects
%   2 chapters, (but 4 chapter dissertation)
%   1 published
%   1 submitted
%     need to get comments from other committee members, as they've never seen it
The past section of the presentation was focused on what had been accomplished since the last committee meeting in Winter quarter 2015. Since then, Peter Smits has published one paper in PNAS and submitted a second to Evolution. Both of these papers are expected to serve as independent chapters. Peter Smits solicited reviews of the paper submitted to Evolution from committee members, as he had not sent a copy of it to Rick, David, or Graham prior to submission.

% current projects
Peter Smits has two current projects: accumulation of regional diversity accumulation in brachiopods, and ratio of ecotype occurrences in Cenozoic mammals. Both of these projects were started after the previous committee meeting and this was the first time they were presented to the committee as a group.

%   brachiopods
%     very at risk of unimportance/irrelevancy
%       methods are not a substitute for new ideas
%       LDG is a bog and not worth my time
%       instead focus more on geosse style how taxa get into regions/assembly
%     new taxonomic group?
%     new temporal range (shorter, Cenozoic)?
%     need more meaningful geographic units
%       going all the way back to proposal
%     methods are fine, but question and mostly the system need rethinking
%       why? what hypotheses to do i have?
For the project on regional diversity accumulation in brachiopods, there were a few major concerns. The general consensus was that the project is very much at risk of unimportance or irrelevance. The focus on latitudinal diversity gradients and the use of non-biologically meaningful regions really sinks the impact/utility of the paper. This is a continuing reminder that methodological complexity is no substitute for interesting research. 

A few suggestions included focusing on a specific temporal window and taxonomic combinations where actual hypotheses of assembly can be test (e.g. glacial-interglacial periods), identifying real bio-geographic units, and generally refocusing on a possibly \uppercase{geosse}-style analysis of how taxa move into and out of regions during the assembly of a regional species pool. 

Peter is also most likely including either biological or environmental covariates in future analysis. A key future advance, however, is identifying meaningful bio-geographic provinces in order to better approximate regional-level assembly.

%   mammals
%     break up ground dwelling: plantigrade, digitigrade, unguligrade
%       help deal with ``ground dwelling'' problem
%     vegetation/envrionment reconstructions for north america
%       have conversation directly with Alroy 2000 (climate no effect on mammal evolution, but what about plant communities)
%       is everything just driven by environment type?
%       need to meet with Rick and his new postdoc about Cenozoic plant reconstruction
%         possible collaboration effort because dataset gets fuck huge
%     need to figure out how to better phrase certain parameter names
%     need to figure out a better way of doing or presenting effect of dietary category covariate
%     climate 
%       use (interquartile) range in addition to mean?
%       other isotope curves --> Mg-Ca curves which might behave better/reflect actual temp
%         actual estimate of temp
%       isotope curves are global. if i can get something environmental i have improved on this limitation
%     good call for AmNat? need to keep speaking evolutionary biology.
For the project on the ratio of mammalian ecotype occurrence over the Cenozoic, there were generally fewer concerns regarding the focus or direction of the project. Instead, however, most of the concerns were regarding data-types and choice of covariates of interest. A general problem that was identified was the fact that ``ground dwelling'', one of my responses of intesest, is too broad a category and is obfuscating possible results. Graham Slater suggested breaking up ground dwelling by locomotor form (i.e. plantigrade, digitigrade, unguligrade), which should be very do-able with a little research. 

Additionally, the use of the \(\delta\)O\(^{18}\) isotope curve as a proxy for climate was called into question for a few reasons. First, it is a measure of global temperature, not the temperature of North America. Second, it is a very assumption laden temperature proxy. And finally, it does not that are reflect the aspects of the environment actually of interest. Some alternatives that were considered were the Mg/Ca isotope record as it may be a less assumption laden temperature proxy, or the use of a North American plant-biome reconstruction. The potential source of this reconstruction is ambiguous but two possibilities were brought up: using published compendia of Cenozoic biome reconstructions, or analysis of Cenozoic fossil plant record. Incidentally, Rick Ree currently has a post-doc who may potentially have a lot of information for the latter. Peter Smits will be meeting with Rick and his post-doc in the near future to discuss this and other research questions.

Finally, Peter will be working on better ways of presenting the results of this analysis as presentation is currently unoptimised.

% future
%   estimated spring/summer 2017 finish time
%   how can/should i market myself?
%     evolutionary biologist/biology departments
%       (outside chance paleobiologist in geology department)
%     issues surrounding always having to prove myself in either
%     \uppercase{questions person not analysis person}
%       don't be a hired gun
%       lead with questions, context, why, hypotheses and \uppercase{never} the analysis
%     build a research program/identity. currently my identity is too quixotic and erratic (mercentary, even).
%   need to get list of potential post-doc people by spring quarter/next committee meeting
%     a short list was developed with Slater and Foote
%     have it resonate with your questions! 
%     build a research program/identity. currently my identity is too quixotic and erratic.
The final section of the committee meeting was discussion of the future. Rick and Ken had to leave by this point, so the only committee members present were Graham and Michael. Discussion was focused around what requirements there are going forward, what should be ready for the next committee meeting in Spring quarter 2016, when Peter Smits can be expected to finish, and how Peter Smits should approach marketing himself in terms of both post-doc positions and future jobs. Graham, Michael, and Ken (prior conversation) agree that Peter will most likely finish in Spring/Summer 2017 with four papetrs/chapters: the published PNAS one, the paper submitted to Evolution, and whatever comes of the two current projects. Peter Smits is expected to have prepared a list of potential post-doc positions to apply to in Fall 2016 which would then start in Fall 2017. 

\end{document}
