\documentclass{beamer} 
\usepackage{amsmath,amsthm}
\usepackage{graphicx,microtype,parskip}
\usepackage{caption,subcaption,multirow}
\usepackage{attrib}

\frenchspacing

\usetheme{default}
\usecolortheme{whale}

\setbeamertemplate{navigation symbols}{}

\setbeamercolor{title}{fg=blue,bg=white}

\setbeamercolor{block title}{fg=white,bg=gray}
\setbeamercolor{block body}{fg=black,bg=lightgray}

\setbeamercolor{block title alerted}{fg=white,bg=darkgray}
\setbeamercolor{block body alerted}{fg=black,bg=lightgray}

\AtBeginSection[]
{
  \begin{frame}
    \tableofcontents[currentsection]
  \end{frame}
}

\title{}
\author{Peter D Smits}
\institute{Committee on Evolutionary Biology, University of Chicago}
\date{}

\begin{document}

\begin{frame}
  \tableofcontents
\end{frame}

\section{Since last meeting}
\begin{frame}
  \begin{alertblock}{Since last meeting}
    \begin{itemize}
      \item Evolution 2015 talk
      \item GSA 2015 talk
      \item Chapter 1 published (PNAS)
        \begin{itemize}
          \item Effects of biotic traits on mammal species duration
        \end{itemize}
      \item Chapter 2 submitted (Evolution)
        \begin{itemize}
          \item Interplay between extinction intensity and selectivity in brachiopod extinction
          \item Submitted early October, still in review?
        \end{itemize}
      \item Did not submit DDIG
    \end{itemize}
  \end{alertblock}
\end{frame}


\begin{frame}
  \begin{block}{Review of possible chapter 1}
    \begin{itemize}
      \item Published in PNAS
      \item I took all of your comments very seriously and they really improved the paper.
        \begin{itemize}
          \item Rick for forcing on the phylo (didn't do figure, but made me use it).
          \item Ken and David for pushing about modern extinction risk.
          \item Michael and Ken for helping me write it in english.
        \end{itemize}
      \item Sorry I didn't send it to anyone except Michael and Ken.
    \end{itemize}
  \end{block}
\end{frame}

\begin{frame}
  \begin{block}{Review of possible chapter 2}
    \begin{itemize}
      \item Submitted to Evolution
      \item What my patterns of extinction in Australia project eventually turned into.
        \begin{itemize}
          \item Sorry about that.
          \item Primarily drive by sample size issues.
        \end{itemize}
      \item Sorry I didn't send it to anyone except michael and ken before submitting it.
        \begin{itemize}
          \item This is actually a really good time to get all of your comments!
        \end{itemize}
    \end{itemize}
  \end{block}
\end{frame}


\section{Current projects}
\subsection{Brachiopods}
\begin{frame}
  \frametitle{Regional patterns in the diversification of Paleozoic brachiopods}
  \begin{alertblock}{Question}
    How does differential taxonomic gain and loss contribute to regional (e.g. latitudinal) diversity?
  \end{alertblock}
\end{frame}

\begin{frame}
  \begin{block}{Motivation}
    \begin{itemize}
      \item latitudinal diversity gradients
        \begin{itemize}
          \item through lense of a diversification process
        \end{itemize}
      \item regional as opposed to global
        \begin{itemize}
          \item variation within regions may not match global pattern \\(more biologically relevant?)
          \item partial follow up to brachiopod survival work
        \end{itemize}
    \end{itemize}
  \end{block}
\end{frame}

\begin{frame}
  \frametitle{Brachiopod latitudinal diversity}
  \begin{center}
    \includegraphics[width=\textwidth,height=0.8\textheight,keepaspectratio=true]{figure/powell_2007}
  \end{center}

  \attrib{\small{Powell 2007 \emph{G. Eco. Biogeo.}}}
\end{frame}

\begin{frame}
  \frametitle{Variation in bioversity gradient}
  \begin{columns}
    \begin{column}{0.4\textwidth}
      \includegraphics[width=\textwidth,height=0.8\textheight,keepaspectratio=true]{figure/powell_2015_b}
    \end{column}
    \begin{column}{0.6\textwidth}
      \includegraphics[width=\textwidth,height=0.7\textheight,keepaspectratio=true]{figure/powell_2015_a}
    \end{column}
  \end{columns}

  \attrib{\small{Powell \textit{et al} 2015 \emph{Paleobio.}}}
\end{frame}

\begin{frame}
  \frametitle{``Modes'' of latitudinal diversity}
  \begin{columns}
    \begin{column}{0.55\textwidth}
      \includegraphics[width=\textwidth,height=0.8\textheight,keepaspectratio=true]{figure/powell_2007_a}
    \end{column}
    \begin{column}{0.5\textwidth}
      \includegraphics[width=\textwidth,height=0.8\textheight,keepaspectratio=true]{figure/powell_2007_b}
    \end{column}
  \end{columns}

  \attrib{\small{Powell 2007 \emph{G. Eco. Biogeo.}}}
\end{frame}

\begin{frame}
  \frametitle{Change in evenness + diversity}
  \begin{center}
    \includegraphics[width=\textwidth,height=0.8\textheight,keepaspectratio=true]{figure/powell_2015}
  \end{center}

  \attrib{\small{Powell \textit{et al} 2015 \emph{Paleobio.}}}
\end{frame}


\begin{frame}
  \frametitle{Model structure: Markov model}
  \begin{center}
    \includegraphics[width=\textwidth,height=0.8\textheight,keepaspectratio=true]{figure/mm_diagram}
  \end{center}
\end{frame}

\begin{frame}
  \frametitle{Model structure: hidden state}
  \begin{center}
    \includegraphics[width=\textwidth,height=0.8\textheight,keepaspectratio=true]{figure/hidden_state}
  \end{center}
\end{frame}

\begin{frame}
  \frametitle{Observed diversity}
  \begin{center}
    \includegraphics[width=\textwidth,height=0.8\textheight,keepaspectratio=true]{figure/obs_div}
  \end{center}
\end{frame}

\begin{frame}
  \frametitle{Estimated latent diversity}
  \begin{center}
    \includegraphics[width=\textwidth,height=0.8\textheight,keepaspectratio=true]{figure/true_div}
  \end{center}
\end{frame}

\begin{frame}
  \frametitle{Turnover probability}
  \begin{center}
    \includegraphics[width=\textwidth,height=0.8\textheight,keepaspectratio=true]{figure/turnover}
  \end{center}
\end{frame}

\begin{frame}
  \frametitle{Observation probability}
  \begin{center}
    \includegraphics[width=\textwidth,height=0.8\textheight,keepaspectratio=true]{figure/observation}
  \end{center}
\end{frame}
\begin{frame}
  \frametitle{Gain probability}
  \begin{center}
    \includegraphics[width=\textwidth,height=0.8\textheight,keepaspectratio=true]{figure/entrance}
  \end{center}
\end{frame}

\begin{frame}
  \frametitle{Loss probability}
  \begin{center}
    \includegraphics[width=\textwidth,height=0.8\textheight,keepaspectratio=true]{figure/extinction}
  \end{center}
\end{frame}

\begin{frame}
  \frametitle{Change in diversity}
  \begin{center}
    \includegraphics[width=\textwidth,height=0.8\textheight,keepaspectratio=true]{figure/est_diff}
  \end{center}
\end{frame}

\begin{frame}
  \begin{block}{Major assumptions}
    \begin{itemize}
      \item first-order Markov process
        \begin{itemize}
          \item can lead to some taxa existing longer than in actuality
        \end{itemize}
      \item any taxon can occur in any geographic unit \\independent of other units
      \item all of the above possibly controlled for by sampling parameter
        \begin{itemize}
          \item further assumes all times and places can be considered similar
        \end{itemize}
      \item possible direction 
        \begin{itemize}
          \item increase taxonomic and/or temporal scope
          \item more lattitudinal bands
        \end{itemize}
    \end{itemize}
  \end{block}
\end{frame}

\subsection{Mammals}
\begin{frame}
  \frametitle{Changes in Cenozoic mammal ecotype composition}

  \begin{alertblock}{Question}
    How do occurrence ratios of mammalian ecotypes change over time?
  \end{alertblock}
\end{frame}

\begin{frame}
  \frametitle{Environmental shift}
  \begin{center}
    \includegraphics[width=\textwidth,height=0.8\textheight,keepaspectratio=true]{figure/stromberg_na}
  \end{center}

  \attrib{\small{Stromberg 2005 \emph{PNAS}}}
  % Smits2015 implies that increases in ``specialists'' would be due to speciation
\end{frame}

\begin{frame}
  \frametitle{Possible link?}
  \begin{columns}
    \begin{column}{0.55\textwidth}
      \includegraphics[width=\textwidth,height=0.8\textheight,keepaspectratio=true]{figure/loco_diff_est}
    \end{column}
    \begin{column}{0.55\textwidth}
      \includegraphics[width=\textwidth,height=\textheight,keepaspectratio=true]{figure/cohort_est}
    \end{column}
  \end{columns}

  \attrib{\small{Smits 2015 \emph{PNAS}}}
\end{frame}

\begin{frame}
  \begin{block}{Details and covariates}
    \begin{itemize}
      \item Interest is specifically change in \alert{composition}, \\and not taxonomic turnover.
      \item Covariates
        \begin{itemize}
          \item body size of taxon i
          \item dietary category of taxon i
          \item climate (dO18) of time bin j
        \end{itemize}
    \end{itemize}
  \end{block}
\end{frame}

% approach
%   multi-logit regression
%   K-1 regressions, softmax link function to categorical distribution
%     interpret like K-1 logistic regressions
%     K-1 because identifiability, only k = 1, \dots, K -1 because 
%     Kth regression parameter constrained to be 0


% foote will want me to figure out how to represent this without math
%   i have some drafts
\begin{frame}
  \frametitle{Multi-logit regression}
  \begin{equation*}
    \begin{aligned}
      y_{i} &\sim \mathrm{Categorical}(K, \pi) \\  % k = response, j = cohort, d = predictor
      \pi_{k} &= \frac{\exp(\beta_{k, j[i]} X_{i})}{\sum_{k = 1}^{K} \exp(\beta_{k, j[i]} X_{i})} \\ 
      &\text{ where } \beta_{K, j[i]} X_{i} = 0 \\
      \beta_{k, j} &\sim \mathcal{N}(\beta_{k}^{\prime}, \sigma_{k}) \\
      \beta_{k, j}[1] &\sim \mathcal{N}(\beta_{k}^{\prime}[1] + \alpha_{k} U_{k}, \sigma_{k}) \\
    \end{aligned}
  \end{equation*}
\end{frame}

\begin{frame}
  \frametitle{Change in occurrence ratio over time}
  \begin{center}
    \includegraphics[width=\textwidth,height=0.8\textheight,keepaspectratio=true]{figure/occurrence_prob}
  \end{center}
\end{frame}

\begin{frame}
  \frametitle{Effect of body size on occurrence probability}
  \begin{center}
    \includegraphics[width=\textwidth,height=0.8\textheight,keepaspectratio=true]{figure/trait_eff}
  \end{center}
\end{frame}

\begin{frame}
  \frametitle{Effect of dietary category on occurrence probability}
  \begin{center}
    \includegraphics[width=\textwidth,height=0.8\textheight,keepaspectratio=true]{figure/diet_eff}
  \end{center}
\end{frame}

\begin{frame}
  \begin{block}{Further developments}
    \begin{itemize}
      \item \uppercase{\alert{note}} currently single flat mean; allow trend/multiple?
        \begin{itemize}
          \item time order is not currently modeled
        \end{itemize}
      \item phylogenetic effect to be included (except k = K)
      \item climate as cohort-level predictor, integrating over uncertainty?
      \item observed taxa represent a proportional sample of reality
        \begin{itemize}
          \item how can this be overcome in a \alert{model based} framework?
        \end{itemize}
      \item limits to complexity of model due to sample size
    \end{itemize}
  \end{block}
\end{frame}


\section{Timeline}
\begin{frame}
  \begin{block}{Things to consider}
    \begin{itemize}
      \item TAing this spring and next year
      \item Funding?
        \begin{itemize}
          \item FMNH fellow (but I don't spend time at the museum).
        \end{itemize}
      \item Estimates for time of completion?
      \item Post-doctoral opportunities?
    \end{itemize}
  \end{block}
\end{frame}



\end{document}
