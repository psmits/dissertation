\abstract

Both macroevolution and macroecology are concerned with the patterns in evolutionary and ecological data, respectively, that arise when observing multiple species over time and/or space. Species traits are an operational tool for bridging the conceptual and analytical divide between macroevolution and macroecology. The macroevolutionary and macroecological patterns at the heart of this dissertation are species extinction, species pool functional composition. In each of these studies, hypotheses and analysis were framed in terms of how a species functional ecology can be associated with or shape these emergent patterns. 

In my first study, I analyzed the fossil record of North American mammals for the Cenozoic in order to test two long standing hypotheses of how species durations are structured: the survival of the unspecialized hypothesis, and the Law of Constant Extinction. Species duration was modeled as a function of multiple species traits, species' phylogenetic relatedness, and species' origination cohort. My results supported the conclusion that generalist mammal species will, on average, have a greater duration than more specialized mammal species. I also found that only some of factors associated with extinction risk for Modern mammals could be considered risk factors for mammals from the rest of the Cenozoic. Finally, I found evidence of species extinction risk increasing with species duration, a result that is counter the Law of Constant Extinction.

My second study also deals with the survival of the unspecialized hypothesis as well as the Law of Constant Extinction, this time with the global record of post-Cambrian Paleozoic brachiopods. In addition to these hypotheses, I also analyzed the relationship between extinction intensity and the strength of trait selection. I found support for greater survival among environmental generalists than specialists. I also found a negative correlation between intensity of extinction and the effects of geographic range and environmental preference. These results support the conclusion that as extinction intensity increases, the selective difference of traits increases. This study provides evidence for the qualitative evidence for the difference between background and mass extinction.

The final study of this dissertation is an analysis of how the functional composition of the North American mammal regional species pool has changed over time and in response to a changing environmental context. The goals of this analysis are to understand when different ecotypes enriched or depleted, and to undestand how changes to environmental context may shape the functional diversity of the regional species pool. My results add considerable nuance to the taxon-focused narrative of North American environmental change. I find that mammal diversity is more strongly shaped by changes to origination rate among a few ecotypes rather than being driven by selective extinction one or more ecotypes. I also found that all arboreal ecotypes decline through out the Paleogene and disappear from the species pool by the Neogene. Additionally, I found that most herbivore ecotypes expand their relative contribution to functional diversity over time. 


My desire with this dissertation is present the types of analyses and results that are possible through a synthesis of macroevolution and macroecology. By using an extremely flexible and expressive modeling strategy, I have been able to translate precise scientific questions into statistical models. The first step to building any dialogue is to agree on a common language and I've emphasised an expressive statistical framework with which to phrase our analyses. My hope is that this approach to using paleontological data for answering questions about the processes underlying emergent patterns in evolutionary and ecological data fosters a stronger relationship between the disciplines of macroevolution and macroecology as well as paleontologists and neontologists.
