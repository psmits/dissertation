\abstract

Macroevolution and macroecology are concerned with the patterns in evolutionary and ecological data, respectively, which arise when observing multiple species over time and/or space. Species extinction and species pool functional composition are the macroevolutionary and macroecological patterns at the heart of this dissertation. The hypotheses and analyses in the three studies forming this dissertaiton were all framed in terms of how species functional traits can shape these emergent patterns. 

In my first study, I analyzed the Cenozoic fossil record of North American mammals to test two long standing hypotheses: the survival of the unspecialized hypothesis, and the Law of Constant Extinction. My analysis centers around a model of species duration as a function of multiple species traits, species' phylogenetic relatedness, and species' origination cohort. My results support the conclusion that generalist species will, on average, have a greater duration than more specialized species. I also find that species extinction risk increases with species duration, a result that is counter the Law of Constant Extinction. Additionally, I find that only some of the factors associated with extinction risk for Modern mammals could be considered risk factors for mammals from the rest of the Cenozoic, indicating a difference between the modern biodiversity crisis and ``normal'' extinction dynamics.

My second study also deals with the survival of the unspecialized and the Law of Constant Extinction, but focuses on a different system: post-Cambrian Paleozoic brachiopods. An additional aspect of this study is an analysis of the relationship between extinction intensity and the strength of trait selection. I find support for greater survival among environmental generalists than specialists. I also find evidence that for geographic range and environmental preference, as extinction intensity increases, the selective importance of these traits increases. This result is evidence for a qualitative difference between background and mass extinction.

The final study is an analysis of the changing functional composition of the North American mammal regional species pool over the last 65 million years. The goals of this analysis are to understand when functional groups are enriched or depleted, and how changes to environmental context may shape these changes. I find that mammal diversity is more strongly shaped by changes to origination probability rather than changes to extinction probability. I also find that all arboreal ecotypes declined throughout the Paleogene and disappeared from the species pool by the Neogene. Additionally, I found that most herbivore ecotypes expand their relative contribution to functional diversity over time.

My desire with this dissertation is to present the types of analyses and results that are possible through a synthesis of macroevolution and macroecology. The first step to building any dialogue is to agree on a common language and I've emphasised an expressive statistical framework with which to phrase our questions in a common tongue. My hope is that this studies serve as an example of how to use paleontological data to unite questions about the processes underlying macroevolutionary and macroecological patterns.
