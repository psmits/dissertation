\documentclass[12pt,letterpaper]{article}

\usepackage{amsmath, amsthm}
\usepackage{microtype, parskip}
\usepackage[comma,numbers,sort&compress]{natbib}
\usepackage{lineno}
\usepackage{docmute}
\usepackage{caption, subcaption, multirow, morefloats, rotating}
\usepackage{wrapfig}

\frenchspacing

\begin{document}

\section*{Introduction}

% Taxon occurrence as a function of both emergent biological traits and its environmental context?

% species pool concept
%   the set of species which can form communities in a region
%     local diversity is subset of species pool
%   what affects regional species pool composition?
%     species traits
%     environmental characteristics
%   demography
% goal 
%   understand how species traits and environmental factors effect species demography
% occurrence cube -- the basis hierarchical model
%   fundamental unit of ecology is the occurrence -- Gotelli
%   species, site, time
%   fourth-corner type problem
%     species, sites
%     differences between local species pools
%   this analysis
%     species, time
%     time gives info on ``true'' vs observed occurrence
%     differences in regional species pool through time
%     not modelling between site differences
% issues of defining hierarchy
%   effect of environmental context on occurrence as mediated by traits
%   effect of traits on occurrence as mediated by environmental context
%   these are different questions
%     the former gets asked more --> assembly
%     i'm working with the later --> demography
%   species occurrence i at time t --> is the unit of analysis
%     species is the individual-level
%     site is the first group-level
%     time is the second group-level
%     seems logical to me
%   (the combination of 4th corner and my model does exist) 
%     (i think it makes more sense in my framework)
%     (species occurrence i at sight j at time t)
%   does this relate to the row to column ratio of the datasets?
%     i have lots of species, few time points
%       occurrence matrix is sparse
%       species have durations


% system
%   Cenozoic mammals of NA
%   lots of things to think about
%     what specific hypotheses are of interest? 
%     why those ones?
% after \citep{Smits2015}
%   decrease in extinction risk over time
%     notice the pattern at the paleogene-neogene boundary
%   higher extinction risk in arboreal taxa; 2 possible situations
%     this effect is constant for all time
%     Paleogene-Neogene transition
%       neutral effect of arboreality during Paleogene
%       strong selection against arboreality during Neogene
%       Neogene effect is stronger than Paleogene effect
%       means that overall mean effect is closer to Neogene
%   former implies no appreciable demographic differences over Cenozoic
%   latter implies a difference in demography between Paleogene and Neogene
% events of interest
%   faunal shift at Paleogene-Neogene boundary
%     Oligo-Miocene boundary (Chattian--Aquitanian)
%     no known climatic events
%     from kind of modern to mostly modern species pool
%     (mulitple european events)
%   rise of grasses
%     janis
%     stromberg
%     (fortelius and europe)
%   other climate events? PETM, mid-miocene climatic optimum


% analysis of mammal genus fossil record for north america
%   time is discrete as 2-My bins
%   starting post-K/Pg boundary
%   (ignores quarternary)

% individual-level covariates
%   intercept term, varying by time
%   locomotor type/category
%       arboreal, digitigrade, plantigrade, unguligrade, fossorial, scansorial
%   dietary category
%     carnivore, herbivore, insectivore, omnivore
%   body size
%     rescale of log body mass

% group-level covariates
%   intercept
%   mean of temperatue estimate at time t
%   interquartile range of temperatue estimate at time t
%   plant community phase following Graham

% compare model assuming perfect sampling to model allowing for imperfect sampling
% imperfect sampling
%   two-state, discrete time hidden markov model with absorbing state
%     observation model for Royle and Dorazio
%     used a lot in paleobiology
%       Liow
%   a continuous-time analogue would be PyRate
%     strong assumptions about probability of observing over species duration
% issues surrounding model complexity
%   including covariates adds a lot of complexity
%   only doing approximate Bayesian inference (ADVI)
%   see methods section for longer discussion

% why consider phylogeny?
%   may or may not be an issue in community ecology/assembly/whatever the fuck
%     that's whatever
%   but i'm doing stuff over time
%     so we actually progress along the tree
%     much better change of phylogeny actually mattering
%       clade replacement in NA (carnivores, herbivores, etc)
%   so i'm just thinking of it as how it affects occurrence
%     GLMM approach is super effective for this purpose
%   note that i'm not actually modeling the diversification process
%     that's what a true J-S model (or PyRate) do
%     i'm just accounting for phylogenetic autocorrelation in species occurrence


% goals of this analysis
%   questions being addressed
%   how these questions are answered

\end{document}
