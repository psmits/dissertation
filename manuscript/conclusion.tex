\chapter{Conclusion} \label{ch:conclusion}


% major themes
%   macroevolution and macroecology together
%     unified by common interest in species traits
%   emergent patterns
%   species selection is more than just species selection
% minor themes
%   survival of the unspecialized
%   law of constant extinction
%   community assembly but not really 
%     i don't think of it as an assembly process
%     i think of it as an evolutionary process
%     species enter and leave all the time
%     in order to stay stable, the system must be resistant to this

Macroevolution and macroecology are disciplines devoted to explaining emergent patterns in evolutionary and ecological data. These disciplines are linked through the analysis of the distribution of trait values across time, space, and/or species \citep{Mcgill2006,Weber2017}. Emergent evolutionary and ecological patterns in time can require at least a million of years to observe \citep{Uyeda2011a}. Paleontological and phylogenetic data preserve aspects of these large scale temporal patterns. Paleontological data is unique however in being empirical observations of these dynamics while phylogenetic data only preserves the branching history leading to the diversity pattern exhibited by the tips.

In the studies presented as a part of this dissertation, I have emphasized functional traits. These are traits which directly relate to the way in which a taxon interacts with its environment \citep{Mcgill2006}. When these traits are defined for a species, they are called species traits \citep{Mcgill2006}; examples include species geographic range, average body size, trophic level, environmental preference, etc. Functional traits are an excellent window in macroevolution and macroecology because of their obvious selective importance; an organism which cannot interact with its environment is by definition maladapted. Additionally, by focusing on functional traits as well as the inclusion of multiple traits in analysis improves overall process-based inference because the ways in which species interact have been emphasized.


\section{Review of results}

All of the studies conducted here are analyses of fundamentally emergent patterns which are not reducible to properties of their constituents. In each of these studies, hypotheses and analysis are framed in terms of how a species functional ecology can be associated with or shape these emergent patterns. 

The emergent pattern at the heart of both the first and second studies (Chapters \ref{ch:death_taxa}, \ref{ch:preserve}) is species duration. A species endures because of the continued success of individuals of that species but the duration of that species is only knowable by integrating across all individuals.

The third study (Chapter \ref{ch:coping}) deals with a fundamentally different emergent pattern: the functional composition of a regional species pool. All species are assumed to belong to a single functional group, and the set of species present in a community determines the functional composition of that community. A regional species pool is the set of species present in all communities. Thus while the functional composition of a community depends on the set of interactors at that locality, the functional composition of a regional species pool depends on the possibility of those interactors being present in at least one constituent community.

In Chapter \ref{ch:death_taxa} I tested two long standing hypotheses of how species survival is structured: the survival of the unspecialized hypothesis CITATIONS, and the Law of Constant Extinction CITATIONS. I analyzed how the distribution of mammal species durations is affected by differences in multiple species traits, species' phylogenetic relatedness, and species' origination cohort. My results supported the conclusion that generalist mammal species will, on average, have a greater duration than more specialized mammal species. I also found that phylogeny and origination cohort contribute sub-equally to variation in species duration. Finally, I found evidence of species extinction risk increasing with species duration, a result that is counter the Law of Constant Extinction.

Chapter \ref{ch:preserve} also deals with the survival of the unspecialized hypothesis as well as the Law of Constant Extinction, this time with the global record of post-Cambrian Paleozoic brachiopods. In addition to these hypotheses, I also analyzed the relationship between extinction intensity and the strength of trait selection, namely ``do the selective differences between traits increase or decrease with average fitness increases or decreases?''


% review of chapters
%   chapter 1
%     setup and key questions
%       north american mammals
%       cenozoic
%       species duration as function of traits, phylo, time
%       survival of the unspecialized
%       law of constant extinction
%     results
%       generalists expected to have greater duration than specialists
%       extinction may be age dependent
%       phylo AND time are associated with differences in duration
%       body size not associated with differences in duration
%   chapter 2
%     setup and key questions
%       global brachiopods
%       post-crambrian paleozoig
%       species duration as function of traits, time
%         correlation in effect
%       survival of the unspecialized
%       law of constant extinction
%       as average fitness increases, what happens to the variance/differences in fitness?
%     results
%       when average duration high, traits matter less?
%       when average duration low, traits matter more?
%       generalist environmental preference is better than specialist
%   chapter 3
%     setup and key questions
%       north american mammals
%       cenozoic
%       species occurrence as function of functional group and enviro factors
%     results
%       can't just use one prob, need orig and surv to be diff
%       orig more important to diversifcation rate than ext
%         can even find source(s) of peaks!
%       evidence of correlation btwn ecotypes in origination probability
%       arboreal taxa decline through paleogene and become absent/rare by neogene
%       env context affects orig, not surv

% what have we learned?
%   everything is nuanced
%   nothing is simple and easy
%   we have to accept complexity in order to learn
%   this means making our methods reflect this complexity
%     the paleontological record is a very complicated beast
%     lets not be afraid to tame the beast!

% what's next?
