\chapter{Conclusion}

% conclusion
% how did the three chapters fit into the major themes?
%   what are the types of emergent patterns covered here?
%     species duration
%     geographic range
%     environmental preference
%     species traits (e.g. body size, diet, locomotor)
%   what are the theories dealt with here?
%     background extinction
%     survival of the unspecialized
%     law of constant extinction
%     species selection
%     species distribution modeling
%     maxent community assembly
% what does the future look like?
%   we need is a new way of thinking about our questions
%     biological organisation is both real and imagined
%     Gould talked about how we need to think hierarchically
%       his thoughts were wrt clades
%       levels of organisation
%       species selection
%     we need to increase our complexity
%   improving translation of questions into analysis
%     what is ``operationalization''?
%     thinking about data analysis
%     thinking about structured data


% major themes
%   macroevolution and macroecology together
%     unified by common interest in species traits
%   emergent patterns
%   species selection is more than just species selection
% minor themes
%   survival of the unspecialized
%   law of constant extinction
%   community assembly but not really 
%     i don't think of it as an assembly process
%     i think of it as an evolutionary process
%     species enter and leave all the time
%     in order to stay stable, the system must be resistant to this

% review of chapters
%   chapter 1
%     setup and key questions
%       north american mammals
%       cenozoic
%       species duration as function of traits, phylo, time
%       survival of the unspecialized
%       law of constant extinction
%     results
%       generalists expected to have greater duration than specialists
%       extinction may be age dependent
%       phylo AND time are associated with differences in duration
%       body size not associated with differences in duration
%   chapter 2
%     setup and key questions
%       global brachiopods
%       post-crambrian paleozoig
%       species duration as function of traits, time
%         correlation in effect
%       survival of the unspecialized
%       law of constant extinction
%       as average fitness increases, what happens to the variance/differences in fitness?
%     results
%       when average duration high, traits matter less?
%       when average duration low, traits matter more?
%       generalist environmental preference is better than specialist
%   chapter 3
%     setup and key questions
%       north american mammals
%       cenozoic
%       species occurrence as function of functional group and enviro factors
%     results
%       can't just use one prob, need orig and surv to be diff
%       orig more important to diversifcation rate than ext
%         can even find source(s) of peaks!
%       evidence of correlation btwn ecotypes in origination probability
%       arboreal taxa decline through paleogene and become absent/rare by neogene
%       env context affects orig, not surv

% what have we learned?
%   everything is nuanced
%   nothing is simple and easy
%   we have to accept complexity in order to learn
%   this means making our methods reflect this complexity
%     the paleontological record is a very complicated beast
%     lets not be afraid to tame the beast!

% what's next?
