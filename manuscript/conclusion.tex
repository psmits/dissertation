\chapter{Conclusion} \label{ch:conclusion}


% major themes
%   macroevolution and macroecology together
%     unified by common interest in species traits
%   emergent patterns
%   species selection is more than just species selection
% minor themes
%   survival of the unspecialized
%   law of constant extinction
%   community assembly but not really 
%     i don't think of it as an assembly process
%     i think of it as an evolutionary process
%     species enter and leave all the time
%     in order to stay stable, the system must be resistant to this

Macroevolution and macroecology are disciplines devoted to explaining emergent patterns in evolutionary and ecological data. These disciplines are linked through the analysis of the distribution of trait values across time, space, and/or species \citep{Mcgill2006,Weber2017}. Emergent evolutionary and ecological patterns in time can require at least a million of years to observe \citep{Uyeda2011a}. Paleontological and phylogenetic data preserve aspects of these large scale temporal patterns. Paleontological data is unique however in being empirical observations of these dynamics while phylogenetic data only preserves the branching history leading to the diversity pattern exhibited by the tips.

In the studies presented as a part of this dissertation, I have emphasized functional traits. These are traits which directly relate to the way in which a taxon interacts with its environment \citep{Mcgill2006}. When these traits are defined for a species, they are called species traits \citep{Mcgill2006}; examples include species geographic range, average body size, trophic level, environmental preference, etc. Functional traits are an excellent window in macroevolution and macroecology because of their obvious selective importance; an organism which cannot interact with its environment is by definition maladapted. Additionally, by focusing on functional traits as well as the inclusion of multiple traits in analysis improves overall process-based inference because the ways in which species interact have been emphasized.


\section{Summary}

All of the studies conducted here were analyses of fundamentally emergent patterns which are not reducible to the properties of their constituents. In each of these studies, hypotheses and analysis were framed in terms of how a species functional ecology can be associated with or shape these emergent patterns. Because of the complexity of processes which shape these emergent patterns, as well as the vagaries of the fossil record, each of these studies required the development of specific inference devices (e.g. statistical models) which attempt to estimate the actual quantities of interest to that analysis. 

The emergent pattern at the heart of both the first and second studies (Chapters \ref{ch:death_taxa}, \ref{ch:preserve}) is species duration. A species endures because of the continued success of individuals of that species but the duration of that species is only knowable by integrating across all individuals. The third study (Chapter \ref{ch:coping}) deals with a fundamentally different emergent pattern: the functional composition of a regional species pool. A regional species pool is the set of species present in all communities. While the functional composition of a community depends on the set of interactors at that locality, the functional composition of a regional species pool depends on the possibility of those interactors being present in at least one constituent community.

In Chapter \ref{ch:death_taxa} I tested two long standing hypotheses of how species durations are structured: the survival of the unspecialized hypothesis CITATIONS, and the Law of Constant Extinction \citep{VanValen1973}. I analyzed how the distribution of mammal species durations is affected by differences in multiple species traits, species' phylogenetic relatedness, and species' origination cohort. My results supported the conclusion that generalist mammal species will, on average, have a greater duration than more specialized mammal species. I also found that phylogeny and origination cohort contribute sub-equally to variation in species duration. Finally, I found evidence of species extinction risk increasing with species duration, a result that is counter the Law of Constant Extinction.

Chapter \ref{ch:preserve} also deals with the survival of the unspecialized hypothesis as well as the Law of Constant Extinction, this time with the global record of post-Cambrian Paleozoic brachiopods. In addition to these hypotheses, I also analyzed the relationship between extinction intensity and the strength of trait selection, namely ``do the selective differences between traits increase or decrease with average fitness increases or decreases?'' I found a negative correlation between intensity of extinction and the effects of geographic range and environmental preference. As with Chapter \ref{ch:death_taxa}, I also found support for greater survival among environmental generalists than specialists. These results supported the conclusion that, at least for Paleozoic brachiopods, as extinction intensity increases, the selective difference of traits increases. In this analysis this means that when average duration decreases (e.g. intensity is high) the effect of genus geographic range increases in magnitude and taxa which favor epicontinental environments are expected to have a greater duration than those which favor open-ocean environments. I also find that the change in magnitude of effect is expected to be greater for environmental preference than for geographic range as the overall effect of former have a much greater variance than that of the later.

The final study of this dissertation (Chapter \ref{ch:coping}) was an analysis of how the functional composition of the North American mammal regional species pool changed over time and in response to multiple environmental factors. The goals of this analysis were to understand when are different ecotypes enriched or depleted in the regional species pool, and to undestand how changes to environmental context may affect changes to the functional diversity of the regional species pool. By focusing on functional groups instead of taxonomic groups, the results from this study are phrased in terms of species interactions and not differences in clade diversity. My results add considerable nuance to the taxon-focused narrative of North American environmental change. There are many results and conclusions from this analysis, so I focus here on a few key results. I found that mammal diversity is more strongly shaped by changes to origination rate among a few ecotypes rather than being driven by selective extinction one or more ecotypes. I also found that all arboreal ecotypes decline through out the Paleogene and disappear from the species pool by the Neogene. Additionally, I found that most herbivore ecotypes expand their relative contribution to functional diversity over time. Finally, I found that the environmental factors analyzed here structure differences in ecotype origination probability but not survival probability.


% what have we learned?
%   everything is nuanced
%   nothing is simple and easy
%   we have to accept complexity in order to learn
%   this means making our methods reflect this complexity
%     the paleontological record is a very complicated beast
%     lets not be afraid to tame the beast!
\section{Synthesis}

The first two studies, when considered together, add a considerable degree of nuance to our understanding of multiple macroevolutionary hypotheses such as the Law of Constant Extinction and the survival of the unspecialized.

First and foremost, the results of neither study support the Law of Constant Extinction \citep{VanValen1973}. Instead, I found evidence for extinction risk increasing with taxon duration. Instead, these results are consistent with those of a ``nearly-neutral'' theoretical model of macroevolution \citep{Rosindell2015a}. While the dynamics of this model are described as ``Red Queen'' \citep{Rosindell2015a}, this is not strictly true as Red Queen dynamics as described by Van Valen \citep{VanValen1973} require that extinction does not increase with taxon age. Instead, the dynamics of the nearly-neural model are ``Red Queen'' in the sense that while all species are increasing in expected fitness, their relative fitnesses do not change. Interestingly, the decrease absolute in extinction risk towards the Modern \citep{Raup1982a,Foote2003}, which translates to an increase in expected species duration, may reflect similar dynamics to the nearly-neutral model.

The first two studies also provide broad support for the hypothesis of the survival of the unspecialized \citep{Simpson1944}, at least with respect to the covariates included in either model. The survival of the unspecialized appears to be a nearly universal pattern in macroevolution \citep{Simpson1944,Liow2004a,Liow2007b,Nurnberg2013a,Nurnberg2015,Baumiller1993,Raia2016}. I did find, however, that for post-Cambrian Paleozoic brachiopods when extinction intensity increases, the relationship with environmental preference and duration changes from pattern where intermediate environmental preference are favored to one where more specialized taxa from one end of the environmental spectrum are favored. This result adds a degree of nuance to the survival of the unspecialized, specifically with regards to when it is expected to ``hold.'' My conclusion is that during periods of low intensity extinction risk or so called background extinction CITATION, the survival of the unspecialized hypothesis will hold. However, as extinction intensity increases, this hypothesis may not accurately describe differences in extinction risk across taxa. As such, the survival of the unspecialized may serve as an excellent default for studies of trait selection and taxon extinction.





% what's next?
\section{Future}

