\chapter{Conclusion} \label{ch:conclusion}

Macroevolution and macroecology are disciplines devoted to explaining emergent patterns in evolutionary and ecological data. These disciplines are linked through the analysis of the distribution of trait values across time, space, and/or species \citep{Mcgill2006,Weber2017}. Emergent evolutionary and ecological patterns in time can require at least a million of years to observe \citep{Uyeda2011a}. Paleontological and phylogenetic data preserve aspects of these large scale temporal patterns. Paleontological data is unique however in being empirical observations of these dynamics while phylogenetic data only preserves the branching history leading to the diversity pattern exhibited by the tips. Additionally, phylogenies of only extant taxa are most likely unrepresentative of the evolutionary history of that clade; for example, multiple rounds of extinction and radiation can obscure the actual diversification process from possibly being inferred \citep{Mitchell2015a,Mitchell2014b,Liow2010a,Quental2009}.

In the studies presented as a part of this dissertation, I have emphasized functional traits. These are traits which directly relate to the way in which a taxon interacts with its environment \citep{Mcgill2006}. When these traits are defined for a species, they are called species traits \citep{Mcgill2006}; examples include species geographic range, average body size, trophic level, environmental preference, etc. Functional traits are an excellent window in macroevolution and macroecology because of their obvious selective importance: an organism which cannot interact with its environment is by definition maladapted. Additionally, the simultaneous analysis of multiple functional traints improves overall process-based inference because it more accurately accounts for the ways in which species interact. 


\section{Summary}

All of the studies conducted here were analyses of fundamentally emergent patterns which are not reducible to the properties of their constituents. In each of these studies, hypotheses and analysis were framed in terms of how a species functional ecology can be associated with or shape these emergent patterns. Because of the complexity of processes which shape these emergent patterns, as well as the vagaries of the fossil record, each of these studies required the development of specific inference devices (e.g. statistical models) which attempt to estimate the actual quantities of interest to that analysis. 

The emergent pattern at the heart of both the first and second studies (Chapters \ref{ch:death_taxa}, \ref{ch:preserve}) is species duration. A species endures because of the continued success of individuals of that species but the duration of that species is only knowable by integrating across all individuals. The third study (Chapter \ref{ch:coping}) deals with a fundamentally different emergent pattern: the functional composition of a regional species pool. A regional species pool is the set of species present in all communities. While the functional composition of a community depends on the set of interactors at that locality, the functional composition of a regional species pool depends on the possibility of those interactors being present in at least one constituent community. 

In Chapter \ref{ch:death_taxa} I tested two long standing hypotheses of how species durations are structured: the survival of the unspecialized hypothesis \citep{Simpson1944}, and the Law of Constant Extinction \citep{VanValen1973}. I analyzed how the distribution of mammal species durations is affected by differences in multiple species traits, species' phylogenetic relatedness, and species' origination cohort. My results supported the conclusion that generalist mammal species will, on average, have a greater duration than more specialized mammal species. I also found that phylogeny and origination cohort contribute sub-equally to variation in species duration. Additionally, I found that only some of factors associated with extinction risk for Modern mammals could be considered risk factors for mammals from the rest of the Cenozoic. For example, while there is an increased risk of extinction with increasing body size amoungst Modern fauna, this was not the normal case for Cenozoic. Finally, I found evidence of species extinction risk increasing with species duration, a result that is counter the Law of Constant Extinction.

Chapter \ref{ch:preserve} also deals with the survival of the unspecialized hypothesis as well as the Law of Constant Extinction, this time with the global record of post-Cambrian Paleozoic brachiopods. In addition to these hypotheses, I also analyzed the relationship between extinction intensity and the strength of trait selection, namely ``do the selective differences between traits increase or decrease with average fitness increases or decreases?'' I found a negative correlation between intensity of extinction and the effects of geographic range and environmental preference. As with Chapter \ref{ch:death_taxa}, I also found support for greater survival among environmental generalists than specialists. These results supported the conclusion that, at least for Paleozoic brachiopods, as extinction intensity increases, the selective difference of traits increases. In this analysis this means that when average duration decreases (e.g. intensity is high) the effect of genus geographic range increases in magnitude and taxa which favor epicontinental environments are expected to have a greater duration than those which favor open-ocean environments. I also find that the change in magnitude of effect is expected to be greater for environmental preference than for geographic range as the overall effect of former have a much greater variance than that of the later.

The final study of this dissertation (Chapter \ref{ch:coping}) was an analysis of how the functional composition of the North American mammal regional species pool changed over time and in response to multiple environmental factors. The goals of this analysis were to understand when are different ecotypes enriched or depleted in the regional species pool, and to undestand how changes to environmental context may affect changes to the functional diversity of the regional species pool. By focusing on functional groups instead of taxonomic groups, the results from this study are phrased in terms of species interactions and not differences in clade diversity. My results add considerable nuance to the taxon-focused narrative of North American environmental change. There are many results and conclusions from this analysis, so I focus here on a few key results. I found that mammal diversity is more strongly shaped by changes to origination rate among a few ecotypes rather than being driven by selective extinction one or more ecotypes. I also found that all arboreal ecotypes decline through out the Paleogene and disappear from the species pool by the Neogene. Additionally, I found that most herbivore ecotypes expand their relative contribution to functional diversity over time. Finally, I found that the environmental factors analyzed here structure differences in ecotype origination probability but not survival probability.


% what have we learned?
%   everything is nuanced
%   nothing is simple and easy
%   we have to accept complexity in order to learn
%   this means making our methods reflect this complexity
%     the paleontological record is a very complicated beast
%     lets not be afraid to tame the beast!
\section{Synthesis}

These three chapters are united by their analysis of species functional traits. Analysis of species traits is the easiest way to unite macroevolutionary and macroecological inference by using this commonality to develop and test integrated hypotheses \citep{Mcgill2006,Weber2017}. All inferences lead to more hypotheses as we realize what we do and do not know. Inferences from analysis of macroevolutionary patterns may lead to hypotheses about macroecological patterns, such as the relationship between extinction risk and diversity in arboreal mammals (Chapter \ref{ch:death_taxa}, and \ref{ch:coping}).

The first two studies, when considered together, add a considerable degree of nuance to our understanding of multiple macroevolutionary hypotheses such as the Law of Constant Extinction and the survival of the unspecialized.

First and foremost, the results of neither study support the Law of Constant Extinction \citep{VanValen1973}. Instead, I found evidence for extinction risk increasing with taxon duration. Instead, these results are consistent with those of a ``nearly-neutral'' theoretical model of macroevolution \citep{Rosindell2015a}. While the dynamics of this model are described as ``Red Queen'' \citep{Rosindell2015a}, this is not strictly true as Red Queen dynamics as described by Van Valen \citep{VanValen1973} require that extinction does not increase with taxon age. Instead, the dynamics of the nearly-neural model are ``Red Queen'' in the sense that while all species are increase in expected fitness, their relative fitnesses do not change. Interestingly, the decrease absolute in extinction risk towards the Modern \citep{Raup1982a,Foote2003}, which translates to an increase in expected species duration, may reflect similar dynamics to the nearly-neutral model. One of the innovations behind this result may be allowing species survival to decay non-linearly with duration by modeling durations as observations from a Weibull distribution as opposed to an Exponential distribution (Chapter \ref{ch:death_taxa}, and \ref{ch:preserve}), a modeling decision that is becoming more commonplace \citep{Crampton2016a,Ezard2012b}.

These two studies also provide broad support for the hypothesis of the survival of the unspecialized \citep{Simpson1944}, at least with respect to the covariates included in either model. The survival of the unspecialized appears to be a nearly universal pattern in macroevolution \citep{Simpson1944,Liow2004a,Liow2007b,Nurnberg2013a,Nurnberg2015,Baumiller1993,Raia2016}. I did find, however, that for post-Cambrian Paleozoic brachiopods when extinction intensity increases, the relationship with environmental preference and duration changes from pattern where intermediate environmental preference are favored to one where more specialized taxa from one end of the environmental spectrum are favored. This result adds a degree of nuance to the survival of the unspecialized, specifically with regards to when it is expected to ``hold.'' My conclusion is that during periods of low intensity extinction risk or ``background extinction'' \citep{Jablonski1986,Foote2007b}, the survival of the unspecialized hypothesis will hold. However, as extinction intensity increases, this hypothesis may not accurately describe differences in extinction risk across taxa. As such, the survival of the unspecialized may serve as an excellent default for studies of trait selection and taxon extinction. This conclusion can be considered evidence for the qualitative difference between background and mass extinction \citep{Jablonski1986}.

% third study
%   large scale environmental change
%     evidence of constant change w/o sudden restructuring
%   able to estimate changing relative diversity of functional groups
%     loss of arboreal types by end of paleogene
%     gain/constant rel. div. in non-arboreal herbivores
%   phrasing as functional groups really improves inference
%     also easier to understand (Pete Wagner said this about my PNAS paper)
%     taxonomy stands in for too much to make strong process-based inference
%     traits also allow a fitness/selection argument to be made
In the third study I analyzed how the functional composition of a regional species pool changes over millions of years (Chapter \ref{ch:coping}). While this pattern is macroecological, many of the hypotheses and questions encompassed by this study were generated from macroevolutionary analyses. For example, the result from the first study that arboreal taxa have a greater extinction risk than other mammal locomotor groups did not have an unambiguous explanation for if and how extinction risk could have changed over the Cenozoic: always present but high risk, or higher extinction risk in the Neogene compared to the Paleogene (Chapter \ref{ch:death_taxa}). The third study, in its analysis of the relative diversity of mammal ecotypes, was able to more fully resolve the previous results as I found that diversity of arboreal ecotypes declined through the Paleogene, becoming extremely rare or entirely absent from the species pool by the Neogene (Chapter \ref{ch:coping}). This result is much more nuanced than the either of two proposed processes (Chapter \ref{ch:death_taxa}) and speaks to the improved insights and inferences a unified macroevolutionary and macroecological research program can provide.

The third study also, methodologically, represents the strongest unification of macroevolutionary and macroecological approaches to inference (Chapter \ref{ch:coping}). The question at the heart of this study is ``when are mammal ecotypes enriched or depleted relative there average diversity'' is both macroevolutionary and macroecological, inference needed to represent this. The model at the center of this study was a combination of a birth-death process with a fourth-corner model from ecology. The birth-death process provided a mechanism for changes to species diversity over time while the fourth-corner model recast diversification in terms of the relative contribution of functional groups to the regional species pool. The results of this study could then be phrased in how well adapted the functional groups are to their changing environmental context.



% what's next?
\section{Future}

% consider nonlinear relationships between covariates and response
% what does density-dependent origination mean, biologically?
%   niche partioning
% spatial patterns to break the regional species pool into a set of local communities
There are many possible ways to expand on the analyses presented in this dissertation. There are also many unanswered questions which have been raised by each of these analyses which require future study.

A major limitation the fossil record of North American mammals is poor spatial resolution for the entire Cenozoic, something that restricts macroecological analyses (Chatter \ref{ch:coping}). While a regional species pool represents the set of all species present in a region, the individual dynamics communities give a much more complete idea of why species pool diversity changes. For example, I've been unable to estimate how changes to functional diversity of local communities scales up to the functional diversity of the regional species pool these communities are drawn from \citep{Harrison2008}. Given a fossil record with a high resolution spatial record, the types of analyses presented in Chapter \ref{ch:coping} could be expanded to incorporate the spatial dynamics of functional diversity. The Bayesian modeling framework used throughout this dissertation makes this imminently possible given the right dataset \citep{Banerjee2004}. The results of such an analysis would shed a lot of light on how the functional diversity of communities can vary spatially and how those communities respond differently to environmental change. A restricted analysis, both temporally and spatially, that focuses on a high resolution fossil record would be an appropriate place to start for this analysis; for example, the late Neogene of North America is extensively sampled as evidence by the MioMap, Neotoma, and FaunMap projects (http://www.ucmp.berkeley.edu/neomap/use.html) and may be an ideal stating point for this type of analysis.

Preservation of the fossil record is a pervasive issue in paleontology \citep{Foote1996e,Foote1997c,Foote1999a,Foote2001,Lloyd2011,Wang2016b}. The first two studies gloss over issues of preservation by either ignoring it (Chapter \ref{ch:death_taxa}) or including a sampling proxy as a covariate (Chapter \ref{ch:preserve}); these decisions were partially due to limitations of the models underlying both those studies. In contrast, in Chapter \ref{ch:coping} I directly modeled the preservation process and allowed preservation probability to vary with time and allowed species body size to potentially affect differences in preservation probability across taxa. However, I ignored the possible differences in preservation over time due to species functional group or changes to environmental context. I chose to limit the possible covariates which could affect of preservation because this type of analysis was beyond the scope of Chapter \ref{ch:coping}. 

All of three of these studies could be improved by incorporating more information about fossil preservation such as how functional groups and environmental context can shape differences in preservation probability. For example, when studying marine invertebrates covariates such as sea-level or shell composition (e.g. aragonitic vs calcitic) are all potentially very important for better understanding the differential preservation of species \citep{Foote2015b,Peters2002a,Peters2010,Hannisdal2011}. Given a more representative model of preservation, even more complete and accurate macroevolutionary and macroecological inferences can be made.



\section{Final thoughts}

The three studies presented in this dissertation are all representative of my rhetoric championing integrated macroevolutionary and macroecological study (Chapter \ref{ch:intro}). Each study emphasizes the importance of framing hypotheses in terms of ecological interactions in order to make strong inferences. By using an extremely flexible and expressive modeling strategy, I have been able to translate precise scientific questions into statistical models. The scope of inferences that can be made for each study are clear and conditioned on the explicit modeling assumptions made for each analysis. The first step to building a dialogue is agreeing on a common language and my strategy has been to emphasis a common statistical framework with which to phrase our analyses. My hope is that this approach to using paleontological data for answering questions about the processes underlying emergent patterns in evolutionary and ecological data fosters a stronger relationship between the disciplines of macroevolution and macroecology as well as paleontologists and neontologists.

