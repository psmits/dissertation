\chapter{Introduction} \label{ch:intro}

% introduction
% what are the major themes of the work covered by this dissertation?
%   macroevolution and macroecology
%     simultaneous interest in both
%   emergent patterns in paleontological data
%     extinction: differences in species fitness, species selection
%     regional species pool composition

Species traits are the bridge between evolution and ecology \citep{Mcgill2006,Weber2017}. A trait is an identifiable property of an organism, such as individual body size, while a species trait is an identifiable property of the entire species, such as the average body size or geographic range of a species \citep{Mcgill2006}. A class of species traits called functional traits are those traits which clearly describe a species means of interacting with its environment such as leaf surface area or trophic role \citep{Mcgill2006}. In macroevolutionary studies, analyses are typically focused on a patterns associated with a single single trait or are instead traitless a analysis of diversity and the diversification process \citep{Silvestro2014a,Silvestro2015b,Pires2015a,Rabosky2013,Slater2015c,Hunt2007a,Hunt2006a,Liow2008,Payne2007}. Macroecological studies are frequently concerned with describing the distribution of species and individuals over space or time, such as shifts in community composition along some gradient or axis \citep{Smith2008b,Smith2004,Brown1995,Brown1989,Bush2007,Eronen2009,Fortelius2002,Jernvall2002,Jernvall2004}. My desire with this dissertation is present the types of analyses and results that are possible through a synthesis of both macroevolution and macroecology; my approach is to develop inference devices (i.e. statistical models) to better understand the interactions between the effects of multiple species traits, as well as those of a species' temporal and environmental context, on diversity and differential diversification both in space and time.


\section{Emergent patterns, macroevolution, and macroecology}

An emergent pattern is one that is not observable or predictable from its individual constituent parts. Emergence is ubiquitous in biological systems: cells form a tissue with a complex function, species extinction requires all individual members to die for possibly unrelated reasons, and a species global geographic range is the product of many individual home ranges. The history of a species, or set of species, over time is inherently an emergent pattern as the temporal history of a species is not knowable from an instantaneous sample. Macroevolution and macroecology are the studies of emergent patterns in evolutionary and ecological data, respectively \citep{Brown1989,Brown1995,Stanley1979,Stanley1975}. Traditionally, macroevolution is the study of patterns over time while macroecology is concerned with patterns over space, but I find this division overly reductive.

Both macroevolution and macroecology are disciplines concerned with emergent patterns; they both implicitly and explicitly accept a hierarchical perspective on biology as emergence is not possible without different levels of organization, however they are defined. Even if an analysis is concerned with only a single species, the concept of a species is itself an emergent property or label for a collection of populations and individuals which all share a common evolutionary history. While it may be argued that the species label or identity is a non-biological construct or is simply heuristic for understanding the complexity of populations and reproductive isolation, we are still concerned with patterns associated with that construct as well as its intrinsic properties (e.g. extinction, conservation, ecosystem services) \citep{Coyne2004,Jablonski2008a}. My opinion is that any level with discernible emergent properties unique to it are potentially worth studying, though special attention to the species level most likely affords the greatest translation between paleontological and neontological studies. In the analyses presented in this dissertation the levels of organizations thare are studied are mammal species and brachiopod genera.

Macroevolution is much more than evolution above the species level \citep{Foote2007b}; this is overly reductive and assigns too much meaning to a single level of organization rather than embracing the multitudes of possible levels of organisation. Instead, I propose defining macroevolution, \textit{qua} field of study, as the study of emergent evolutionary patterns; this means patterns of speciation/extinction (diversity) as well as trait evolution (disparity), both of which are only observable when considering more than one taxon or when considering the temporal history of one or more taxa. In complement, macroecology is then the study of emergent ecological patterns, which means patterns in spatial distribution or community composition which are observable only when considering more than one taxon or when considering the temporal history of one or more taxa \citep{Brown1989,Brown1995,Smith2008b}. Because taxa inherently respond differently and individually to environmental changes, both biotic and abiotic, macroecological patterns are those due to the similarity in response across individuals \citep{Blois2009}.

Species selection is enshrined as one of the most important patterns in macroevolution \citep{Stanley1975,Stanley1979,Vrba1986,Jablonski2008a,Rabosky2010b,Simpson2016a,Pennell2014}. Rabosky and McCune \citep{Rabosky2010b} portray species selection as resultant phenomenon of the heritability of speciation and extinction rates. This definition is an expansion of which phenomenon fall under this category by divorcing the idea of levels of organization from species selection. The result of this distinction is to avoid the unproductive paleobiological debate surrounding species selection versus species sorting which has been the cause for a considerable amount of confusion and rhetorical fights \citep{Vrba1984a,Vrba1986,Lloyd1993,Pennell2014}, not the least of which is the adoption of ``sorting'' as an important term for understanding community assembly \citep{Urban2008,Loeuille2008,Holt2006,Cottenie2005,Soininen2014,VanderGucht2007,Shipley2006}. However, I think the definition presented by Rabosky and McCune \citep{Rabosky2010b} misses the mark as an operational definition to inspire and guide future study. Species inherit more than just speciation and extinction rates; they also inherit traits which themselves may be linked to differences in speciation or extinction rates due to their effects on species fitness. Species fitness is a concept that is rarely discussed and difficult to define yet is vitally important to understanding species selection as process \citep{Cooper1984,Palmer2012}. Here I adopt an inclusive definition of fitness: species fitness is, minimally, the expected duration of that species \citep{Cooper1984}. Differences in expected species duration that are associated with species traits are then the product of (species) selection.

A generative definition of species fitness (i.e. relating to speciation rate) is more difficult to develop for a variety or reasons, not the least of which being that the why and how speciation rates can vary both across species and time is not well understood \citep{Rabosky2015c,Rabosky2013e,Coyne2004}. At a minimum, given the earlier definition of fitness and its relation to duration, species fitness with respect to speciation would require the association of differences speciation rate with one or more species traits; this difference in fitness is then the raw material for selection. The fundamental issue with this definition, however, is ``why does having more daughter species increase species fitness?'' In the case of duration, the link to individual fitness is obvious because for a species to persist the individuals of that species must be reproducing and continuing to exist. The species fitness in terms of duration is just the emergent property resulting from the distribution of individual fitnesses. The issue with the generative definition of species fitness presented above is that the daughter species do not have the same identity as their ``mother'' as both species have separate and distinct durations. The term of a species through time or a series of descendent species is a lineage. In effect, the above generative definition of fitness is actually one of lineage fitness; the ability for a lineage to persist in time.

Extinction is a property of, or a phenomenon affecting, species as it requires the death of all organisms within a species which do not have to all occur for the same reason \citep{Simpson2016a}. Extinction is a fundamentally emergent phenomenon that is the ultimate manifestation of selection; it is also central to macroevolutionary studies and the definition of (species) fitness used here \citep{Cooper1984}. Extinction is featured centrally in one of the few ``laws'' in macroevolution and paleobiology: the Law of Constant Extinction \citep{VanValen1973,Liow2011a}. This law states that a species risk of going extinct is independent of that species age \citep{VanValen1973,Liow2011a}, a conclusion reached via analysis of patterns of (higher) taxon survival patterns. The Red Queen hypothesis was proposed as a process that would result in the observation underlying the Law of Constant Extinction \citep{VanValen1973}, though it has obviously grown to have a life of its own \citep{Liow2011a}.

The functional composition of a community or species pool is a property of that unit; observing a single species at a locality does not reveal the functional composition of the community in which it interacts. The composition of a community or species pool in terms of functional groups is a community ecology exercise. Comparing the distribution of functional groups across communities or species pools is where community ecology and macroecology intersect \citep{Mcgill2006,Brown1995,Smith2008b}. In paleobiology, a successful means of classifying marine invertebrate functional groups has been a three dimensional classification scheme called an ``ecocube'' which uses consistently identifiable functional traits to label both possible and observed functional groups \citep{Bush2007,Bambach2007}. This approach also emphasizes the presence or absence of different functional groups and how functional diversity can change over time. It is this strategy that inspires the third study presented in this dissertation.



\section{Structured data and modelling emergent patterns}

%   analysis of structured data via hierarchical Bayesian models
%     structured data requires a structured model
%       this is under-appreciated
%     mobilize as much data as possible in a single model
%       large amounts of covariate information
%         focus on how species interact with their environment
%       multi-level model for multiple levels of organization


% inference devices
An inference device is a theoretical tool for improving our knowledge by processing new information and observations \citep{McElreath2016,Jaynes2003a}; this device has initial conditions describing what we know (e.g. nothing), mechanisms for updating this knowledge to reflect new information, and can then produce an updated ``picture'' that better represents our current knowledge as well as the uncertainty surrounding this knowledge. Each inference device has a specific and narrow purpose and functionality \citep{McElreath2016}; unless the mechanisms are similar, a device for processing the rate of imperfections in the manufacturing of widgets cannot process the queuing times of callers to a help line.  

We can think of the well known battery of statistical tests \citep{Sokal2011} as re-usable inference devices with very narrow utility; these are unmodifiable tools for handling very specific questions and data. All Bayesian statistical models act as inference devices because they fulfill the requirements described above: initial conditions, updating mechanism, and output as updated knowledge along with the uncertainty surrounding that knowledge \citep{McElreath2016,Jaynes2003a}. By developing a new model for each new question there is a precision of translation; the model actually reflects the questions at hand, something that is preferable to forcing questions and data to fit into pre-made inference devices (e.g. models, tests) that do not update knowledge in a means relevant to the actual question(s) of interest.

Structure occurs naturally in the collection of data. For example, imagine a drug trial that takes place across multiple hospitals. It is possible to consider the results from all hospitals in aggregate by ignoring the hospital labels; alternatively, these results can be considered individually by hospital. In a biological example, imagine the study of individuals within a single species that are collected from multiple locations. For many reasons, we might expect that individuals from the same location are more similar to each other than to individuals from other locations. The goal in the analysis of structured data through hierarchical or multi-level models is to leverage this structure into the analysis in order to improve estimation by having groupings share information about associated parameter estimates \citep{Gelman2013d,McElreath2016}. 

Two of the most important analytical approaches at the core of macroevolutionary study are the birth-death process for diversification in both discrete and continuous time \citep{Raup1973,Raup1985,Nee1992,Nee1994a,Nee2001,Nee2006b,Stadler2013b}, and the random walk heuristic for continuous trait evolution \citep{Raup1974a,Felsenstein1985b,Bookstein1987b,Gingerich1993,Roopnarine2001a,Roopnarine2003b,Roopnarine1999,Sheets2001,Hunt2006a,Hunt2007a}. All three of the studies covered by this dissertation make use of some variant to the birth-death process. The first two studies are analyses of extinction, which is a pure-death processes. The third study utilizes a discrete-time birth-death process to model species presence in a species pool.

Similar analytical foundations are harder to identify for macroecology as a whole, so instead I will focus on species distribution models (SDMs) as a powerful framework for understanding the distribution of species, both due to environmental factors \citep{Elith2009} and species traits \citep{Shipley2006}. SDMs are a class of models which attempt to operationalize the multitude of processes which result in the distribution of one or more species in both space and/or time. In effect, SDMs are a means of operationalizing the concept of a species' ``realized niche'' in order to understand the limits on a species distribution \citep{Elith2009}. Typically, SDMs are used to analyze the relationship between species presence at a locality and the environmental factors which characterize that locality. From this analysis, the possible distribution of a species in space can be then estimated and compared to the observed distribution of that species \citep{Elith2009,Austin2006,Phillips2006a}. The maximum entropy theory of community assembly, and its related model, view community assembly as an ecological sorting process where traits mediate the effects of environmental filters \citep{Shipley2006,Warton2015a}; also called ``community assembly via trait selection'' (CATS regression). By analyzing the composition of species at localities based on their traits, the strength and relative importance of the traits that most directly structure community composition be elucidated. 

Both of these approaches to analyzing species distributions can be united in a single fourth-corner model \citep{Warton2015a,Brown2014c}. The fourth-corner problem is an old problem in community ecology originating in the multivariate analysis literature: assuming species distribution is the result of functional traits interacting with environmental factors, how do we estimate which interactions are important and their relative strengths \citep{Legendre1997,Dray2008}? By phrasing the fourth-corner problem as a model based framework, results are much more easily interpretable and actually provide estimates of the effects of species traits and environmental factors instead of the simple significance provided from the older Monte Carlo based methodology \citep{Brown2014c,Jamil2013,Pollock2012,Pollock2015}.

The third study in this dissertation makes extensive use of this framework by casting the fourth-corner problem into an additional dimension: time. By combining the fourth-corner framework outlined above \citep{Warton2015a,Brown2014c} with the birth-death process used for modeling diversification into a single unified model of species occurrence through time as a function of both species traits and changing environmental context I've developed a powerful analytical bridge between macroecology and macroevolution.

I emphasize model-based approaches to analysis as well as question or study specific models because a common language is necessary for clear, coherent, and translatable results that actually relate to the question(s) at hand. Some of the greatest limits to paleobiological, macroevolutionary, and macroecological study are a lack of strong, mechanistic predictive theories that can be expressed mathematically. Some of the greatest strides in advancing discussions of macroevolutionary and macroecological theory disputes have come from translating verbal theory into mathematical and statistical models \citep{Raup1973,Nee1992,Felsenstein1985b,Hunt2006a,Hunt2007a,Shipley2006}. The complex realities of the biological processes which shape diversity are rarely integrated into paleobiological analyses of macroevolutionary and macroecological patterns. A move to a model-forward approach to paleobiology, heavily steeped in evolutionary and ecological theory, would be beneficial for advancement of theories in macroevolution and macroecology. 

Paleobiologists historically believe that neontologists ignore their approaches and insights into macroevolution and macroecology study and theory \citep{Sepkoski2009,Sepkoski2015}, but without a concerted effort to engage within the same theoretical framework and language when developing scientific questions and the related analytical tools (i.e. statistical models) this worry and resentment is all but preordained. Because the systems paleobiologists study are unknown to, have no direct impact on, or are contextualized wrt the systems studied by neontologists, a push towards unification and synthesis most likely has to begin with the paleobiological community; luckily, it appears that the neontological community is receptive paleobiological insight \citep{Fritz2013a}. The simplest and fastest way to begin not just a dialogue but a unification is by translating verbal macroevolutionary and macroecological theories from paleobiology into statistical models that are readable by all researchers, both paleontological and neontological.


\section{Study summaries}  % section needs expanding/restructuring

Each of the three studies that make up this dissertation involve developing a hierarchical model to describe structured data with the goal of making macroevolutionary and/or macroecological inference. The first two studies are decidedly macroevolutionary in bent as they are analyses of trait-based extinction patterns in mammals and brachiopods, respectively. The third study is an analysis of mammal species pool temporal dynamics and is of a strong macroecological bent, though makes use of a macroevolutionary model of diversification in order to describe species turnover.

The first study presented is an analysis of North American mammal species durations and trait-based extinction risk. This analysis is principally concerned with the long standing hypothesis of the ``survival of the unspecialized'' which states that average or generalist species are expected to have a greater duration than specialists or other extreme forms \citep{Simpson1944,Liow2004a,Liow2007b}. Species duration is an proxy for species extinction risk as species with a shorter duration experience a greater extinction risk than species with a long duration. In this study, differences in species extinction risk based on multiple functional traits are estimated while also taking into account time of species origination as well as its relative phylogenetic position. Finally, the possibility of species age affecting extinction risk is also considered because while the Law of Constant Extinction is extremely hard to ``test'' it has never definitively been proven \citep{VanValen1973,Liow2011a}.


The second study presented is also an analysis of species durations, but this time focuses on all post-Cambrian Paleozoic brachiopod genera. The question at the center of this study is ``what happens to the effects of functional traits on survival when average survival increases or decreases?'' Unlike the previously described study, which focused on the average effects of functional traits on survival, this study requires estimates of how the effects of functional traits vary through time. The key parameters are those of the correlation matrix of the effects of these traits on duration and the average duration of species originating at the same time. This study also has results relevant to the ``survival of the unspecialized'' wrt the effect of environmental preference on survival, and the Law of Constant Extinction by allowing survival to be a function of species duration.

As mentioned above, the third study presented is decidedly more macroecological in focus as it is an analysis of how a regional species pool changes over time due to species turnover and a changing environmental context. The fundamental question is ``when are certain ecotypes enriched or depleted wrt their diversity history?'' To that end, I analyze the set of North American mammals for the Cenozoic and the changing functional composition of that species pool from nearly the beginning of the Cenozoic to almost the very recent (64-2 million years ago). In this analysis, functional composition of the species pool is described as the relative diversity of 18 different mammal ecotypes which are defined for every species as its dietary and locomotor combination. The occurrence of an ecotype, both in terms of origination and survival, is modeled as a function of that species environmental context as described by the dominant plant groups in North America as well as global temperature estimates.

All three of these analyses feature a hierarchical Bayesian model developed explicitly for each study in order to clearly attempt to answer the questions at hand. Each of these studies exemplifies my earlier rhetoric of how to build and advance macroevolutionary and macroecological study and theory through the explicit phrasing of scientific questions, precision of translation from question to analysis, and the mobilization of domain specific knowledge to cast results both in terms of the system specific insights as well as the theoretical insights.
