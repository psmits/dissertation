\chapter{Introduction}

% introduction
% what are the major themes of the work covered by this dissertation?
%   macroevolution and macroecology
%     simultaneous interest in both
%   emergent patterns in paleontological data
%     extinction: differences in species fitness, species selection
%     regional species pool composition

Species traits are the bridge between evolution and ecology \citep{Mcgill2006,Webber2017}. A trait is an identifiable property of an organism, such as individual body size, while a species trait is some identifiable property of the entire species, such as the average body size of a species \citep{Mcgill2006}. A class of species traits called functional traits are those traits which describe a species means of interacting with their environment \citep{Mcgill2006}. In macroevolutionary studies, analyses are typically focused on a patterns associated with a single single trait or are instead traitless a analysis of diversity and the diversification process. Macroecological studies are frequently concerned with describing the distribution of species and individuals over space or time, such as shifts in community composition along some gradient or axis. My desire has been to synthesize macroevolution and macroecology through the study of the effects of multiple traits on diversity and differential diversification, both in space and time.



\section{Emergent patterns, macroevolution, and macroecology}

An emergent pattern is one that is not observable or predictable from its individual constituent parts. Emergence is ubiquitous in biological systems: cells form a tissue with a complex function, species extinction requires all member individuals to die, and a species global geographic range is the product of many individual ranges. This history of a species or set of species over time is inherently an emergent pattern as the temporal history of a species is not knowable from an instantaneous sample. Macroevolution and macroecology are the studies of emergent patterns in evolutionary and ecological data, respectively \citep{Brown1989,Brown1995,Stanley1979,Stanley1975}. Traditionally, macroevolution is the study of patterns over time while macroecology is concerned with patterns over space, but I find this overly reductive.

Both macroevolution and macroecology are disciplines concerned with emergent patterns. Both disciplines implicitly and explicitly accept a hierarchical perspective on biology as emergence is not possible without different levels of organization, however nebulous they may be. Even if an analysis is concerned with only a single species, the concept of a species is itself an emergent property or label for a collection of populations and individuals. And while it may be argued that the species label or identity is a non-biological construct or heuristic for understanding the complexity of populations and reproductive isolation, we are still concerned with patterns and properties of that construct (e.g. extinction, conservation, ecosystem services).

Macroevolution is much more than evolution above the species level; this too is overly reductive and assigns too much meaning to a single level of organization. Macroevolution is the study of any emergent evolutionary pattern; this means patterns of speciation/extinction (diversity) or trait evolution (disparity) which are observable only when considering more than one taxon or when considering the temporal history of one or more taxa.

Macroecology is then the study of emergent ecological patterns, which means patterns in spatial distribution or community composition which are observable only when considering more than one taxon or when considering the temporal history of one or more taxa. Because taxa inherently respond differently and individually to environmental changes, both biotic and abiotic, macroecological patterns are those due to the similarity in response across individuals.




Species selection is enshrined as one of the most important patterns in macroevolution \citep{Stanley1975,Stanley1979,Vrba1986,Jablonski2008a,Rabosky2010b,Simpson2016a,Pennell2014}. \citet{Rabosky2010b} portray species selection as the resultant pattern caused by the heritability of speciation and extinction rates, but I think this misses the mark. Species inherit more than just speciation and extinction rates; they also inherit traits which themselves may be linked to differences in speciation or extinction rates due to their effect on species fitness. This last concept, species fitness, is a nebulous concept that is rarely discussed yet is vitally important to species selection \citep{Cooper1984,Palmer2012}. Here I adopt phrasing of \citet{Cooper1984} and I require that species fitness is, at a minimum, defined as the expected duration of a species. Differences in expected species duration that are associated with species traits are then the product of selection.

Defining a generative aspect of species fitness (i.e. relating to speciation rate) is more difficult partially because the why and how speciation rate can vary is not fully understood \citep{Rabosky2015c,Rabosky2013e}. At a minimum, given the earlier definition of fitness wrt extinction, species fitness wrt speciation would require the association with differences speciation rate and species traits; this difference in fitness is then the raw material for selection.

Extinction is the ultimate manifestation of selection and is central to macroevolutionary studies as well as the definition of species fitness used here. In his analysis of the geologic duration of higher taxa, \citet{VanValen1973} proposed the ``Law of Constant Extinction'' to explain the perceived pattern of age-independent extinction; stating that a species' risk of extinction does not change with taxon age. \citet{VanValen1973} proposed the ``Red Queen hypothesis'' as a mechanism behind this law.

The functional composition of a community or species pool.

% Simpson1944 % tempo and mode
% Simpson1953 % major features
% VanValen1971b % adaptive zones and mammal orders
% VanValen1973 % red queen hypothesis/law of constant extinction
% Raup1973 % random trees
% Raup1974a % random walk evolution
% Stanley1975 % species selection
% Stanley1979 % macroevolution book
% Felsenstein1985b % birth of pcm 
% Vrba1986 % sorting vs selection
% Brown1989 % macroecology paper
% Brown1995 % macroecology book
% Jablonski2008 % biotic interactions and scale/level
% Jablonski2008a % species selection review
% Smith2008a % macroecology expansion
% Rabosky2010b % biology reclaims species selection
% Simpson2016a % the case for species selection


\section{Structured data and modelling emergent patterns}

%   analysis of structured data via hierarchical Bayesian models
%     structured data requires a structured model
%       this is under-appreciated
%     mobilize as much data as possible in a single model
%       large amounts of covariate information
%         focus on how species interact with their environment
%       multi-level model for multiple levels of organization

Structured data is the hallmark of emergent patterns. Structure occurs naturally in the collection of data. A simple example would be the results of drug trials from multiple hospitals; it is possible to consider these results in aggregate by ignoring the effect of hospital, or these results can be considered individually by hospital. The goal in the analysis of structured, hierarchical, or multi-level data is to integrate this structure into the analysis. For biological data, a simple example would be studying the same species at multiple locations; individuals from the same location may be more similar than individuals from different locations. But incorporating this reality into the analysis, both estimates and predictions are improved.

I emphasize model-based approaches to analysis because a common language is necessary for clear, coherent, and translatable results. Some of the greatest limits to paleobiological, macroevolutionary, and macroecological study are a lack of strong, mechanistic predictive theories that can be expressed mathematically. Some of the greatest strides in macroevolutionary and macroecological studies wrt to advancing theory disputes have come from translating verbal theory to mathematical and statistical models \citep{Hunt2006a,Hunt2007,Shipley2006}. The reality of the complexity underlying biological process is rarely integrated into paleobiological analyses of macroevolutionary and macroecological patterns and processes. A move to a model-forward approach to paleobiology, heavily steeped in evolutionary and ecological theory, would be beneficial for paleobiology as well as the studies of macroevolution and macroecology. Too often paleobiologists malign their position as ignored by neontologists when discussing macroevolution and macroecology, without a concerted effort to utilize the same language when both phrasing questions and developing analytical tools (i.e. statistical models) then this worry and resentment is all but preordained. The simplest and fastest way to begin an actual dialogue is by translating verbal macroevolutionary and macroecological theories from paleobiology into statistical models that are readable by all researchers.

Two of the most important analytical approaches at the core of macroevolutionary study are the birth-death process for diversification \citep{Raup1973,Raup1985,Nee1992,Nee1994a,Nee2001,Nee2006b,Stadler2013b}, and the random walk heuristic for trait evolution \citep{Raup1974a,Felsenstein1985b,Bookstein1987b,Gingerich1993,Roopnarine2001a,Roopnarine2003b,Roopnarine1999,Sheets2001,Hunt2006a,Hunt2007a}. Discrete-state Markov processes also feature when modeling the evolution of discrete traits and biogeography CITATIONS.

All three of the studies covered by this dissertation make use of some variant to the birth-death process. The first two studies are analyses of extinction, which is a pure-death processes. The third study ultizes a discrete-time birth-death process to model species presence in a species pool.

Similar cores are harder to identify for macroecology as a whole, so here I identify species distribution models as a power framework for understading the distribution of species, both do to environmental factors \citep{Elith2009} and species traits \citep{Shipley2006}. Both of these approaches attempt to operationalize the multitude of processes which result in the distribution of one or more species in both space and time. The maximum entropy theory of community assembly view community assembly as an ecological sorting process where traits mediate the effects of environmental filters \citep{Shipley2006,Warton2015a}; also called ``community assembly via trait selection'' (CATS). Species distribution models or SDMs are a class of models which relate species geographic presence-absence distribution to the environmental factors which characterize those apperances \citep{Elith2009,Austin2006,Phillips2006a}. SDMs are a means of operationalizing a species' ``realized niche'' CITATION. These approaches can be united in a single fourth-corner model \citep{Warton2015a,Brown2014c}. The fourth-corner problem is an old problem in community ecology originating in the multivariate analysis literature: assuming species distribution is the result of functional traits interacting with environmental factors, how do we estimate which interactions are important and their relative strengths \citep{Legendre1997,Dray2008}? Advances in SDMs, along with advances surrounding CATs regression \citep{Renner2013}, have made this very possible and interpretable \citep{Brown2014c,Jamil2013,Pollock2012,Pollock2015}.

The third study in this dissertation makes extensive use of the macroecological framework by casting the fourth-corner problem into an additional dimension: time. The combination of clear macroecological theory and modelling approaches with the birth-death process at core of diversification into a single unified model in order to understand how species pool functional composition changes over time is a powerful bridge between macroecology and macroevolution.



\section{Study summaries} 

Each of the chapters that make up this dissertation feature structured data and a hierarchical model describing this structure. The first two chapters have a decidedly macroevolutionary bent while the third has a macroecological bent. However, all three are informed by both evolution and ecology; after all, the focus is of all studies is the relationship between functional traits to emergent patterns.




The first chapter is an analysis of North American mammal species duration wrt multiple species traits, origination cohort, and shared evolutionary history (i.e. phylogeny). All of these factors are emergent properties of a species, with origination cohort and shared evolutionary history also being data structuring factors. Additionally, the possibility of species age affecting extinction risk is also considered as a possible test of the Law of Constant Extinction \citep{VanValen1973}. This analysis is also concerned with the long standing ``survival of the unspecialized'' hypothesis wrt species duration/extinction risk \citep{Simpson1944}.

The second chapter is an analysis of post-Cambrian brachiopod genus duration wrt multiple species traits and origination cohort. As with the first chapter, all of these factors are emergent properties of species, and origination cohort is used as a structuring factor. Similarly, the possibility of age-dependent extinction is allowed. What distinguishes this study from the first is the additional focus on the relationship between the selective importance of these emergent factors and the higher-level average fitness of an origination cohort.

The third chapter is an analysis of the functional composition of the North American mammal regional species pool over time with the goal of understanding when are specific ecotypes enriched or depleted and how these patterns may be associated with species' environmental context. The structuring factors in this analysis are both the mammal ecotypes and the temporal units of the discrete time-series describing species presence and absence in the regional species pool. This analysis presents a multi-level model where the properties of the structuring factor (e.g. environmental context) are included as covariates affecting the emergent ecotype-specific pattern. 



%   chapter 1: death and taxa
%     Cenozoic mammal species survival
%     model of species duration and relationship with
%       diet, locomotor categories
%       geographic range
%       body size
%       phylogeny
%       origination cohort       
%     also allow species age to affect extinction risk (Weibull distribution)
%     what structures differences in mammal species duration?
%   chapter 2: interplay between selection and intensity
%     post-Cambrian Paleozoic brachiopod genus survival
%     model of species duration and relationship with
%       environmental preference (non-linear)
%       geographic range
%       body size
%     also allow species age to affect extinction risk (Weibull distribution)
%     allow effect of covariates to vary with time
%       estimate correlation between changes in extinction intensity and selectivity
%     ``if average extinction risk increases, how are the selective differences associated with covariates expected to change (given the model)?''
%   chapter 3: species pool dynamics
%     regional species pool Cenozoic mammals of North America
%     model of species presence as a function of
%       ecotype
%       environmental context
%         global temperature
%         regional plant context
%       pure-presence vs birth-death
%     how does the set and relative abundance of species ecologies change given
%       species turnover
%       changing environmental context

% rejoinder
%   all three chapters deal with emergent phenomenon
%     two aspects of extinction
%       differences species fitness; time vs phylogeny
%       interplay between intensity and selectivity
%     many million year dynamics of a regional species pool, itself emergent
%   i use a clear, extremely flexible and powerful means of modeling structured data
%     clearly stated parameteric models of macroevolution and macroecology
%       all assumptions are obviously stated (and known by author prior to interpretation)
%     strong translational fidelity between question(s) and model
%       improves downstream inference, especially in these large model


