\chapter{Introduction}

% introduction
% what are the major themes of the work covered by this dissertation?
%   macroevolution and macroecology
%     simultaneous interest in both
%   emergent patterns in paleontological data
%     extinction: differences in species fitness, species selection
%     regional species pool composition
%   analysis of structured data via hierarchical Bayesian models
%     structured data requires a structured model
%       this is under-appreciated
%     mobilize as much data as possible in a single model
%       large amounts of covariate information
%         focus on how species interact with their environment
%       multi-level model for multiple levels of organization
% emergent patterns
%   what is an emergent pattern?
%     property of complex systems with multiple levels of organization
%   species selection
%     differences in species fitness associated with emergent properties of the species
%     selection causes differences in fitness
% analysis of structured data
%   what is structured data and why should we care about it?
%     structred data is the hallmark of emergent patterns
%   what is a hierarchical Bayesian model?

% the studies conducted as a part of my dissertation
%   chapter 1: death and taxa
%     Cenozoic mammal species survival
%     model of species duration and relationship with
%       diet, locomotor categories
%       geographic range
%       body size
%       phylogeny
%       origination cohort       
%     also allow species age to affect extinction risk (Weibull distribution)
%     what structures differences in mammal species duration?
%   chapter 2: interplay between selection and intensity
%     post-Cambrian Paleozoic brachiopod genus survival
%     model of species duration and relationship with
%       environmental preference (non-linear)
%       geographic range
%       body size
%     also allow species age to affect extinction risk (Weibull distribution)
%     allow effect of covariates to vary with time
%       estimate correlation between changes in extinction intensity and selectivity
%     ``if average extinction risk increases, how are the selective differences associated with covariates expected to change (given the model)?''
%   chapter 3: species pool dynamics
%     regional species pool Cenozoic mammals of North America
%     model of species presence as a function of
%       ecotype
%       environmental context
%         global temperature
%         regional plant context
%       pure-presence vs birth-death
%     how does the set and relative abundance of species ecologies change given
%       species turnover
%       changing environmental context

% rejoinder
%   all three chapters deal with emergent phenomenon
%     two aspects of extinction
%       differences species fitness; time vs phylogeny
%       interplay between intensity and selectivity
%     many million year dynamics of a regional species pool, itself emergent
%   i use a clear, extremely flexible and powerful means of modeling structured data
%     clearly stated parameteric models of macroevolution and macroecology
%       all assumptions are obviously stated (and known by author prior to interpretation)
%     strong translational fidelity between question(s) and model
%       improves downstream inference, especially in these large model


An emergent pattern is one that is not observable or predictable from its individual constituent parts. Emergence is ubiquitous in biological systems: cells form a tissue with a complex function, species extinction requires all member individuals to die, and a species global geographic range is the product of many individual ranges. 

The ubiquity of emergent patterns also speaks to the importance of multiple levels of organization; the reality that processes can be driven by or affect emergent patterns. An example of this type of interaction is species selection, where differences in fitness between species is causes by an emergent property and not the properties of the individuals within that species.

Emergent patterns are the hallmark of complex systems.

Emergent patterns and both macroevolution and macroecology

Emergent patterns in biology

Emergent patterns in paleobiology


