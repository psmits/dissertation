%%%%%%%%% PROJECT DESCRIPTION  -- 15 pages (including Prior NSF Support)

\required{Project Description}
\begin{center}
\emph{Maximum of 15 pages}
\end{center}•
%The Project Description (including Results from Prior NSF Support, which is
%limited to five pages) may not exceed 15 pages. Visual materials, including charts,
%graphs, maps, photographs and other pictorial presentations are included in the
%15-page limitation. PIs be cautioned that the project description must
%be self-contained and that URLs that provide information related to the proposal
%should not be used. \\
%
%All proposals to NSF are reviewed utilizing the two merit review criteria,
%intellectual merit and broader impacts. \\
%
% The Project Description should provide a clear statement of the work 
% to be undertaken and must include: objectives for the period of the proposed 
% work and expected significance; relation to longer-term goals of the PI's 
% project; and relation to the present state of knowledge in the field, 
% to work in progress by the PI under other support and to work in progress 
% elsewhere.


\required{Broader Impacts}
% As in the project summary, broader impacts must be called out separately 
% in the project description.  You may be able to give more specific
% examples, or discuss how you've previously achieved these impacts.
% It should be similar, but not identical, to the Broader Impacts statement
% in the project summary

\required{Results From Prior NSF Support}
% 5 pages or fewer of the 15 pages for entire description document.
% include results from NSF grants received in the past 5 years.
% If supported by more than one grant, choose the most relevant one.

% For each grant, include: 
%	(a) NSF award number, amount, dates of support 
%	(b) The title of this project
%	(c) Publications resulting from this research
%	(d) Summary of the results of the completed work
%	(e) A brief description of data samples available and other research products not described 	      elsewhere
%	(f) For renewed support, a description of the relationship between the completed and 			      proposed work

% Due to space limitations, it is often advisable to use citations rather
% than putting the titles of the publications in the body 
% of this section
