\setcounter{page}{1}
\renewcommand{\thepage}{Data Management Plan - Page \arabic{page} of 1}
\required{Supplementary Documentation}
\required{Data Management Plan}
\begin{center}
\emph{Maximum of 2 pages}
\end{center}•
% Maximum of 2 pages
%------------------------------

% This supplement should describe how the proposal will conform to NSF policy on
% the dissemination and sharing of research results and may include:
% 
% 1. The types of data, samples, physical collections, software, curriculum
%    materials, and other materials to be produced in the course of the project;
% 
% 2. The standards to be used for data and metadata format and content (where
%    existing standards are absent or deemed inadequate, this should be documented
%    along with any proposed solutions or remedies);
% 
% 3. Policies for access and sharing including provisions for appropriate
% protection   of privacy, confidentiality, security, intellectual property, or
% other rights or   requirements;
% 
% 4. Policies and provisions for re-use, re-distribution, and the production of
%    derivatives; and
% 
% 5. Plans for archiving data, samples, and other research products, and for
%    preservation of access to them.
% 
% A valid Data Management Plan may include only the statement that no detailed
% plan is needed, as long as the statement is accompanied by a clear
% justification. Proposers who feel that the plan cannot fit within the supplement
% limit of two pages may use part of the 15-page Project Description for
% additional data management information. Proposers are advised that the Data
% Management Plan may not be used to circumvent the 15-page Project Description
% limitation.