\documentclass[11pt,letterpaper]{article}

\usepackage{amsmath, amsthm}
\usepackage{graphicx}
\usepackage{microtype, parskip}
\usepackage{bibentry}
\usepackage[numbers,sort&compress]{natbib}
\usepackage{docmute}

\frenchspacing

%%%%%%%%%%%%%%%%%%%%%%%%%%%%%%%%%%%%%%%%%%%%%%%%%%%%%%%%%%%%%%%%%%%%%%%%%
\pagestyle{plain}                                                      %%
%%%%%%%%%% EXACT 1in MARGINS %%%%%%%                                   %%
\setlength{\textwidth}{6.5in}     %%                                   %%
\setlength{\oddsidemargin}{0in}   %% (It is recommended that you       %%
\setlength{\evensidemargin}{0in}  %%  not change these parameters,     %%
\setlength{\textheight}{8.5in}    %%  at the risk of having your       %%
\setlength{\topmargin}{0in}       %%  proposal dismissed on the basis  %%
\setlength{\headheight}{0in}      %%  of incorrect formatting!!!)      %%
\setlength{\headsep}{0in}         %%                                   %%
\setlength{\footskip}{.5in}       %%                                   %%
%%%%%%%%%%%%%%%%%%%%%%%%%%%%%%%%%%%%                                   %%
\newcommand{\required}[1]{\section*{\hfil #1\hfil}}                    %%
\renewcommand{\refname}{\hfil References Cited\hfil}                   %%
\bibliographystyle{plainnat}                                           %%
%%%%%%%%%%%%%%%%%%%%%%%%%%%%%%%%%%%%%%%%%%%%%%%%%%%%%%%%%%%%%%%%%%%%%%%%%

\begin{document}
\setcounter{secnumdepth}{0}
\section{Context for improvement}

The proposed research is testing the following hypotheses: how does ecology affect the spatial distribution of taxa through time, is extinction non-random with respect to biology, and is extinction taxon age independent? By leveraging the mammal fossil record and integrating over these hypotheses, I am able ask the question ``why do species go extinct at different rates?'' 

I will be analyzing the effect of ecology on both spatial distributions and extinction risk in mammals using the Cenozoic fossil records of North America, Europe, and South America. By analyzing these patterns in a hierarchical Bayesian framework, both the individual continent and the underlying global patterns can be inferred simultaneously.

Previous large scale analyses of the mammal fossil record have been restricted primarily to North America and Europe. The inclusion of a third continent allows for stable and simultaneous estimates of both continental and global patterns. While I have access to a large collection of South American fossil mammal occurrence information, there is incomplete ecological information necessary for this study. By visiting collections in South America (i.e. Argentina) I will be able to measure specimens are build upon this dataset. 

My research (PDS) straddles the interests of both my advisor, Dr. Kenneth Angielzcyk, and my co-advisor, Dr. Michael Foote. Dr. Angielzcyk's research is focused on synapsids, a group of mammal-relatives, during the Permian and at the Permian--Triassic mass extinction. Dr. Foote's research is primarily focused on long-term evolutionary and spatial patterns of marine invertebrates. While I work on both spatio-temporal dynamics and extinction risk, my dissertation is independent from either of them because it focuses on the Cenozoic, mammals, the analytical framework I am applying, and questions I am asking. Because of this, my research is not an extension of any either advisor's existing projects and I have not received fiscal support from either of them.


\textbf{Timeline for the proposed DDIG study:}

\begin{tabular}[H]{l p{12cm}}
  Summer 2015 & Travel to Buenos Aires and La Plata, Argentina to measure specimens. \\
  Fall/Winter 2015-2016 & Perform analysis of both spatial and survival data. \\
  Summer 2016 & Present survival analyses at the Evolution 2016 meeting in Austin, Texas. \\
  Fall/Winter 2016-2017 & Present spatial analyeses at the 2016 Geological Society of America Meeting in Seattle, Washigton. Also prepare manuscripts for submission.
\end{tabular}

\end{document}
