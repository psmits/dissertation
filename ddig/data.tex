\section{Data management plan}

The major data and analytical products of the proposed project are 1) anatomical measurements of specimens from museum collections, 2) organized and updated ecological information, and 3) statistical analysis code. All information gathered will be stored indefinitely on both the PI's and Co-PI's personal computers. Additionally, they will be archived on an external hard drive indefinitely in case of the loss of either personal computer.

All anatomical measurements and ecological information used in the proposed study will be provided as supplementary material for all papers produced from this research. These data will also be archived using the data storage service Dryad (http://datadryad.org). Finally, all measurements and ecological information will be available through the Co-PI's personal website (http://home.uchicago.edu/psmits/home.html).

Museums and other institutions where specimens will be measured will be named in all subsequent presentations and papers. These institutions will also be provided with all measurements made to housed specimens, as well as reprints of all related papers.

Anatomical measures, body mass estimates, and updated ecological information will be sent to the Paleobiology Database (http://paleobiodb.org) which is the largest repository of palentological taxonomic, occurrence, and ecological information. 

All code used in the proposed analyses will be archived using Dryad, along with the relevant data as discussed above. Code will also be made available through the Co-PI's GitHub page (http://github.com/psmits), a free code sharing and archiving service, as well as through the Co-PI's website (http://home.uchicago.edu/psmits/home.html).

The Co-PI will present the results of the proposed research at conferences and publish said results in peer-reviewed journals in the fields of evolution, paleontology, evolutionary ecology, and global conservation.
