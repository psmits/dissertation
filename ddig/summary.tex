\documentclass[11pt,letterpaper]{article}

\usepackage{amsmath, amsthm}
\usepackage{graphicx}
\usepackage{microtype, parskip}
\usepackage{bibentry}
\usepackage[numbers,sort&compress]{natbib}
\usepackage{docmute}

\frenchspacing

%%%%%%%%%%%%%%%%%%%%%%%%%%%%%%%%%%%%%%%%%%%%%%%%%%%%%%%%%%%%%%%%%%%%%%%%%
\pagestyle{plain}                                                      %%
%%%%%%%%%% EXACT 1in MARGINS %%%%%%%                                   %%
\setlength{\textwidth}{6.5in}     %%                                   %%
\setlength{\oddsidemargin}{0in}   %% (It is recommended that you       %%
\setlength{\evensidemargin}{0in}  %%  not change these parameters,     %%
\setlength{\textheight}{8.5in}    %%  at the risk of having your       %%
\setlength{\topmargin}{0in}       %%  proposal dismissed on the basis  %%
\setlength{\headheight}{0in}      %%  of incorrect formatting!!!)      %%
\setlength{\headsep}{0in}         %%                                   %%
\setlength{\footskip}{.5in}       %%                                   %%
%%%%%%%%%%%%%%%%%%%%%%%%%%%%%%%%%%%%                                   %%
\newcommand{\required}[1]{\section*{\hfil #1\hfil}}                    %%
\renewcommand{\refname}{\hfil References Cited\hfil}                   %%
\bibliographystyle{plainnat}                                           %%
%%%%%%%%%%%%%%%%%%%%%%%%%%%%%%%%%%%%%%%%%%%%%%%%%%%%%%%%%%%%%%%%%%%%%%%%%

\begin{document}
\setcounter{secnumdepth}{0}
\section{Project summary}

\textbf{Overview}:
How does ecology affect species spatial distribution and temporal duration? Is extinction taxon age independent (i.e. Law of Constant Extinction)? The proposed research leverages the excellent mammal fossil record to model the effect of biological traits on both province structure and species survival. Using the fossil records from the Cenozoic of North America, Europe, and South America, Co-PI Smits will study 1) the relation between taxon ecology and changes in taxon provinciality and 2) the effect of ecology on taxon duration. Provinciality will be measured using a network theoretic approach. Taxon durations will be modeled in a survival analytical framework to directly estimate the effect of ecology and as well as taxon age on extinction risk. The ecological traits of interest are diet, locomotion, and body mass because of their direct relation to a species interactions with the environment. This grant will enhance these studies by funding the improving South American fossil mammal occurrence and ecological information beyond what is available in current database.

\textbf{Intellectual merit}:
I propose to analyze how ecology affects both spatial distribution and temporal duration of mammalian species. The mammal fossil record allows me to model complex relationships that are both more spatially and temporally explicit than normally possible. By estimating changes in the spatial heterogeneity of selection along with complementary ecologically explicit survival analyses, it may be possible to actually address the major question of ``why do species go extinct at different rates?'' 
Determining if extinction is due to randomness or the product of selection is impossible without information about the taxa. A more mechanistic approach where extinction is modeled via traits related to environmental preference, the relative importance of these scenarios can be elucidated. By tracking the number, membership, and similarity of communities over time and how this relates to taxon ecology, it should be possible to estimate spatial heterogeneity of selection pressure and how this may differ between ecological groups.

The Law of Constant Extinction, that extinction within an adaptive zone is taxon age-independent, underlies both the Red Queen hypothesis and many approaches to quantifying extinction. 
The survival analytical approach used here explicitly models the effect of both ecology and taxon age on survival, allowing the Law of Constant Extinction to be tested.

I have the unique opportunity to collaborate with Dr. Richard Madden and include his dataset of South American fossil occurrence information in the synthesis, which has been historically restricted to North America and Europe. The inclusion of South America into the analysis of mammalian diversification will provide a much needed Southern Hemisphere perspective and will greatly improve the generality of these analyses and conclusions. 


\textbf{Broader impacts}:
I have participated in multiple museum outreach programs. He will to continue with similar initatives at the Field Museum of Natural History such as the ``Meet a Scientist'' where I can directly engage the public about my research by presenting specimens in the middle of the museum. One of my goals to provide access to data collected on South American fossil mammals by making the data available for future research. Because most of the South American fossil mammal literature is in Spanish, it is effectively unavailable to many researchers, making its future availably extremely important for future syntheses. Related to this and pending funding, I plan to learn conversational Spanish so that I can engage with students and other researchers in South America (i.e. Argentina) about my project. 


\end{document}
