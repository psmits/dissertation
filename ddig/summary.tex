\documentclass[11pt,letterpaper]{article}

\usepackage{amsmath, amsthm}
\usepackage{graphicx}
\usepackage{microtype, parskip}
\usepackage{bibentry}
\usepackage[numbers,sort&compress]{natbib}
\usepackage{docmute}

\frenchspacing

%%%%%%%%%%%%%%%%%%%%%%%%%%%%%%%%%%%%%%%%%%%%%%%%%%%%%%%%%%%%%%%%%%%%%%%%%
\pagestyle{plain}                                                      %%
%%%%%%%%%% EXACT 1in MARGINS %%%%%%%                                   %%
\setlength{\textwidth}{6.5in}     %%                                   %%
\setlength{\oddsidemargin}{0in}   %% (It is recommended that you       %%
\setlength{\evensidemargin}{0in}  %%  not change these parameters,     %%
\setlength{\textheight}{8.5in}    %%  at the risk of having your       %%
\setlength{\topmargin}{0in}       %%  proposal dismissed on the basis  %%
\setlength{\headheight}{0in}      %%  of incorrect formatting!!!)      %%
\setlength{\headsep}{0in}         %%                                   %%
\setlength{\footskip}{.5in}       %%                                   %%
%%%%%%%%%%%%%%%%%%%%%%%%%%%%%%%%%%%%                                   %%
\newcommand{\required}[1]{\section*{\hfil #1\hfil}}                    %%
\renewcommand{\refname}{\hfil References Cited\hfil}                   %%
\bibliographystyle{plainnat}                                           %%
%%%%%%%%%%%%%%%%%%%%%%%%%%%%%%%%%%%%%%%%%%%%%%%%%%%%%%%%%%%%%%%%%%%%%%%%%

\begin{document}
\setcounter{secnumdepth}{0}
\section{Project summary}

\textbf{Overview}:
How does ecology affect species spatial distribution and temporal duration? Is extinction taxon age independent (i.e. Law of Constant Extinction)? The proposed research leverages the excellent mammal fossil record to model the effect of biological traits on both province structure and species survival. Using the fossil records from the Cenozoic of North America, Europe, and South America, Co-PI Smits will study 1) the relation between taxon ecology and changes in taxon provinciality and 2) the effect of ecology on taxon duration. Provinciality will be measured using a network theoretic approach. Taxon durations will be modeled in a survival analytical framework to directly estimate the effect of ecology and as well as taxon age on extinction risk. The ecological traits of interest are diet, locomotion, and body mass because of their direct relation to a species interactions with the environment. This grant will enhance these studies by funding the improving South American fossil mammal occurrence and ecological information beyond what is available in current database.

\textbf{Intellectual merit}:
This spatial and temporal resolution of the mammalian fossil record provides a unique opportunity to ask much more complex questions than is normally possible with the terrestrial record. The record of mammalian evolution is one of the corner stones of paleobiology, analysis has been historically restricted to North America and Europe. The inclusion of South America will greatly improve the generality of these analyses and conclusions. 
The Law of Constant Extinction, that extinction is taxon age independent, underlies virtually all statistical approaches for estimating extinction intensity. As evidence contrary to this law is becoming increasingly common, it is important to explicitly model and test this law's hypotheses. The approach outlined above is an explicit statistical framework for estimating the effect of taxon's age, as well as ecology, on extinction risk. 
The degree of species provinciality is associated the degree of spatial heterogenity in selective pressures. By quantifying the similarity of biotic provinces, analysis of changes in ecologically explicit degrees of species spatial heterogenity for whole continents can be accomplished. 
By averaging over the studies of spatial heterogeneity of selection and the relation between ecological and taxon duration, it may be possible to actually address a fundamental question of paleobiology: ``why do species go extinct at different rates?''


\textbf{Broader impacts}:
Co-PI Smits has participated in multiple museum outreach programs in the past. He will to continue with similar initatives at the Field Museum of Natural History such as the ``Meet a Scientist'' where I can directly engage the public about my research by presenting specimens in the middle of the museum. One of my goals to provide access to data collected on South American fossil mammals by making the data available for future research. Because most of the South American fossil mammal literature is in Spanish, it is effectively unavailable to many researchers, making its future availably extremely important for future syntheses. Related to this and pending funding, I plan to learn conversational Spanish so that I can engage with students and other researchers in South America (i.e. Argentina) about my project. 


\end{document}
