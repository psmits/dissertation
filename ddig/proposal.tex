\documentclass[11pt,letterpaper]{article}

\usepackage{amsmath, amsthm}
\usepackage{graphicx,hyperref}
\usepackage{microtype, parskip}
\usepackage{bibentry}
\usepackage[numbers,sort&compress]{natbib}
\usepackage{docmute}

\frenchspacing

%%%%%%%%%%%%%%%%%%%%%%%%%%%%%%%%%%%%%%%%%%%%%%%%%%%%%%%%%%%%%%%%%%%%%%%%%
\pagestyle{plain}                                                      %%
%%%%%%%%%% EXACT 1in MARGINS %%%%%%%                                   %%
\setlength{\textwidth}{6.5in}     %%                                   %%
\setlength{\oddsidemargin}{0in}   %% (It is recommended that you       %%
\setlength{\evensidemargin}{0in}  %%  not change these parameters,     %%
\setlength{\textheight}{8.5in}    %%  at the risk of having your       %%
\setlength{\topmargin}{0in}       %%  proposal dismissed on the basis  %%
\setlength{\headheight}{0in}      %%  of incorrect formatting!!!)      %%
\setlength{\headsep}{0in}         %%                                   %%
\setlength{\footskip}{.5in}       %%                                   %%
%%%%%%%%%%%%%%%%%%%%%%%%%%%%%%%%%%%%                                   %%
\newcommand{\required}[1]{\section*{\hfil #1\hfil}}                    %%
\renewcommand{\refname}{\hfil References Cited\hfil}                   %%
\bibliographystyle{plainnat}                                           %%
%%%%%%%%%%%%%%%%%%%%%%%%%%%%%%%%%%%%%%%%%%%%%%%%%%%%%%%%%%%%%%%%%%%%%%%%%

\title{\uppercase{Dissertation Research:}\\ Cenozoic mammals and the biology of extinction}
\author{PI: Kenneth D. Angielczyk, Co-PI: Peter D. Smits}
\date{}

\begin{document}
\setcounter{secnumdepth}{0}
\maketitle
\section{Introduction}
Why certain species go extinct while others do not is one of the most fundamental questions in paleobiology. It is expected that for the majority of geological time, extinction is biologically non-random \citep{Jablonski1986,Alexander1977,Harnik2011,Johnson2002b,Kitchell1986,Nurnberg2013a,Payne2007}. Determining which biological factors or traits influence extinction risk and how is vital for understanding the differential diversification of life during the Phanerozoic. Periods of background extinction also represent the majority of geologic time, remain relatively predictable and change slowly, and thus providing a better opportunity to study how traits are related to survival than periods of mass extinction \citep{Jablonski1986,Raup1988}. The Law of Constant Extinction \citep{VanValen1973} posits that extinction risk of taxa within a given adaptive zone is age independent (memoryless), however the generality of this statement is possibly suspect \citep{Drake2014,Raup1975,Sepkoski1975,Finnegan2008}. By analyzing survival patterns within adaptive zones during periods of background extinction, it should be possible to determine if extinction is best modeled as age independent or dependent.

I simple expectation based on purely stochastic grounds, where extinction is not selective, is that abundant and widespread taxa are less likely to go extinct that rare and restricted taxa \citep{Raup1991b}. For example, species with larger geographic ranges tend to have lower extinction rates than species with smaller geographic ranges \citep{Jablonski1986,Harnik2013,Nurnberg2013a,Jablonski2003,Roy2009c}. However, this common pattern does not explain why certain taxa may be less prone to extinction than others. In the example, how range size is formed varies between clades and thus remains a black box for most taxa \citep{Jablonski1987} and so determining if differential extinction is a purely stochastic process or is actually the product of selection is impossible. Instead, by focusing on traits related to environmental preference, the traits which may underly why a taxon is abundant or widespread, the process of selection underlying differential extinction may be elucidated.

It is under this framework that I propose to study how ecological traits associated with environmental preference have affected mammalian differential extinction. Cenozoic mammals represent an ideal group and time period as it is the fossil record is well sampled and the ecology and phylogeny of these taxa are generally understood. Mammals are motile organisms which can track their preferred environmental context over time and space. If a mammal species requires rare or fragile environmental conditions, or is a poor disperser, this would limit the availability of suitable environments or ability to track the preferred environment. Three important traits that describe the relationship between mammals and their environmental context are body size, dietary category, and locomotor category \citep{Smith2004,Smith2008b,Damuth1981a,Damuth1979,Jernvall2004,Lyons2005,Lyons2010}. Each of these traits describe different aspects of a taxon's adaptive zone such as energetic cost, population density, expected home range size, set of potential prey items, and dispersal ability among others. 

The fossil records of North American and Europe are frequently the subject of large spatial and temporal scale analyses of diversification \citep{Jernvall2004,Jernvall2002,Fortelius2002,Janis2000,Alroy1996a,Alroy1998,Alroy2000g,Liow2008,Raia2006,Tomiya2013}. These two continents are frequently used as proxies for global mammalian diversification even though there are known differences between them \citep{Liow2008,Tomiya2013}. The continent of South America, though well sampled for certain time periods, is rarely included and integrated into large spatial and temporal scale analyses \citep{Stromberg2013,Marshall1982}. My goal is to include the South American along with the North American and European mammal fossil records in order to better understand the different selective pressures on mammal distributions and survival.

%Environmental availability, along with stability, is crucial for both the establishment and persistence of a species.
%During the Cenozoic, primarily between the Paleogene--Neogene, there was a shift from a predominately closed environment to a predominately open environment \citep{Janis1993a,Blois2009,Rose2006}. 
%Additionally, with increased likelihood of prey item occurrence, abundance can increase \citep{VanValen1989,Brown1987,Damuth1979,Silva1997,Janis2000} which can effect both survival and increase occupancy \citep{Jernvall2004,Brown1984,Jernvall2002,Fortelius2002}.

Dietary categories are coarse groupings of similar dietary ecologies: carnivores, herbivores, omnivores, and insectivores. Because dietary category describes, roughly, the trophic position of a taxon and its related stability, it is predicted that more stable categories will have longer durations than less stable categories. Stability here being ``distance'' from primary productivity, thus it is expected that herbivores will have greater duration than carnivores. Omnivorous taxa are expected to have average taxon durations compared to the other two categories. If dietary category is not found to be important for modeling survival it may mean that trophic category is not a major factor for determining species level survival and that other factors, such as body size, may dominate.

Mammalian herbivores and carnivores have been found to have a greater diversification rate than omnivores \citep{Price2012} which may indicate that these traits are better for survival. However diversification can be caused either by an increase in origination relative to extinction or a decrease in extinction relative to origination. Which scenario occurred, however, is (currently) impossible to determine from a phylogeny of only extant organisms \citep{Rabosky2010a} which means that analysis of the fossil record is required. If survival is found to be similar between all dietary categories, this may mean that the differential diversification patterns observed by \citet{Price2012} are due to differences in speciation and not extinction.

Locomotor categories describe the motility of a taxon, the plausibility of occurrence, and dispersal ability. For example, an obligate arboreal taxon can only occur in locations with a minimum of tree cover and can most likely only disperse to other environments with suitable tree cover. Dispersal ability is important for determining the extent of a taxon's geographic range \citep{Birand2012,Jablonski2006a,Gaston2009} and affects both the taxon's extinction risk and regional community evenness. The categories of interest are arboreal, ground dwelling, and scansorial. With the transition from primarily closed to open environments, there is an expected shift in stability associated with arboreal and ground dwelling taxa. It is expected that arboreal taxa during the Paleogene will have a greater expected duration than Neogene taxa while the opposite will be true for ground dwelling taxa. In comparison, taxon duration of scansorial taxa is expected to remain relatively similar between the two time periods because it represents a mixed environmental preference that may be viable in either closed or open environments. If locomotor category is a poor descriptor of survival this may mean that it is either a poor descriptor of dispersal ability, which may or may not affect mammalian survival. 

An organisms body size, here defined as (estimated) mass, has an associated energetic cost in order to maintain homeostasis which in turn necessitates a supply of prey items. Many life history traits are associated with body size: reproductive rate, metabolic rate, home range size, among others \cite{Peters1983a,Damuth1979,Brown1987,Smith2004}. Body size can possibly affect species extinction because as body size increases, home range size increases \citep{Damuth1979}. If individual home range size scales up to reflect minimum total species geographic range, we would expect that taxa with larger body sizes would have lower extinction rates than species with smaller body sizes. This expectation, however, may not be right. An alternative but plausible scenario is where as body size increases, reproductive rate decreases \citep{Johnson2002b}, populations get smaller \citep{White2007}, and generations get longer \citep{Martin1993a} all of which can increase extinction risk, as has been observed \citep{Liow2008,Davidson2012}. However, the relationship between body size and extinction rate at the generic level has been found to vary between continents \citep{Tomiya2013,Liow2008}. By expanding to include a third continent, South America, and analyzing specific level data I hope to elucidate how differences in taxonomic diversity at a continental level might affect body size mediated extinction rate. If body size is found to be unimportant for modeling survival, as in the generic level analysis of \citet{Tomiya2013}, this means that other biotic or abiotic factors may dominate. This may also mean that individual level home range size does not scale to increased species level range size, and there is therefore no correlated decrease in extinction rate. If increase in body size increases extinction risk, this may be due to traits correlated with body size and not necessarily body size itself \citep{Johnson2002b}.

In addition to the individual effects, the interactions between body size, locomotor category, and dietary category may also be important for modeling taxonomic survival. For example, a small bodied arboreal taxon of any trophic category during the heavily forested and warm time of the Paleogene would be expected at once to have both a small body size determined range, a large potential geographic range determined by locomotion, as well as an increased availability of resources. Together this would mean that relative survival would be expected to be less than, greater than, and greater than average respectively. 


%%%%%
% pick up from here
%%%%%


To analyze differential mammalian survival, I propose a survival analysis approach similar to that described above for Permian brachiopods.  Mammalian occurrence data will be collected primarily through a combination of the PBDB, Neogene Old World Database (NOW; \url{http://www.helsink.fi/science/now/}), and museum collections. North American fossil mammal data are well represented in the PBDB because of the extensive work of Alroy \citep{Alroy1996a,Alroy1998,Alroy2000g}. European fossil mammal data is also well represented between the PBDB and NOW. South American fossil mammal data is available through the PBDB, but has poor overall coverage. Because of this, South American fossil mammal data will be gathered via various museums such as the Field Museum of Natural History and the American Museum of Natural History as well as published occurrence compilations. With the South American taxa, taxonomy and sampling may not be as well resolved as for North and South America and it may be necessary to restrict analysis to the most taxonomically resolved and sampled groups such as Notoungulata, Marsupials, Carnivora, and Primates.

As described above, duration will be measured as the difference between the observed FAD and LAD of every taxon. Taxa which originated prior to the Cenozoic and all taxa that are either extant or went extinct within 2 My of the present will be censored. This threshold is to limit the effect of the improved record of the Recent.

Dietary category, locomotor category, and body size will be considered constant throughout the duration of a taxon and will be modeled as time-independent covariates of survival. While body size is actually a distribution of values, it is quite common to use a single estimate of mean body size as an aggregate trait in studies of clade-wise dynamics \citep{Jablonski2008a}. While all three of these traits may have evolved over a taxon's duration, this will not be considered as part of this study.

While many analyses of survivorship are done using generic data \citep{Tomiya2013,Liow2008,Harnik2013,Finnegan2008,Foote2006}, there are potential biases in accurately modeling a specific level process using generic level data \citep{Raup1975,Sepkoski1975,Simpson2006,Raup1991a,VanValen1979}. In order to assess some of the differences between generic and specific level survival, I will estimate specific and generic level survival models. Using an approach similar to previous work on estimating specific level origination and extinction rates from generic level survival curves \citep{Foote1988}, I will measure the deviance between extinction rate directly estimated from the specific survivorship and the specific level extinction rates estimated from the generic level survival data. In addition to empirical comparison between generic and specific level survival, simulations of diversification with varying levels of cryptic speciation (anagenesis). This may also act as a proxy for generic level diversification because a lineage having a long duration because it is not correctly broken up can be considered analogous to a genus persisting because it continues to speciate.

In order to account for environmental shifts, two different time-dependent covariates will be used. \(\delta O^{18}\) isotope information for the whole Cenozoic \citep{Zachos2008} will be used as a global climate proxy. Additionally, the Paleogene--Neogene divide, which may reflect global environmental shift, will be modeled as a time-dependent step-function.
% 
%

\section{Preliminary results and proposed work}


%
%

\section{Intellectual merit}
%
%

\section{Broader impacts}
%
%

\bibliography{proposal}

\end{document}
