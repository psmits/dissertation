\documentclass[11pt,letterpaper]{article}

\usepackage{amsmath, amsthm}
\usepackage{graphicx,hyperref}
\usepackage{microtype, parskip}
\usepackage{bibentry}
\usepackage[numbers,sort&compress]{natbib}
\usepackage{docmute}

\frenchspacing

%%%%%%%%%%%%%%%%%%%%%%%%%%%%%%%%%%%%%%%%%%%%%%%%%%%%%%%%%%%%%%%%%%%%%%%%%
\pagestyle{plain}                                                      %%
%%%%%%%%%% EXACT 1in MARGINS %%%%%%%                                   %%
\setlength{\textwidth}{6.5in}     %%                                   %%
\setlength{\oddsidemargin}{0in}   %% (It is recommended that you       %%
\setlength{\evensidemargin}{0in}  %%  not change these parameters,     %%
\setlength{\textheight}{8.5in}    %%  at the risk of having your       %%
\setlength{\topmargin}{0in}       %%  proposal dismissed on the basis  %%
\setlength{\headheight}{0in}      %%  of incorrect formatting!!!)      %%
\setlength{\headsep}{0in}         %%                                   %%
\setlength{\footskip}{.5in}       %%                                   %%
%%%%%%%%%%%%%%%%%%%%%%%%%%%%%%%%%%%%                                   %%
\newcommand{\required}[1]{\section*{\hfil #1\hfil}}                    %%
\renewcommand{\refname}{\hfil References Cited\hfil}                   %%
\bibliographystyle{plainnat}                                           %%
%%%%%%%%%%%%%%%%%%%%%%%%%%%%%%%%%%%%%%%%%%%%%%%%%%%%%%%%%%%%%%%%%%%%%%%%%

\title{\uppercase{Dissertation Research:}\\ Cenozoic mammals and the biology of extinction}
\author{PI: Kenneth D. Angielczyk, Co-PI: Peter D. Smits}
\date{}

\begin{document}
\setcounter{secnumdepth}{0}
\maketitle
\section{Introduction}
Why certain species go extinct while others do not is one of the most fundamental questions in paleobiology. It is expected that for the majority of geological time, extinction is biologically non-random \citep{Jablonski1986,Alexander1977,Harnik2011,Johnson2002b,Kitchell1986,Nurnberg2013a,Payne2007}. Determining which biological factors or traits influence extinction risk and how is vital for understanding the differential diversification of life during the Phanerozoic. Periods of background extinction also represent the majority of geologic time, remain relatively predictable and change slowly, and thus providing a better opportunity to study how traits are related to survival than periods of mass extinction \citep{Jablonski1986,Raup1988}. The Law of Constant Extinction \citep{VanValen1973} posits that extinction risk of taxa within a given adaptive zone is age independent (memoryless), however the generality of this statement is possibly suspect \citep{Drake2014,Raup1975,Sepkoski1975,Finnegan2008}. By analyzing survival patterns within adaptive zones during periods of background extinction, it should be possible to determine if extinction is best modeled as age independent or dependent.

I simple expectation based on purely stochastic grounds, where extinction is not selective, is that abundant and widespread taxa are less likely to go extinct that rare and restricted taxa \citep{Raup1991b}. For example, species with larger geographic ranges tend to have lower extinction rates than species with smaller geographic ranges \citep{Jablonski1986,Harnik2013,Nurnberg2013a,Jablonski2003,Roy2009c}. However, this common pattern does not explain why certain taxa may be less prone to extinction than others. In the example, how range size is formed varies between clades and thus remains a black box for most taxa \citep{Jablonski1987} and so determining if differential extinction is a purely stochastic process or is actually the product of selection is impossible. Instead, by focusing on traits related to environmental preference, the traits which may underly why a taxon is abundant or widespread, the process of selection underlying differential extinction may be elucidated.

It is under this framework that I propose to study how ecological traits associated with environmental preference have affected mammalian differential extinction. Cenozoic mammals represent an ideal group and time period as it is the fossil record is well sampled and the ecology and phylogeny of these taxa are generally understood. Mammals are motile organisms which can track their preferred environmental context over time and space. If a mammal species requires rare or fragile environmental conditions, or is a poor disperser, this would limit the availability of suitable environments or ability to track the preferred environment. Three important traits that describe the relationship between mammals and their environmental context are body size, dietary category, and locomotor category \citep{Smith2004,Smith2008b,Damuth1981a,Damuth1979,Jernvall2004,Lyons2005,Lyons2010}. Each of these traits describe different aspects of a taxon's adaptive zone such as energetic cost, population density, expected home range size, set of potential prey items, and dispersal ability among others. 


%Environmental availability, along with stability, is crucial for both the establishment and persistence of a species.
%During the Cenozoic, primarily between the Paleogene--Neogene, there was a shift from a predominately closed environment to a predominately open environment \citep{Janis1993a,Blois2009,Rose2006}. 
%Additionally, with increased likelihood of prey item occurrence, abundance can increase \citep{VanValen1989,Brown1987,Damuth1979,Silva1997,Janis2000} which can effect both survival and increase occupancy \citep{Jernvall2004,Brown1984,Jernvall2002,Fortelius2002}.

Dietary categories are coarse groupings of similar dietary ecologies: carnivores, herbivores, omnivores, and insectivores. Because dietary category describes, roughly, the trophic position of a taxon and its related stability, it is predicted that more stable categories will have longer durations than less stable categories. Stability here being ``distance'' from primary productivity, thus it is expected that herbivores will have greater duration than carnivores. Omnivorous taxa are expected to have average taxon durations compared to the other two categories. If dietary category is not found to be important for modeling survival it may mean that trophic category is not a major factor for determining species level survival and that other factors, such as body size, may dominate.

Mammalian herbivores and carnivores have been found to have a greater diversification rate than omnivores \citep{Price2012} which may indicate that these traits are better for survival. However diversification can be caused either by an increase in origination relative to extinction or a decrease in extinction relative to origination. Which scenario occurred, however, is (currently) impossible to determine from a phylogeny of only extant organisms \citep{Rabosky2010a} which means that analysis of the fossil record is required. If survival is found to be similar between all dietary categories, this may mean that the differential diversification patterns observed by \citet{Price2012} are due to differences in speciation and not extinction.

Locomotor categories describe the motility of a taxon, the plausibility of occurrence, and dispersal ability. For example, an obligate arboreal taxon can only occur in locations with a minimum of tree cover and can most likely only disperse to other environments with suitable tree cover. Dispersal ability is important for determining the extent of a taxon's geographic range \citep{Birand2012,Jablonski2006a,Gaston2009} and affects both the taxon's extinction risk and regional community evenness. The categories of interest are arboreal, ground dwelling, and scansorial. With the transition from primarily closed to open environments, there is an expected shift in stability associated with arboreal and ground dwelling taxa. It is expected that arboreal taxa during the Paleogene will have a greater expected duration than Neogene taxa while the opposite will be true for ground dwelling taxa. In comparison, taxon duration of scansorial taxa is expected to remain relatively similar between the two time periods because it represents a mixed environmental preference that may be viable in either closed or open environments. If locomotor category is a poor descriptor of survival this may mean that it is either a poor descriptor of dispersal ability, which may or may not affect mammalian survival. 

An organisms body size, here defined as (estimated) mass, has an associated energetic cost in order to maintain homeostasis which in turn necessitates a supply of prey items. Many life history traits are associated with body size: reproductive rate, metabolic rate, home range size, among others \cite{Peters1983a,Damuth1979,Brown1987,Smith2004}. Body size can possibly affect species extinction because as body size increases, home range size increases \citep{Damuth1979}. If individual home range size scales up to reflect minimum total species geographic range, we would expect that taxa with larger body sizes would have lower extinction rates than species with smaller body sizes. This expectation, however, may not be right. An alternative but plausible scenario is where as body size increases, reproductive rate decreases \citep{Johnson2002b}, populations get smaller \citep{White2007}, and generations get longer \citep{Martin1993a} all of which can increase extinction risk, as has been observed \citep{Liow2008,Davidson2012}. However, the relationship between body size and extinction rate at the generic level has been found to vary between continents \citep{Tomiya2013,Liow2008}. By expanding to include a third continent, South America, and analyzing specific level data I hope to elucidate how differences in taxonomic diversity at a continental level might affect body size mediated extinction rate. If body size is found to be unimportant for modeling survival, as in the generic level analysis of \citet{Tomiya2013}, this means that other biotic or abiotic factors may dominate. This may also mean that individual level home range size does not scale to increased species level range size, and there is therefore no correlated decrease in extinction rate. If increase in body size increases extinction risk, this may be due to traits correlated with body size and not necessarily body size itself \citep{Johnson2002b}.

In addition to the individual effects, the interactions between body size, locomotor category, and dietary category may also be important for modeling taxonomic survival. For example, a small bodied arboreal taxon of any trophic category during the heavily forested and warm time of the Paleogene would be expected at once to have both a small body size determined range, a large potential geographic range determined by locomotion, as well as an increased availability of resources. Together this would mean that relative survival would be expected to be less than, greater than, and greater than average respectively. 

Mammalian species durations will be modeled in a survival analysis framework where the effect of the above traits can be quantified as the effect upon a taxon's duration. This form of analysis has a long history in paleobiology \citep{Simpson1944,Simpson1953,VanValen1979,Baumiller1993,Foote1988} but has fallen out of favor in recent years. Importantly, because some taxa may have originated before the Cenozoic or not have gone extinct yet, this information can be explicitly modeled as a censored duration which increases the total information included in the model. What makes survival analysis different from various linear modeling strategies such as logistic regression is that both duration and event occurrence are incorporated explicitly. In logistic regression, for example, only event occurrence is modeled, while in linear regression only duration is modeled. Importantly, this particular analytical framework has become quite mature since it's apparent abandonment by paleontology as it is commonly used in epidemiology, engineering, and demography \citep{Kleinbaum2005}. 

Many paleontological analyses of extinction selectivity and survivorship are performed at the generic level \citep{Tomiya2013,Liow2008,Harnik2013,Finnegan2008,Foote2006} there are potential biases in accurately modeling a specific level process using generic level data \citep{Raup1975,Sepkoski1975,Simpson2006,Raup1991a,VanValen1979}. Differences in species and generic level extinction risk can be attributed to speciation. Namely, if a trait has no effect on extinction risk at the specific level but decreases generic extinction risk this is most likely caused by a higher speciation rates in taxa exhibiting that trait. Because of this potentially elucidating property, I will be analyzing both species and generic level extinction risk. Importantly, this moves beyond simple analysis of diversification rates but partially decomposes the two aspects of diversification: speciation and extinction.

% phylogenetic comparative methods and difficulties decomposing diversification

\section{Preliminary results and proposed work}
Current analyses have been restricted to North America and Europe principally because of data availability. Fossil occurrence information was downloaded from both the Paleobiology Database (PBDB; \url{http://paleodb.org/} and the Neogene Old World Database (NOW; \url{http://www.helsink.fi/science/now/}). Dietary and locomotor assignments for each species were taken from the PBDB and the NOW. Body size estimates are based on a combination of measurement data from the PBDB, NOW, and a species by species literature search. These estimates are based on the common practice of using regression equations for estimating mass from measures such as tooth area or skull length \citep{Alroy1998,Tomiya2013,Jernvall2004,Alroy2009,Slater2013a}.

Duration was measured as the difference between first appearance (FAD) and last appearance (LAD), in millions of years. This value is most certainly a truncated version of the true species duration and so some amount of correction must be made for sighting failure \citep{Alroy2014a,Solow1997,Strauss1989}.

Preliminary analyses were performed solely within a maximum likelihood framework combined with multi-model inference. Duration information was fit to either an exponential or Weibull distribution, common distributions for wait-time data used in survival analysis. The exponential distribution is a special case of the Weibull distribution where the scale (acceleration-deceleration) parameter \(k\) is set to 1, meaning that an exponential distribution corresponds to age-independent extinction and thus the Law of Constant Extinction \citep{VanValen1979}. Dietary and locomotor categories were broken up into \(r - 1\) binary predictor variables, where \(r\) is the number of unique states, for a total of five binary predictors. A constant term was also estimated which corresponds to the remaining dietary and locomotor states. Natural log body mass was also used as a predictor. 

% math?

% various graphs

The fossil records of North American and Europe are frequently the subject of large spatial and temporal scale analyses of diversification \citep{Jernvall2004,Jernvall2002,Fortelius2002,Janis2000,Alroy1996a,Alroy1998,Alroy2000g,Liow2008,Raia2006,Tomiya2013}. These two continents are frequently used as proxies for global mammalian diversification even though there are known differences between them \citep{Liow2008,Tomiya2013}. The continent of South America, though well sampled for certain time periods, is rarely included and integrated into large spatial and temporal scale analyses \citep{Stromberg2013,Marshall1982}. My goal is to include the South American along with the North American and European mammal fossil records in order to better understand the different selective pressures on mammal distributions and survival.

I have secured access to a large data set of fossil mammal occurrences and biochronology collected by Dr. Richard D. Madden. This large incredibly complete and accurate collection provides the primary data type used in survival analysis however the traits of interest are mostly absent. And while dietary and locomotor categories can be coarsely assigned to virtually all observed taxa, body size cannot be estimated without reference to an actual physical specimen. 

A wide variety of South American fossil mammals are housed in US institutions like the Field Museum of Natural History and the American Museum of Natural History. I have begun collecting body mass estimates for Notoungulata, a group of South American ungulate-like mammals, however even within this one order I do not have complete coverage. Many species are absent in US collections and are only housed in the national museums of various South American countries, principally Argentina. 

% ungulate hierarchical model
% need a lot more citations, like Rick's book etc.
% browse vs graze + mass as predictors
%   all share locomotion / herbivore
A unique aspect of the South American record is the presence of multiple orders which are either rare or completely absent from other continents \citep{Marshall1982,Macfadden1997,Macfadden2006,Flynn1998a}. This ``splendid isolation'' led to the convergent evolution various ungulate-like taxa such as the orders Notoungulata and Litopterna. With the inclusion of full data on South American ungulate-like species, it would be possible to estimate if all ``ungulate'' taxon durations are drawn from the same underlying distribution and with what degree. By modeling the major orders of ungulate mammals as exchangeable in a Bayesian hierarchical survival model, it is possible to estimate both the order-specific survival distributions and estimate the common underlying distribution along with its accompanying uncertainty \citep{Gelman2013d}.


Future modeling work will be done in a purely Bayesian framework. This shift in analytical style is to 1) better estimate the uncertainty with which estimates are made, 2) combine the analysis of the different continents in hierarchical framework in order to estimate a common underlying distribution of survival times, 3) incorporate phylogenetic distance into a prior on a frailty coefficient \citep{Banerjee2003a,Ibrahim2001}, and to 4) move on a continuous model expansion framework as opposed to a discrete model choice one \citep{Gelman2013d}. All of these changes will dramatically improve the interpretability and meaning of the results.


\section{Intellectual merit}
%
%

\section{Broader impacts}
%
%

\bibliography{proposal}

\end{document}
