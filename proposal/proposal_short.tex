\documentclass[12pt,letterpaper]{article}

\usepackage{amsmath, amsthm}
\usepackage{graphicx}
\usepackage{microtype, parskip}
\usepackage{caption, subcaption, multirow, morefloats}
\usepackage{rotating, longtable}
\usepackage{hyperref}
\usepackage[numbers,sort&compress]{natbib}
\usepackage[nottoc,numbib]{tocbibind}
\usepackage{authblk, attrib, fullpage}
\usepackage{lineno}

\frenchspacing

\setcounter{secnumdepth}{0}

\begin{document}
\begin{titlepage}
  \begin{center}
    \huge{Evolutionary paleoecology and the biology of extinction}

    \vspace{1.5cm}

    \large{Peter D. Smits \\}
    \footnotesize{\href{mailto:psmits@uchicago.edu}{psmits@uchicago.edu}}

    \vspace{1.5cm}

    Dissertation Proposal Hearing \\
    \today \\
    Committee on Evolutionary Biology \\
    The University of Chicago

    \vspace{1.5cm}

    \textit{Committee} \\
    Dr. Michael J. Foote (co-advisor) \\
    Dr. Kenneth D. Angielczyk (co-advisor) \\
    Dr. Richard H. Ree \\
    Dr. P. David Polly
  \end{center}
\end{titlepage}


\section{Introduction}
Why certain taxa go extinct while others do not is a fundamental question in paleobiology. It is expected that for the majority of geologic time, extinction is non-random with respect to biology \citep{Jablonski1986,Alexander1977,Harnik2011,Johnson2002b,Kitchell1986,Nurnberg2013a,Payne2007}. Because of this, determining how different organismal traits may affect extinction risk is extremely important to better understanding what may have produced the observed patterns of taxonomic diversity. 

Traits relating to environmental preference are good candidates for modeling differences in extinction risk. A variety of organismal traits have been shown to be related to differential extinction \citep{Foote2013,Liow2007b,Baumiller1993,Nurnberg2013a,Alexander1977,Kitchell1986}, especially with regards to the relationship between adaptation to variable environments and increased species longevity. A simple expectation based purely on stochastic grounds is that taxa with a preference for rare environments will be more at risk than taxa which prefer abundant environments. As environments change in availability, a taxon's instantaneous risk of extinction would then be expected change in concert. Taxa are also expected to be adapting to their environment, possibly increasing or decreasing their environmental tolerance and thus changing their instantaneous extinction risk. 

Related to environmental preference is taxon geographic range size. Species with larger geographic ranges tend to have lower extinction rates than species with smaller geographic ranges \citep{Jablonski1986,Harnik2013,Nurnberg2013a,Jablonski2003,Roy2009c}. However, how range size is ``formed'' is different between clades \citep{Jablonski1987} and thus remains a black box for most taxa. Thus, the utility of focusing on organismal traits related to environmental preference is that the black box can be ``opened.''

An important principle of extinction is the Law of Constant extinction \citep{VanValen1973} posits that extinction risk within a given adaptive zone is taxon age independent can be modeled as a memoryless (exponential) process. And while the generality of this statement is possible suspect \citep{Drake2014,Raup1975,Sepkoski1975,Finnegan2008,Raup1991a}, the Law of Constant extinction is theorized to hold during periods of environmental stability and is thus considered extremely difficult, if not impossible, to test \citep{Liow2011a}. However, it is conceivable that if the context of a taxon's adaptive zone is modeled along with the trait-based extinction risk of a taxon it may be possible to truly test the Law of Constant extinction.

In order to better understand the environmental context of a taxon, what level processes (global, regional, local) might shape the observed diversity and if there are systematic differences between ecotypes is extremely important to identify. The relationship between endemic and cosmopolitan taxa, or \(\alpha\) and \(\beta\) diversity respectively, is a measure of regional community composition and can be used to estimate what level processes may have dominated. If endemism is high then local processes may dominate while if cosmopolitan taxa are abundant then regional processes may dominate. Additionally, if multiple regions are correlated with each other or regions are correlated with global abiotic factors then global processes may dominate. These scenarios are not mutually exclusive. 

It is under this framework that I propose to study how ecological traits associated with environmental preference have affected both differential survival and cosmopolitan-endemism dynamics. I will be studying two distantly related and biotically different groups: Permian brachiopods and Cenozoic mammals. Both of these groups are considered to have very good fossil records able to reflect long term evolutionary patterns \citep{Mark1977}. These two time periods were chosen because they represent periods of approximately the same length (47 My and 65 My) and of climatic change, global warming and global cooling respectively. Also, these two groups are a marine and terrestrial system respectively and the traits associated with environmental preference and range size (described below) are fundamentally very different. Both patterns of survival and community connectedness will be measured for both of these groups. The differences between these two groups in terms of life-habit and environmental preference, along with global climatic context, provides a fantastic scenario to understand how long-term, large-scale processes away from mass extinctions proceed.


\section{Survival and environmental preference}
\subsection{Questions}
\begin{itemize}
  \item Do traits related to environmental preference have different distributions of taxonomic duration?
    \begin{itemize}
      \item Is survival best modeled by a single trait or multiple?
      \item How do other factors, such as climate, affect these patterns?
    \end{itemize}
  \item Is extinction taxon-age independent or dependent?
\end{itemize}

\subsection{Hypotheses and predictions}
\subsubsection{Mammals}
% diet
% locomotor category
% body size
\subsubsection{Brachiopods}
% substrate
% habitat
% affixing strategy

\subsection{Methods}
Using a survival analytical framework, I will model taxonomic duration using the above traits as predictors. Importantly, the fact that taxa both originate before the time period of interest and go extinct after that time period is explicitly incorporated. Duration time distributions can be modeled using multiple theoretical distributions, such as the exponential and Weibull distributions, which allows for the possible age-independence or age-dependence can be estimated and compared.

% survival analytical framework allows for inclusion of both duration and state
% parametric approaches assign a (plausible) distribution to times

% estimating parameters allows inference into both tempo and mode


\section{Average \(\alpha, \beta\) diversity over time}
\subsection{Questions}
\begin{itemize}
  \item How does the ratio of cosmopolitan to endemic taxa, per locality, change over time?
    \begin{itemize}
      \item Is this pattern different between taxa exhibiting different traits?
      \item How does this pattern very in relation to phylogenetic similarity?
    \end{itemize}
  \item When would we expect global, regional, and/or local processes to most strongly shape taxonomic patterns?
\end{itemize}

\subsection{Hypotheses and predictions}
\subsubsection{Mammals}
% diet
% locomotor category
\subsubsection{Brachiopods}
% substrate
% habitat
% affixing strategy

\subsection{Methods}
Community composition will be measured using a bigeographic network structure where localities and occurrences are connected as a bipartite network \citep{Sidor2013,Vilhena2013,Vilhena2013b}. Here, localities will be defined as grid cells from an equal area map projection and taxa will be defined as either generic or specific occurrences depending on the study (generic for brachiopods, and both generic and specific for mammals). 

Modified from \citet{Sidor2013}, community composition will be measured via for metrics: relative number of endemic (\(E\)), relative locality occupancy per taxon (\(Occ\)), biogeographic connectedness (\(BC\)), and code length \citep{Rosvall2008,Rosvall2009a}. \(E\) is a measure of \(\alpha\) diversity, \(Occ\) a measure of \(\beta\) diversity, \(BC\) a measure of regional evenness, and code length is an estimate of provinciality.

If global processes are important to patterns of community connectedness and environmental interactions than it is expected that these will be correlated with global climate measures. Additionally, if two or more regions have similar or correlated patterns of community connectedness, it is expected that global processes may play a roll in shaping these environments. Regional processes are expected to dominate when \(E\) is low, \(Occ\) is high, \(BC\) is high, and code length is high. In contrast, local processes are expected to dominate when \(E\) is high, \(Occ\) is low, \(BC\) is low and code length is low. The different scales are not mutually exclusive, however, and one or more scales might be involved in shaping patterns of community connectedness and environmental interactions. Importantly, which process scales are dominant may change over time.


\clearpage
\bibliographystyle{abbrvnat}
\bibliography{proposal}
\end{document}
