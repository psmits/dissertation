\documentclass[12pt,letterpaper]{article}

\usepackage{amsmath, amsthm}
\usepackage{graphicx}
\usepackage{microtype, parskip}
\usepackage{caption, subcaption, multirow, morefloats}
\usepackage{rotating, longtable}
\usepackage{hyperref}
\usepackage[numbers,sort&compress]{natbib}
\usepackage[nottoc,numbib]{tocbibind}
\usepackage{authblk, attrib, fullpage}
\usepackage{lineno}

\frenchspacing

\setcounter{secnumdepth}{0}

\begin{document}
\begin{titlepage}
  \begin{center}
    \huge{Evolutionary paleoecology and the biology of extinction}

    \vspace{1.5cm}

    \large{Peter D. Smits \\}
    \footnotesize{\href{mailto:psmits@uchicago.edu}{psmits@uchicago.edu}}

    \vspace{1.5cm}

    Dissertation Proposal Hearing \\
    May 12, 2014\\
    Committee on Evolutionary Biology \\
    The University of Chicago

    \vspace{1.5cm}

    \textit{Committee} \\
    Dr. Michael J. Foote (co-advisor) \\
    Dr. Kenneth D. Angielczyk (co-advisor) \\
    Dr. Richard H. Ree \\
    Dr. P. David Polly
  \end{center}
\end{titlepage}

\linenumbers
\modulolinenumbers[2]

\section{Introduction}
Why certain taxa go extinct while others do not is a fundamental question in paleobiology. It is expected that for the majority of geologic time, extinction has been non-random with respect to biology \citep{Jablonski1986,Alexander1977,Harnik2011,Johnson2002b,Kitchell1986,Nurnberg2013a,Payne2007}. Determining how different organismal traits may affect extinction risk is then extremely important for better understanding what processes underlie the observed patterns of taxonomic diversity. 

A variety of organismal traits have been shown to be related to differential extinction \citep{Foote2013,Liow2007b,Baumiller1993,Nurnberg2013a,Alexander1977,Kitchell1986}, especially with regards to the relationship between adaptation to variable environments and increased species longevity. A simple expectation based purely on stochastic grounds is that taxa with a preference for rare environments will be more at risk than taxa which prefer abundant environments. As environments change in availability, a taxon's instantaneous risk of extinction would then be expected to change in concert. Taxa are also expected to be adapting to their environment, possibly increasing or decreasing their environmental tolerance and thus changing their instantaneous extinction risk. 

Related to environmental preference is geographic range size. Taxa with larger geographic ranges tend to have lower extinction rates than species with smaller geographic ranges \citep{Jablonski1986,Harnik2013,Nurnberg2013a,Jablonski2003,Roy2009c}. However, how range size is ``formed'' is different between clades \citep{Jablonski1987} and thus remains a black box for most taxa. Thus, traits relating to environmental preference are good candidates for modeling differences in extinction risk. The utility of focusing on organismal traits related to environmental preference is that the black box of range size can be ``opened.''

An important principle of extinction is the Law of Constant Extinction \citep{VanValen1973} which posits that, within a given adaptive zone, extinction risk is taxon age independent and can be modeled as a memoryless (exponential) process. And while the generality of this statement is possibly suspect \citep{Drake2014,Raup1975,Sepkoski1975,Finnegan2008,Raup1991a}, the Law of Constant Extinction is theorized to hold during periods of environmental stability and is thus considered extremely difficult, if not impossible, to test \citep{Liow2011a}. However, it is conceivable that if the context of a taxon's adaptive zone is modeled along with trait-based extinction risk it may be possible to truly test the Law of Constant Extinction.

In order to better understand the environmental context of a taxon, identifying what level processes (global, regional, local) might have shaped the observed diversity and if there are systematic differences between ecotypes is extremely important. The relationship between endemic and cosmopolitan taxa, or \(\alpha\) and \(\beta\) diversity respectively, is a measure of regional community composition and can be used to estimate what processes may have dominated. If endemism is high then local processes may dominate while if cosmopolitan taxa are abundant then regional processes may dominate. Additionally, if multiple regions are correlated with each other or regions are correlated with global abiotic factors then global processes may dominate. These scenarios are not mutually exclusive. 

It is under this framework that I propose to study how ecological traits associated with environmental preference have affected both differential survival and cosmopolitan-endemism dynamics. I will be studying two biotically different groups: Permian brachiopods and Cenozoic mammals. Both of these groups are considered to have very good fossil records able to reflect long-term evolutionary patterns \citep{Mark1977}. These two time periods were chosen because they represent periods of approximately the same length (47 My and 65 My) and of climatic change, global warming and global cooling respectively. Also, these two groups are a marine and a terrestrial system respectively and the traits associated with environmental preference (described below) are fundamentally different. Patterns of survival and community connectedness will be measured for both of these groups. The biological and temporal differences between these two groups provides a fantastic scenario to understand how long-term, large-scale processes away from mass extinctions proceed.


\section{Survival and environmental preference}
\subsection{Questions}
Do traits related to environmental preference have different distributions of taxonomic duration? Is survival best modeled by a single trait or multiple? How do other factors, such as climate, affect these patterns? Is extinction taxon-age independent or dependent?

\subsection{Hypotheses and predictions}
\subsubsection{Mammals}
Three mammalian traits that are plausible determinates of differential survival are dietary category, locomotor category, and body size \citep{Price2012,Smith2004,Jernvall2004,Janis1993a,Blois2009,Liow2008,Alroy2000g,Johnson2002b}. These traits describe both the energetic requirements of a taxon as well as the plausibility of occurrence at a given location.

Dietary categories are coarse groupings of similar feeding ecologies: carnivores, herbivores, omnivores, and insectivores. Each of these categories is composed of taxa with a variety of ecologies. Dietary category describes, roughly, the trophic position of a taxon and its related stability. Stability is a descriptor of the ``distance'' from primary productivity and the availability of prey items. When prey item abundance increases, predator abundance can increase \citep{VanValen1989,Brown1987,Damuth1979,Silva1997,Janis2000} which can decrease extinction risk \citep{Jernvall2004,Brown1984,Jernvall2002,Fortelius2002}. It is predicted, then, that herbivores will have greater durations on average than carnivores while omnivorous taxa are expected to have average taxon durations compared to the other two categories. However, it is possible that trophic category may not be an important predictor of extinction risk and instead only affect origination rate \citep{Price2012} and other factors, such as locomotor category, may be more important for estimating extinction risk.

Locomotor categories describe the motility of a taxon, the plausibility of occurrence, and dispersal ability. Here, the categories are arboreal, ground dwelling, and scansorial. Dispersal ability is important for determining the extent of a taxon's geographic range \citep{Birand2012,Jablonski2006a,Gaston2009} which can affect a taxon's extinction risk. Because there was a Cenozoic transition from primarily closed to primarily open habitat \cite{Stromberg2005,Stromberg2013,Janis1993a,Blois2009,Rose2006}, it is expected that arboreal taxa during the Paleogene will have a greater expected duration than Neogene taxa while the opposite will be true for ground dwelling taxa. It is possible that locomotor category is a poor predictor of survival, possibly because it is a poor descriptor of dispersal or dispersal is not important to mammalian survival. It may then be the case that other traits, such as body size, may be better predictors of survival. 

An organisms body size, here defined as (estimated) mass, has an associated energetic cost in order to maintain homeostasis which in turn necessitates a supply of prey items. Many life history traits are associated with body size: reproductive rate, metabolic rate, home range size, among others \cite{Peters1983a,Damuth1979,Brown1987,Smith2004}. As body size increases, for example, home range size also increases \citep{Damuth1979}. If individual home range size scales up to reflect total species geographic range, it is expected that taxa with larger body sizes will have lower extinction risk than species with smaller body sizes. However, as body size increases, reproductive rate decreases \citep{Johnson2002b}, populations get smaller \citep{White2007}, and generations get longer \citep{Martin1993a} all of which can increase extinction risk \citep{Liow2008,Davidson2012}. If increase in body size increases extinction risk, this may in fact be due to traits which are correlated with body size and not necessarily body size itself \citep{Johnson2002b}. If body size is not found to be a good predictor of survival, as see in North American Neogene mammal genera \citep{Tomiya2013}, it may be that, for example, individual level home range size does not scale to increased species level range size and a decrease in extinction risk. 

\subsubsection{Brachiopods}
Three brachiopod traits which may be important for estimating survival are substrate preference, habitat preference, and affixing strategy \citep{Alexander1977,Richardson1997,Richardson1997a,Johansen1989}. These traits describes some aspect of how a taxon interacts with its environment and can potentially limit the potential range of plausible environments.

Substrate preference is related to the chemical and physical processes present at a given location. The three generally used states of substrate affinity are carbonate, clastic, or mixed \citep{Foote2006,Anderson2011a,Nurnberg2013a,Kiessling2007a,Miller2001}. Because of both the long-term decline in carbonates versus clastics \citep{Peters2008} and the dominance of Permian-age clastic beds \citep{Birgenheier2010,Percival2012,Thomas2007,Fielding2008a,Fielding2008}, taxa with clastic type affinities are expected to have longer durations than taxa with any other preference. Additionally, it is predicted that substrate preference, if it captures the same information as modern substrate type, will be a predictor in the best model(s) of survival \citep{Richardson1997,Richardson1997a}. However, if substrate affinity is not found to be important for modeling survival this may be due to one or more reasons. First, substrate affinity, as quantified here, may not be capturing the same information as modern substrate type and thus may act as a poor predictor of survival. Second, it may mean that because clastic type substrates were so dominate during the Permian of Australia that survival may be better explained by other factors, either measured or unmeasured. 

Habitat preference is a description of the environment in which a taxon was found. Because of the diverse range of marine settings and difficulty of precisely inferring paleoenvironment a frequently used, albeit coarse, classification is on-shore versus off-shore \citep{Sepkoski1991,Kiessling2007a,Bottjer1988,Jablonski1991,Jablonski1983b} along with the option of a taxon having no particular habitat preference. Habitat availability is broadly related to sea-level which can change both dramatically and rapidly over time \citep{Olszewski2004}. During the Permian there were four major glaciation events which covered most of the Australian continent \citep{Fielding2008,Birgenheier2010,Fielding2008a,Fielding2006}, which most likely strongly impacted sea-level as well as the availability and constancy of on-shore habitats. It is expected that off-shore adapted taxa will have greater durations than on-shore adapted taxa. If habitat preference is not found to be an important predictor of survival, this may mean that sea-level mediated environmental availability may not determine long term survival. Specifically, while sea-levels may have fluctuated greatly due to high latitude glaciation \citep{Fielding2008,Fielding2008a,Birgenheier2010} it may be that the long term continual availability of habitat over-shadows short term fluctuations. Also, it has been found in the case of Permian brachiopods from Texas that sea-level along with climate change do not wholly explain the observed ecological dynamics \citep{Olszewski2004}, which may mean that habitat availability may not be the singly dominate factor when modeling brachiopod survival. 

Affixing strategy is the manner by with an individual interfaces with the ocean floor. Brachiopods have evolved a variety of different methods to position themselves in various different environmental conditions such as flow speed or mud depth \citep{Rudwick1970,Alexander1977,LaBarbera1978,LaBarbera1981,Richardson1997}. Broadly, these strategies can be classified as pedunculate (presence of a pedicle), reclining (absence of pedicle), and cementing. During the Permian, pedunculate taxa are associated with shallow on-shore environments while reclining taxa are associated with deep off-shore environments \citep{Clapham2007} however these associations are weak as most assemblages are composed of a heterogeneous mix of strategies. Previous analyses have shown that affixing strategy is correlated with both duration and survival \citep{Alexander1977,Johansen1989}. If affixing strategy is found to not be an important predictor in the best model(s) of survival this would mean that, while it is correlated with differential survival \citep{Alexander1977,Johansen1989}, it may only be a minor factor. Additionally, this may indicate that the environmental energetics of Australia were rather uniform or constant with respect to time.

\subsection{Methods}
Using a survival analytical framework, I will model taxonomic duration using the above traits as predictors. Survival analysis is a framework for modeling time till event data, such as time from origination (FAD) till extinction (LAD) \citep{Kleinbaum2005,Simpson1944,VanValen1973}. An important aspect of survival analysis is that the fact that some samples may not have gone extinction can be incorporated explicitly in the model as ``censored'' data \citep{Kleinbaum2005}. In a survival analytical framework, time is a measure of taxon duration and not geological time. By modeling the distribution of survival times as predicted by one or more traits it is possible to estimate the effects and relative predictive powers of each trait.

The Law of Constant Extinction, when translated into a survival analytical framework, states that the hazard function which describes the instantaneous potential of extinction is constant with respect to time (\(h(t) = \lambda\)). This specific case only occurs when survival times are exponentially distributed. By comparing the fit various theoretical distributions of survival it is possible to determine if extinction is random with respect to taxon duration or not. 

By combining inference of the relative importance of various traits and the best fitting theoretical distribution of taxon durations, it is possible to simultaneously describe both the tempo and mode as well as the controls of extinction.


\section{Average \(\alpha, \beta\) diversity over time}
\subsection{Questions}
How does the ratio of cosmopolitan to endemic taxa, per locality, change over time? Is this pattern different between taxa exhibiting different traits? How does this pattern very in relation to phylogenetic similarity? When would we expect global, regional, and/or local processes to most strongly shape taxonomic patterns?

\subsection{Hypotheses and predictions}
\subsubsection{Mammals}
During the Cenozoic there was a global shift from a ``hot house'' environment to an ``ice house'' environment \citep{Zachos2008,Zachos2001}. This transition was accompanied by major shifts in global climatic envelopes and the reorganization of mammalian communities \citep{Janis1993a,Fortelius2002,Blois2009,Alroy2000g,Figueirido2012}. 

It is expected that the patterns of biogeographic community connectedness for herbivorous taxa in a region would be most similar to that for all regional taxa combined and potentially ``drive'' the regional pattern, partially because on average this category represents the majority or plurality of taxa \citep{Jernvall2002}. In contrast, community connectedness for carnivorous taxa is expected to remain constant over time or be correlated with herbivore patterns. Finally, omnivorous taxa are not expected to be correlated with the patterns of either herbivorous or carnivorous taxa and have either a relatively constant or random pattern of connectedness over time.  These predictions are based on the differences in resilience and relationship to primary productivity, with herbivores being more resilient than carnivores and omnivores being random in their resilience \citep{Jernvall2004}. Resilience is defined here as the ability for a taxon to increase in occupancy following a decline \citep{Jernvall2004}.

The Cenozoic global shift from closed, forested habitat in the Paleogene to open, savanna-like habitat during the Neogene would have greatly affected the possible distributions of both arboreal and ground dwelling taxa. Generally this transition would have caused forested environments to become increasingly patchy in their distribution while transitioning from the Paleogene to the Neogene. The global prediction then is that there would have been a relative increase in \(E\) and code length accompanied by a decrease in \(BC\) and \(Occ\) in arboreal taxa over time (terms defined below). The opposite is expected for terrestrial taxa. 

At a regional scale, North American community connectedness is expected to follow the global predictions described above because the vast amount of prior synthesis has focused on North America \citep{Alroy2000g,Alroy1996a,Alroy1998,Barnosky2001a,Simpson1944,Simpson1953,Badgley2013,Blois2009,Figueirido2012,Gunnell1995,Hadly2001}. However, the effect of global climate change on North American diversity remains unresolved and controversial \citep{Alroy2000g,Blois2009,Figueirido2012,Barnosky2001a}, thus it is necessary to determine empirically when global versus regional versus local scale processes may have dominated and how that may have changed over time.

The European mammalian fossil record is also well studied, though research has primarily focused on the Neogene \citep{Jernvall2002,Jernvall2004,Liow2008,Raia2006,Raia2005,Raia2011c}. An important aspect about the European record is that during the Neogene there was little shift in relative dietary category abundance \citep{Jernvall2004} and that the patterns within herbivores (browse--graze transition) were mostly driven by abundant, cosmopolitan taxa \citep{Jernvall2002}. It is predicted then that herbivores will demonstrate the same patterns of community connectedness as Europe as a whole, while omnivores and carnivores will be different from that of herbivores and may demonstrate random or constant patterns of community connectedness through time. 

Patterns of community connectedness for South American mammalian fauna are comparatively less synthesized than those of North American and Europe. Instead, cross--continental dynamics between North and South America during the Neogene are much more studied \citep{Marshall1982}. The South American mammalian faunal record reflects two distinct biotic provinces between the North and the South \citep{Macfadden1997,Macfadden2006,Flynn1998a,Patterson1968}. Because of this, it is expected that South America will have a different pattern of community connectedness than either North America or Europe. Also, there is an expected dramatic increase occupancy in land-dwelling herbivores relative to arboreal and scansorial taxa related to the aridification of high--latitude South America. Additionally, because of this strong biome distinction, it is predicted that provinciality will be high but remain constant over time. % improve this a lot following Dave Polly's comments

\subsubsection{Brachiopods}
During the Permian, the east coast of the Australian continent faced towards the massive Panthalassic Ocean. Because of this, the establishment of populations was most likely limited to within the local area because the distance required to establish elsewhere was most likely too great. It is then expected that community connectedness in Australian Permian brachiopods would be fairly similar at any given time and that changes, specifically decreases in connectedness, would be expected during the four glacial periods \citep{Fielding2008a,Fielding2008}. Dispersal ability of modern brachiopods is limited by the availability and proximity of suitable substrate \citep{Richardson1997,Richardson1997a}. The Permian of Australia is dominated by widespread clastic beds compared to relatively few carbonate beds. The expectation is that the distribution of taxa with a carbonate preference will be patchy and have a high \(E\), low \(Occ\), low \(BC\), and low code length compared to the distribution of clastic preferring taxa (terms defined below). However, if community connectedness is approximately equal between carbonate and clastic preferring taxa this could be caused by approximately equal dispersal ability in both groups, either high or low.

Habitat would be expected to influence community structure if there is an uneven distribution of available habitats in space and time. Rarity of preferred habitat would be expected to lead to high \(E\), low \(Occ\), low \(BC\), and low code length compared to an abundance of preferred habitat (terms defined below). Because of the four major glaciation events during the Permian of Australia, it is expected that the availability of on-shore habitats would be highly variable. It is then expected that during periods of glacial activity community connectedness of on-shore preferring taxa would be extremely low because of rarity of environments in comparison to both periods of non-glacial activity and off-shore habitats at all times. If habitat preference has no effect on community connectedness this may mean that the dispersal ability of on-shore taxa is very high and able to maintain gene flow between potentially isolated habitats.

It is expected that affixing strategy alone will have minimal effect on community connectedness unless affixing strategy is highly correlated with substrate and/or habitat preference. If community connectedness is found to be different between affixing strategies but affixing strategy is not highly correlated with substrate or habitat preference this may be because of spatial heterogeneity in energy levels which may limit reclining versus fixed taxon distributions. This scenario is highly unlikely given knowledge of modern and fossil brachiopod distributions \citep{Rudwick1970,Richardson1997,Richardson1997a}.

\subsection{Methods}
Community composition will be measured using a bigeographic network structure where localities and occurrences are connected as a bipartite network \citep{Sidor2013,Vilhena2013,Vilhena2013b}. Here, localities will be defined as grid cells from an equal area map projection and taxa will be defined as either generic or specific occurrences depending on the study (generic for brachiopods, and both generic and specific for mammals). 

Modified from \citet{Sidor2013}, community composition will be measured via for metrics: relative number of endemic (\(E\)), relative locality occupancy per taxon (\(Occ\)), biogeographic connectedness (\(BC\)), and code length \citep{Rosvall2008,Rosvall2009a}. \(E\) is a measure of \(\alpha\) diversity, \(Occ\) a measure of \(\beta\) diversity, \(BC\) a measure of regional evenness, and code length is an estimate of provinciality.

If global processes are important to patterns of community connectedness and environmental interactions than it is expected that these will be correlated with global climate measures. Additionally, if two or more regions have similar or correlated patterns of community connectedness, it is expected that global processes may play a roll in shaping these environments. Regional processes are expected to dominate when \(E\) is low, \(Occ\) is high, \(BC\) is high, and code length is high. In contrast, local processes are expected to dominate when \(E\) is high, \(Occ\) is low, \(BC\) is low and code length is low. The different scales are not mutually exclusive, however, and one or more scales might be involved in shaping patterns of community connectedness and environmental interactions. Importantly, which process scales are dominant may change over time.

In addition to measures of \(\alpha\) and \(\beta\) diversity, the degree to which taxa at a locality are phylogenetically similar may play an important role in structuring a community assemblage \citep{Webb2002}. For example, closely related taxa may be ``repulsed'' due to competitive exclusion or ``clumped'' because of environmental filtering. By estimating the phylogenetic similarity using, for example, mean pairwise patristic distance \citep{Webb2002} or phylogenetic species variability \citep{Helmus2007a} it should be possible to incorportate these effects of common ancestry into better understanding assemblage structure and inrepretting the importance of various processes at any given time.

The above statistics will be estimated for all time bins comprising the duration of interst. The patterns exhibted by these statistics will be compared within and, in the case of Cenozoic mammals, between regions to estimate what level processes shape the average environmental context of a taxon.


\section{Synthesis}
Underlying all of the above is a foundational question in paleobiology: why do certain taxa go extinct while others do not? Here I have proposed framework using two distinct approaches for understanding both extinction risk and environmental context in two biologically and temporally different groups, Cenozoic mammals and Permian brachiopods. The survival studies proposed above investigate how organismal traits potentially related to environmental preference affect extinction rate. In effect, these traits may determine the ``bounds'' of a taxon's adaptive zone by limiting the total set of interactions to just those for which the taxon is adapted. The community connectedness studies aim to estimate what processes (global, regional, and/or local) may be dominate in shaping the environment and the related set of adaptive zones. Between these studies, as well the use of two disparate groups, it should be possible to determine when, what, and if certain variables matter for survival and, potentially, how they matter. 

\clearpage

\section{Timeline}

Spring/Summer 2014
\begin{itemize}
  \item Evolution Meeting: mammalian survivorship analysis for North America and Europe
  \item South American fossil mammal data from American Museum of Natural History collections
\end{itemize}

Fall 2014/Winter 2015
\begin{itemize}
  \item GSA: survivorship simulation for anagenesis and sampling
  \item Doctoral Dissertation Improvement Grant
\end{itemize}

Spring/Summer 2015
\begin{itemize}
  \item Evolution Meeting: preliminary brachiopod survival results
  \item write and submit survivorship simulation paper
  \item possible South American fossil mammal data from American Museum of Natural History collections
\end{itemize}

Fall 2015/Winter 2016
\begin{itemize}
  \item SVP: mammalian biogeographic connectedness
  \item write and submit mammal connectedness paper
\end{itemize}

Spring/Summer 2016
\begin{itemize}
  \item Evolution Meeting: brachiopod survival analysis
  \item write and submit brachiopod community paper
\end{itemize}

Fall 2016/Winter 2017
\begin{itemize}
  \item GSA: brachiopod community connectedness
  \item write and submit brachiopod survival paper
\end{itemize}

Spring/Summer 2017
\begin{itemize}
  \item Evolution Meeting: survival and communities together
  \item write and submit mammal survival paper
  \item write and review/philosophy paper
  \item \textbf{Defend}
\end{itemize}

\clearpage
\bibliographystyle{abbrvnat}
\bibliography{proposal}

\end{document}
