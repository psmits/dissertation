\documentclass{article}

\begin{document}

1. Common questions that can be answered by comparison between mammals and brachiopods

Including both mammals and brachiopods in your dissertation is an interesting tactic.  At very least it will give you comparative insights into two different systems and put you on track for publications that will resonate with different sub-communities of paleontologists.  You might consider whether you can do more with the comparison by making it an explicit part of your methodology.  At the moment the two parts operate in parallel, with similar (but specifically different) questions being asked with each data set.  Can you frame broader questions (or hypotheses) that can be addressed by direct comparison of the two data sets?  You hint at one in the last sentence of the proposal (‘emergent properties’), but it isn’t developed anywhere else.  You also hint at one in the last sentences of the mammal section (hazard functions and whether they are good models for survivorship). 

In addition to these two broad questions of yours, other ways that you might make use of the comparison is by capitalizing on the differences and similarities between the two data sets (e.g., Permian is ice-house to hot-house, Cenozoic is hot-house to ice-house, both are about the same time duration (47 Ma for Permian, 65 Ma for Cenozoic), single region vs multi continent, marine vs. terrestrial).  If you can frame specific predictive hypotheses for how your traits are expected to mediate response in survivorship in these different permutations, then you will have a strong analytical framework in which to compare results from the two data sets.  In other words, the variety of circumstances associated with the two data sets will give you more power for saying when traits matter and when they don’t and, perhaps even more importantly, how they matter.

In any event, if you can better synthesize the two parts of your study, it will strengthen the contribution you make and broaden its relevance within paleontology and outside. 

2. Predictions (hypotheses) about survivorship and your traits

You make a few predictions about how the traits are likely to affect survivorship, but I encourage you to flesh these out in a more systematic fashion.  At the moment, your proposal relies primarily on model fitting and you expect interpretations to emerge from that exercise.  That’s fine, but it will leave you with less incisive conclusions than if you can offer a series of a priori expectations (grounded in established literature) that you can test with your data.  If nothing else, that framework provides you with a vehicle for making your research appeal to everyone with a vested interest in the established literature to which you refer. 

One key ingredient that I feel is missing is your expectation of how the trait-survivorship relationship is likely to change over time and space.  The  very notion of traits suggests that it should matter which particular trait an organism has, at its impact on fitness or survivorship may be context dependent.  The context varies in your study over time and place and your study would benefit from developing the expectations for different traits in the face of this variation.

For example, consider your mammal traits.  How is each one likely to be related to survivorship?  Is one more likely to be important than another?  Is the trait-survivorship relationship a static one or does it change over time and space? 

You mention body size, which is associated with nutrient requirements, which (in combination with pure scale) suggests that larger mammals require larger ranges.  That’s a good predictive hypothesis, but note that it is about individual mammals, not species ranges.  If large body size requires large home-range size and if large home range sizes translate into large species geographic ranges and if large ranges translate into increased survivorship, then you expect a positive relationship between survivorship and body size.  What about the other traits? 

Dietary category (which you don’t explain very precisely) is essentially trophic level, plus a division of herbivores into browsers and grazers.  We expect all trophic levels to be filled in all times and places (given that both your data sets come from a period in earth history when mammals have fully occupied the trophic hierarchy).  The proportion of one level to another (the “steepness” of the trophic pyramid) might be expected to vary with productivity, which is expected to vary with climate and vegetation type.  High trophic categories might be expected to have lower survivorship because they depend on the food chain below them (presuming that the food chain of the high level taxa includes mammals at its lower levels and not birds, lizards, or some other organism outside the scope of your analysis).  We might expect the relationship between trophic categories to be nearly universal, and for environmental productivity to be the main driver of changes survivorship within a category. 

Locomotor category (which you also don’t explain very precisely) is a different matter altogether.  Locomotor category in mammals is primarily an adaptation to a landscape or biome – open versus closed habitats, deep soil vs rocky substrate, arboreal vs. terrestrial.  We expect all species to have the same locomotor category in homogeneous landscapes and to have varied categories in mixed landscapes.  We know that landscapes (or biomes) have changed dramatically over the Cenozoic and that they vary spatially at any given time.  How do you expect the relative survivorship of arboreal and terrestrial species to change over the Cenozoic?  How do you expect them to vary from one continent to another at any given time?  How do you expect them to interact with body size or diet?  For example, the extensive tropical and subtropical forests of the early Eocene should favor small arboreal taxa at a variety of tropic levels (high productivity offers increased survivorship at the topic of the trophic chain).  Will they have large geographic ranges because of the broad extent of their habitat, or will they have small geographic ranges because of their small body size?   Here is an opportunity to test the relationship between body size and geographic range!  Open grasslands of the late Miocene, however, would favor either large cursorial grazers and large, swift predators, plus small granivores and carnivores – a system dimorphic in body size with potentially dimorphic geographic range sizes (another opportunity to test the body size-range size hypothesis), and probably a steeper trophic pyramid (lower productivity). 

3. Properties of survivorship compared to fitness

The factors influencing survivorship and its relationship to fitness still puzzle me.  We discussed this briefly at SVP.   First a technical question:  How is survivorship calculated?  Line 199 suggests that it is based on FADs and LADs, but not necessarily on extinction (as opposed to speciation) since LAD of a species does not by itself tell whether the species was lost to speciation or to extinction.  For your study it seems important to connect it only with extinction.  Speciation would imply success (=greater “fitness”?).  How will you distinguish between them?

Survivorship seems unlikely to have a straightforward relationship to trait, but rather to be related to the trait-environment interaction.  If a species has a trait and the trait works well in the environment, then we expect the species to have a longer survivorship than the same species in a different environment or a species with a different trait in the same environment.  We expect a decrease in survivorship if either the trait or the environment change.  Traits thus aren’t time-independent (or place independent).  Survivorship also seems to depend on the time of measurement.  For example, in the late Miocene there is a spread of grasslands.  Species living before that spread would have benefited from being forest browsers, perhaps equally well in the early Eocene and the Early Miocene.  Nevertheless, the survivorship of the early Miocene is expected to be shorter than the early Eocene browsers, if I understand the metric correctly.  Does this matter?  Here is where I think you need an explicit hypothesis about the trait-environment interaction to determine whether the trait really has an influence on survivorship.  Survivorship in forest browsers should be high when forest environments are widespread and should decline when they become more restricted.

Finally, the relationship to fitness.  Fitness is essentially reproductive success,.  Relative fitness is whether individuals with certain traits are more reproductively successful than other members of their population (individual fitness).  This kind of fitness is almost completely unrelated to survivorship of the species.  Population fitness is whether the population is reproducing at a level that sustains its population size.  If reproduction exceeds mortality, then the population size expands exponentially (and thus its geographic range may be likely to spread), but if mortality exceeds reproduction then the population size quickly crashes to zero, literally in just a few generations.  Survivorship is essentially time to extinction, but population fitness could be very high (indeed it is likely to be high by definition) until almost the moment of extinction.  Thus, survivorship is a proxy for rate of change in population fitness rather than for fitness itself.  The rate of change in fitness is probably related to changes in selective environment rather than traits (presuming that the traits are evolvable and respond to selection).  This brings me back to the theme about whether you want to be looking at trait x environment interactions rather than traits per se when you are looking at survivability. 

4. Terrestrial Cenozoic Biomes

You probably need geographically explicit data on biomes for your project.  Allan Graham has produced a series of works for North and South America that systematically characterize biomes though the Cenozoic.  You can probably generate coarse geographic maps of these biomes from his work.  See especially

Graham, A.  2011.  A Natural History of the New World:  The Ecology of Plants in the Americas.  University of Chicago Press: Chicago, Illinois.

See also Behrensmeyer et al., 1992, Terrestrial Ecosystems Through Time.  University of Chicago Press.

5. Evolvability of traits and survivorship

All of your traits evolve.  A bad trait will only have an impact on survivability if it doesn’t evolve.  For example, the turnover between forest browsers and grassland grazers that we expected above could be accomplished by extinction of the browsers and radiation of the grazers (low survivability in browsers and increased survivability in grazers) or by evolution and radiation of the browsers into grazers (high survivability of browsers).  How does this factor into your thinking?  One way to approach it would be through explicit hypotheses:  does one kind of trait evolve faster than another (e.g.,  body size > locomotion > trophic category)?  If so, then we might expect the slower evolving trait to have a bigger impact on survivorship. 

6. Properties of individual connectedness metrics (continental patterns verses biotic patterns)

I wonder to what extinct the connectedness metrics are likely to be related to the physical properties of continents rather than the properties of organisms per se.  Europe in the Paleogene and early Neogeene, for example, is a complex system of island arcs, shallow seas, and shifting connections, whereas North America is a large contiguous landmass.  Therefore we probably expect a different pattern of connectedness on the two continents regardless of other circumstances.  Is this important for interpreting your results?  Can you develop a null expectation for each continent that would help determine when train/biotic/environment/climate interactions are important independent of the continental topography?

\end{document}
