\documentclass{beamer}
\usepackage{amsmath,amsthm}
\usepackage{graphicx,microtype,parskip}
\usepackage{caption,subcaption,multirow}
\usepackage{attrib}

\frenchspacing

\usetheme{default}
\usecolortheme{whale}

\setbeamercolor{block title alerted}{fg=white,bg=darkgray}
\setbeamercolor{block body alerted}{fg=black,bg=lightgray}

\setbeamercolor{block title}{fg=white,bg=gray}
\setbeamercolor{block body}{fg=black,bg=lightgray}

\AtBeginSection[]
{
  \begin{frame}
    \tableofcontents[currentsection]
  \end{frame}
}


\title{Evolutionary paleoecology and\\ the biology of extinction}
\author{Peter D Smits}
\institute{Committee on Evolutionary Biology, University of Chicago}

\begin{document}

\begin{frame}
  \maketitle
\end{frame}

\begin{frame}
  \tableofcontents
\end{frame}


\section{Introduction and theory}

\begin{frame}
  \frametitle{Evolutionary paleoecology}
  \begin{quotation}
    \dots the consequences of distinct ecological factors on differential rate dynamics, particularly rates of faunal turnover and diversification.

    \attrib{Kitchell 1985 Paleobiology}
  \end{quotation}

  % Allmon says focus on traits and factors that may affect speciation
  % Tacit inclusion of extinction
\end{frame}

\begin{frame}
  \frametitle{Emergent properties}

  \begin{block}{Species level}
    Trait that cannot be reduced to organismal level
    
    Product of one or more traits/factors
  \end{block}

  % Diagram from grantham's paper? Or a DJ paper?
\end{frame}

\begin{frame}
  \frametitle{Range size}
  % effect on extinction
  Large range size means lower origination and extinction rates than small range size.

  Range size is emergent

  % DJ pictures?
\end{frame}

\begin{frame}
  \frametitle{Survival}

  \begin{columns}
    \begin{column}{0.5\textwidth}
      \begin{block}{Survival function}
        \begin{equation}
        S(t) = P(T > t)
        \label{eq:surv}
      \end{equation}
    \end{block}

    directly describes survival
    \end{column}
    \begin{column}{0.5\textwidth} 
      % figure of idealized survival curve
      % figure of Kaplan--Meier survival curve
    \end{column}
  \end{columns}
\end{frame}

\begin{frame}
  \frametitle{Hazard}
  \begin{block}{Hazard function}
    \begin{equation}
      h(t) = \lim_{\Delta t \to 0} \frac{P(t \leq T < t + \Delta t | T \geq t)}{\Delta t}
      \label{eq:haz}
    \end{equation}
  \end{block}

  % example hazard functions
  %   exponential, Weibull x2, lognormal
\end{frame}

\begin{frame}
  \frametitle{Law of Constant Extinction}

  Van Valen 1973 ``Red Queen'' paper.

  \begin{alertblock}{Definition}
      Survival probability and extinction risk is taxon--age independent.
  \end{alertblock}

  translation: hazard is constant with respect to time (\alert{exponential})

  \begin{equation}
    h(t) = \lambda \iff S(t) = \exp^{-\lambda t}
    \label{eq:constant}
  \end{equation}

  % figure from the Liow Red Queen review
\end{frame}

\begin{frame}
  \frametitle{Brachiopods and mammals: a comparison}

  Permian versus Cenozoic

  marine versus terrestrial

  warming verus cooling

  single region versus multiple regions 
\end{frame}

\begin{frame}
  \frametitle{Series of questions}
  \begin{itemize}
    \item generic level survival in brachiopods %and mammals
      \begin{itemize}
        \item effect of ecological traits (emergence)
        \item distribution of survival
      \end{itemize}
    \item specific level survival in mammals
      \begin{itemize}
        \item generic versus specific survival
        \item anagenesis/species:genus simulation
        \item distribution of survival
      \end{itemize}
    \item community connectedness in mammals
      \begin{itemize}
        \item global versus regional versus local scale processes
      \end{itemize}
  \end{itemize}
\end{frame}


\section{Brachiopods, environmental preference, and extinction}

\begin{frame}
  \frametitle{Ecological traits}
  \begin{itemize}
    \item substrate affinity
      \begin{itemize}
        \item physical, chemical
        \item availability
      \end{itemize}
    \item habitat preference
      \begin{itemize}
        \item energetics
        \item availability
      \end{itemize}
    \item affixing strategy
      \begin{itemize}
        \item energetics
        \item optimality
      \end{itemize}
  \end{itemize}
\end{frame}

\begin{frame}
  \frametitle{Substrate affinity}
  % image
\end{frame}

\begin{frame}
  \frametitle{Habitat preference}
  % image
\end{frame}

\begin{frame}
  \frametitle{Affixing strategy}
  % image
\end{frame}

\begin{frame}
  \frametitle{Assigning substrate and habitat}

  \begin{block}{Probability of assignment}
    \begin{equation}
      P(H_{1}|E) = \frac{P(E|H_{1})P(H_{1})}{P(E|H_{1})P(H_{1}) + P(E|H_{2})P(H_{2})}
      \label{eq:aff}
    \end{equation}
    \attrib{Simpson and Harnik 2009 Paleobiology}
  \end{block}
\end{frame}

\begin{frame}
  \frametitle{Models}
\end{frame}

\begin{frame}
  \frametitle{Preliminary results}
\end{frame}


\section{Ecology and survival in Cenozoic mammals}

\begin{frame}
  \frametitle{Ecological traits}
  \begin{itemize}
    \item dietary category
      \begin{itemize}
        \item energetics
        \item availability
      \end{itemize}
    \item locomotor category
      \begin{itemize}
        \item availability
        \item dispersal
      \end{itemize}
    \item body size
      \begin{itemize}
        \item energetics
        \item home range size
      \end{itemize}
  \end{itemize}
\end{frame}


\section{Community connectedness in Cenozoic mammals}

\begin{frame}
  \frametitle{Community connectedness}
  % figures from Sidor et al 2013
\end{frame}

\begin{frame}
  \frametitle{Average relative number of endemics}

  \begin{columns}
    \begin{column}{0.5\textwidth}
      \begin{equation}
        E = \frac{\sum_{i = 1}^{L} \frac{u_{i}}{n_{i}}}{L}
        \label{eq:end}
      \end{equation}
    \end{column}
    \begin{column}{0.5\textwidth}
      % pictures
    \end{column}
  \end{columns}
\end{frame}

\begin{frame}
  \frametitle{Average relative occupancy per taxon}

  \begin{columns}
    \begin{column}{0.5\textwidth}
      \begin{equation}
        Occ = \frac{\sum_{i = 1}^{N} \frac{l_{i}}{L}}{N}
        \label{eq:occ}
      \end{equation}
    \end{column}
    \begin{column}{0.5\textwidth}
      % pictures
    \end{column}
  \end{columns}
\end{frame}

\begin{frame}
  \frametitle{Biogeographic connectedness}

  \begin{columns}
    \begin{column}{0.5\textwidth}
      \begin{equation}
        BC = \frac{O - N}{LN - N}
        \label{eq:bc}
      \end{equation}
    \end{column}
    \begin{column}{0.5\textwidth}
      % pictures from Sidor et al. 2013
    \end{column}
  \end{columns}
\end{frame}

\begin{frame}
  \frametitle{Code length}

  % illustrate
  % map equation
\end{frame}

\begin{frame}
  \frametitle{Global versus regional versus local scale processes}
\end{frame}

\begin{frame}
  \frametitle{General expectations: dietary category}
\end{frame}

\begin{frame}
  \frametitle{General expectations: locomotor category}
\end{frame}

\begin{frame}
  \frametitle{Community connectedness of North America}
\end{frame}

\begin{frame}
  \frametitle{Community connectedness of Europe}
\end{frame}

\begin{frame}
  \frametitle{Community connectedness of South America}
\end{frame}

\begin{frame}
  \frametitle{Models}
\end{frame}

\begin{frame}
  \frametitle{Preliminary results}
\end{frame}


% end
\begin{frame}
  \frametitle{Acknowledgements}
  \begin{columns}
    \begin{column}{0.5\textwidth}
      \begin{itemize}
        \item \textbf{Committee}
          \begin{itemize}
            \item Kenneth D. Angielczyk (co-advisor)
            \item Michael J. Foote (co-advisor)
            \item P. David Polly
            \item Richard H. Ree
          \end{itemize}
        \item Discussion
          \begin{itemize}
            \item David Bapst
            \item Megan Boatright
            \item Colin Kyle
            \item Darcy Ross
            \item Elizabeth Sander
            \item Graeme Lloyd, Carl Simpson, Graham Slater
          \end{itemize}
      \end{itemize}
    \end{column}
    \begin{column}{0.5\textwidth}
      \includegraphics[height = 0.25\textheight, keepaspectratio = true]{figure/chicago} \\
      \includegraphics[height = 0.25\textheight, width = \textwidth, keepaspectratio = true]{figure/field} \\
    \end{column}
  \end{columns}
\end{frame}

%\begin{frame}
%  \frametitle{Theseus' ship}
%  \begin{quotation}
%    The ship wherein Theseus and the youth of Athens returned from Crete had thirty oars, and was preserved by the Athenians down even to the time of Demetrius Phalereus, for they took away the old planks as they decayed, putting in new and stronger timber in their place, in so much that this ship became a standing example among the philosophers, for the logical question of things that grow; one side holding that the ship remained the same, and the other contending that it was not the same.

%    \attrib{Plutarch, \textbf{Theseus}}
%  \end{quotation}
%\end{frame}

\end{document}
