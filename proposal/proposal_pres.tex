\documentclass{beamer}
\usepackage{amsmath,amsthm}
\usepackage{graphicx,microtype,parskip}
\usepackage{caption,subcaption,multirow}
\usepackage{attrib}

\frenchspacing

\title{Evolutionary paleoecology and the biology of extinction}
\author{Peter D Smits}
\institute{Committee on Evolutionary Biology, University of Chicago}

\AtBeginSection[]
{
  \begin{frame}
    \tableofcontents[currentsection]
  \end{frame}
}


\begin{document}

\begin{frame}
  \maketitle
\end{frame}

\begin{frame}
  \tableofcontents
\end{frame}


\section{Introduction and theory}

\begin{frame}
  \frametitle{Evolutionary paleoecology}
  \begin{quotation}
    \dots the consequences of distinct ecological factors on differential rate dynamics, particularly rates of faunal turnover and diversification.

    \attrib{Kitchell 1985 Paleobiology}
  \end{quotation}

  % allmon says focus on traits and factors that may affect speciation
  % tacit inclusion of extinction
\end{frame}

\begin{frame}
  \frametitle{Emergent properties}

  \begin{block}{\alert{Species level}}
    trait that cannot be reduced to organismal level. 
    
    product of one or more traits/factors.
  \end{block}
\end{frame}

\begin{frame}
  \frametitle{Survival}
\end{frame}

\begin{frame}
  \frametitle{Law of Constant Extinction}

  Van Valen 1973 ``Red Queen'' paper.

  \begin{block}{}
    survival probability/extinction is taxon--age independent.
  \end{block}

  % figure from the Liow Red Queen review
\end{frame}

\begin{frame}
  \frametitle{Brachiopods and mammals: a comparison}

  Permian versus Cenozoic

  marine versus terrestrial

  warming verus cooling

  single region versus multiple regions 
\end{frame}

\begin{frame}
  \frametitle{Series of questions}
\end{frame}


\section{Brachiopods, environmental preference, and extinction}

\begin{frame}
  \frametitle{Ecological traits}
  \begin{itemize}
    \item substrate affinity
    \item habitat preference
    \item affixing strategy
  \end{itemize}
\end{frame}

\begin{frame}
  \frametitle{Methodology}
   
  Survival: time till event

  FAD -- LAD
\end{frame}

\begin{frame}
  \frametitle{Preliminary results}
\end{frame}


\section{Ecology and survival in Cenozoic mammals}

\begin{frame}
  \frametitle{Ecological traits}
  \begin{itemize}
    \item dietary category
    \item locomotor category
    \item body size
  \end{itemize}
\end{frame}

\begin{frame}
  \frametitle{Expectations}
\end{frame}


\section{Community connectedness in Cenozoic mammals}

\begin{frame}
  \frametitle{Ecological traits}
\end{frame}

\begin{frame}
  \frametitle{Expectations}
\end{frame}

\begin{frame}
  \frametitle{Preliminary results}
\end{frame}


% end
\begin{frame}
  \frametitle{Acknowledgements}
  \begin{columns}
    \begin{column}{0.5\textwidth}
      \begin{itemize}
        \item \textbf{Committee}
          \begin{itemize}
            \item Kenneth D. Angielczyk (co-advisor)
            \item Michael J. Foote (co-advisor)
            \item P. David Polly
            \item Richard H. Ree
          \end{itemize}
        \item Discussion
          \begin{itemize}
            \item David Bapst
            \item Megan Boatright
            \item Colin Kyle
            \item Darcy Ross
            \item Elizabeth Sander
            \item Graeme Lloyd, Carl Simpson, Graham Slater
          \end{itemize}
      \end{itemize}
    \end{column}
    \begin{column}{0.5\textwidth}
      \includegraphics[height = 0.25\textheight, keepaspectratio = true]{figure/chicago} \\
      \includegraphics[height = 0.25\textheight, width = \textwidth, keepaspectratio = true]{figure/field} \\
    \end{column}
  \end{columns}
\end{frame}

%\begin{frame}
%  \frametitle{Theseus' ship}
%  \begin{quotation}
%    The ship wherein Theseus and the youth of Athens returned from Crete had thirty oars, and was preserved by the Athenians down even to the time of Demetrius Phalereus, for they took away the old planks as they decayed, putting in new and stronger timber in their place, in so much that this ship became a standing example among the philosophers, for the logical question of things that grow; one side holding that the ship remained the same, and the other contending that it was not the same.

%    \attrib{Plutarch, \textbf{Theseus}}
%  \end{quotation}
%\end{frame}

\end{document}
