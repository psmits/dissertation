\documentclass{beamer} 
\usepackage{amsmath,amsthm}
\usepackage{graphicx,microtype,parskip}
\usepackage{caption,subcaption,multirow}
\usepackage{attrib}

\frenchspacing

\usetheme{default}
\usecolortheme{whale}

\setbeamertemplate{navigation symbols}{}

\setbeamercolor{title}{fg=blue,bg=white}

\setbeamercolor{block title}{fg=white,bg=gray}
\setbeamercolor{block body}{fg=black,bg=lightgray}

\setbeamercolor{block title alerted}{fg=white,bg=darkgray}
\setbeamercolor{block body alerted}{fg=black,bg=lightgray}

\AtBeginSection[]
{
  \begin{frame}
    \tableofcontents[currentsection]
  \end{frame}
}


\title{Evolutionary paleoecology and \\the biology of extinction}
\author{Peter D Smits}
\institute{Committee on Evolutionary Biology, University of Chicago}

\begin{document}

\begin{frame}
  \begin{quotation}
    O species, stunned by your terror of chill death, why fear the Styx, why
    fear the ghosts and empty names, the stuff of poets, the spectres of a
    phantom world? [\dots] Everything changes, nothing dies: the spirit
    wanders, arriving here or there, and occupying whatever body it pleases,
    passing from a wild beast into a human being, from our body into a
    beast, but is never destroyed. As pliable wax, stamped with new designs,
    is no longer what it was; does not keep the same form; but is still one
    and the same.

    \attrib{Ovid, \underline{Metamorphoses}, book XV: 143-–175}
  \end{quotation}
\end{frame}

\begin{frame}
  \maketitle
\end{frame}

\begin{frame}
  \tableofcontents
\end{frame}


\section{Theory}

\begin{frame}
  \frametitle{Extinction}

  \begin{quotation}
    All species that have ever lived are, to a first approximation, dead.

    \tiny{\attrib{Raup 1986 \underline{The Nemesis Affair}}}
  \end{quotation}
\end{frame}

\begin{frame}
  \frametitle{Foundation}

  \begin{alertblock}{Question}
    Why do certain taxa go extinct while others do not?
  \end{alertblock}
\end{frame}

\begin{frame}
  \frametitle{Evolutionary paleoecology}
  \begin{quotation}
    \dots the consequences of distinct ecological factors on differential rate dynamics, particularly rates of faunal turnover and diversification.

    \tiny{\attrib{Kitchell 1985 \textit{Paleobiology}}}
  \end{quotation}

  \vspace{1.3cm}

  \begin{center}
    \begin{tabular}{@{}l@{}}adaptive zone\end{tabular} 
    \hspace{0.5cm}
    \(\rightarrow\) 
    \hspace{1cm}
    \begin{tabular}{@{}l@{}}speciation\\extinction\end{tabular}
  \end{center}
\end{frame}

\begin{frame}
  \frametitle{Van Valen's observation}

  \begin{center}
    \includegraphics[height = 0.7\textheight, keepaspectratio = true]{figure/liow}

    \tiny{\attrib{Liow et al. 2011 \textit{TREE}}}
  \end{center}
\end{frame}

\begin{frame}
  \frametitle{Law of Constant Extinction}

  \begin{alertblock}{Definition}
    Extinction rate, in a given adaptive zone, is taxon--age independent.

    \tiny{\attrib{Van Valen 1973 \textit{Evol. Theory}}}
  \end{alertblock}
\end{frame}

\begin{frame}
  \frametitle{Approach}

  \begin{block}{Framework and setup}
    \begin{itemize}
      \item background extinction
      \item traits related to environmental preference 
        \begin{itemize}
          \item ``bound'' adaptive zone
        \end{itemize}
      \item when/which/what processes dominate: \\global, regional, local 
    \end{itemize}
  \end{block}
\end{frame}

\begin{frame}
  \frametitle{Taxa}

  \begin{columns}
    \begin{column}{0.5\textwidth}
      \begin{block}{brachiopods}
        \begin{itemize}
          \item marine
          \item sessile
          \item Permian (\(\sim 47\) My)
          \item global warming
          \item Australia
        \end{itemize}
      \end{block}

      \begin{center}
        \includegraphics[height = 0.4\textheight, keepaspectratio = true]{figure/tattoo}
      \end{center}
    \end{column}
    \begin{column}{0.5\textwidth}
      \begin{block}{mammals}
        \begin{itemize}
          \item terrestrial
          \item motile
          \item Cenozoic (\(\sim 65\) My)
          \item global cooling
          \item North America, Europe, South America
        \end{itemize}
      \end{block}

      \begin{center}
        \includegraphics[height = 0.4\textheight, keepaspectratio = true]{figure/annyong}
      \end{center}
    \end{column}
  \end{columns}
\end{frame}

\begin{frame}
  \frametitle{Proposed studies}

  \begin{itemize}
    \item Australian Permian brachiopods
      \begin{itemize}
        \item traits: substrate, habitat, affixing strategy
        \item trait based survival 
        \item community connectedness (not shown)
      \end{itemize}
    \item Cenozoic mammals
      \begin{itemize}
        \item traits: dietary and locomotor categories, body size
        \item trait based survival
        \item community connectedness
      \end{itemize}
  \end{itemize}
\end{frame}


\begin{frame}
  \frametitle{Probability of survival}

  \begin{columns}
    \begin{column}{0.5\textwidth} 
      \includegraphics[height = 0.4\textheight, width = \textwidth, keepaspectratio = true]{figure/ideal}

      \includegraphics[height = 0.4\textheight, width = \textwidth, keepaspectratio = true]{figure/prac}

      \tiny{\attrib{Kleinbaum and Klein 2012}}
    \end{column}
    \begin{column}{0.5\textwidth}
      \begin{block}{Survival function}
        \[
          S(t) = P(T > t)
        \]

        \begin{itemize}
          \item \(T\): survival time \(\geq 0\) (duration)
          \item \(t\): specified time 
        \end{itemize}
      \end{block}
    \end{column}
  \end{columns}
\end{frame}

\begin{frame}
  \frametitle{Instantaneous potential of failure (extinction)}

  \begin{block}{Hazard function \(\equiv\) conditional failure rate}
    \[
      h(t) = \lim_{\Delta t \to 0} \frac{P(t \leq T < t + \Delta t | T \geq t)}{\Delta t}
    \]
  \end{block}

  \begin{center}
    \includegraphics[height = 0.5\textheight, width = \textwidth, keepaspectratio = true]{figure/hazard}
  \end{center}
\end{frame}

\begin{frame}
  \frametitle{Formalization of Van Valen}

  \begin{block}{Law of Constant Extinction}
    Hazard is constant with respect to time (\alert{exponential survival}).
  \end{block}

  \begin{center}
    \[
      h(t) = \lambda \iff S(t) = \exp^{-\lambda t}
    \]
  \end{center}
\end{frame}

\begin{frame}
  \frametitle{Community connectedness}
  \begin{definition}
    The degree to which localities are composed of endemic versus cosmopolitan taxa, and how similar this ratio is across localities.
  \end{definition}
\end{frame}

\begin{frame}
  \frametitle{Biogeographic networks}
  \begin{center}
    \includegraphics[height = 0.6\textheight, width = \textwidth, keepaspectratio = true]{figure/vilhena}  % need higher resolution picture

    \tiny{\attrib{Vilhena \textit{et al.} 2013 \textit{Sci. Reports}}}
  \end{center}
\end{frame}

\begin{frame}
  \frametitle{Average relative number of endemics}

  \begin{columns}
    \begin{column}{0.5\textwidth}
      \begin{center}
        \includegraphics[height = 0.5\textheight, width = \textwidth, keepaspectratio = true]{figure/sim_graph}

        \begin{align*}
          u &= \{1, 2, 1\}\\
          n &= \{6, 5, 6\}\\
          L &= 3\\
          E &\approx 0.24
        \end{align*}
      \end{center}
    \end{column}
    \begin{column}{0.5\textwidth}
      \[
        E = \frac{\sum_{i = 1}^{L} \frac{u_{i}}{n_{i}}}{L}
      \]

      \begin{itemize}
        \item \(L\): number of localities
        \item \(u\): number of taxa unique to a locality
        \item \(n\): number of taxa at a locality
        \item \(0 \leq E \leq 1\)
      \end{itemize}
    \end{column}
  \end{columns}
\end{frame}

\begin{frame}
  \frametitle{Average relative occupancy per taxon}

  \begin{columns}
    \begin{column}{0.5\textwidth}
      \begin{center}
        \includegraphics[height = 0.5\textheight, width = \textwidth, keepaspectratio = true]{figure/sim_graph}

        \begin{align*}
          l &= \{1, 2, 3, 1, 1, 2, 3, 3, 1\}\\
          L &= 3\\
          N &= 9\\
          Occ &\approx 0.63 
        \end{align*}
      \end{center}
    \end{column}
    \begin{column}{0.5\textwidth}
      \[
        Occ = \frac{\sum_{i = 1}^{N} \frac{l_{i}}{L}}{N}
      \]

      \begin{itemize}
        \item \(N\): total number of taxa
        \item \(l\): number of localities a taxon occurs at
        \item \(L\): number of localities
        \item \(0 \leq Occ \leq 1\)
      \end{itemize}
    \end{column}
  \end{columns}
\end{frame}

\begin{frame}
  \frametitle{Biogeographic connectedness}

  \begin{columns}
    \begin{column}{0.5\textwidth}
      \begin{center}
        \includegraphics[height=0.8\textheight,width=\textwidth,keepaspectratio=true]{figure/bc}

        \tiny{\attrib{Sidor et al. 2013 \textit{PNAS}}}
      \end{center}
    \end{column}
    \begin{column}{0.5\textwidth}
      \[
        BC = \frac{O - N}{LN - N}
      \]

      \begin{itemize}
        \item \(O\): number of occurrences
        \item \(N\): total number of taxa
        \item \(L\): number of localities
        \item \(0 \leq BC \leq 1\)
      \end{itemize}
    \end{column}
  \end{columns}
\end{frame}

\begin{frame}
  \frametitle{Code length}
  \begin{center}
    \includegraphics[height=0.8\textheight,width=\textwidth,keepaspectratio=true]{figure/map1}

    \tiny{\attrib{Rosvall and Bergstrom 2008 \textit{PNAS}}}
  \end{center}
\end{frame}

\begin{frame}
  \frametitle{Code length}
  \begin{center}
    \includegraphics[height=0.8\textheight,width=\textwidth,keepaspectratio=true]{figure/map2}

    \tiny{\attrib{Rosvall and Bergstrom 2008 \textit{PNAS}}}
  \end{center}
\end{frame}

\begin{frame}
  \frametitle{Code length}
  \begin{center}
    \includegraphics[height=0.8\textheight,width=\textwidth,keepaspectratio=true]{figure/map3}

    \tiny{\attrib{Rosvall and Bergstrom 2008 \textit{PNAS}}}
  \end{center}
\end{frame}

\begin{frame}
  \frametitle{Code length}
  \begin{center}
    \includegraphics[height=0.8\textheight,width=\textwidth,keepaspectratio=true]{figure/map4}

    \tiny{\attrib{Rosvall and Bergstrom 2008 \textit{PNAS}}}

  \end{center}
\end{frame}

\begin{frame}
  \frametitle{Global versus regional versus local scale}

  \begin{columns}
    \begin{column}{0.55\textwidth}
      \includegraphics[height=0.7\textheight,width=\textwidth,keepaspectratio=true]{figure/permian}

      \tiny{\attrib{Sidor et al. 2013 \textit{PNAS}}}
    \end{column}
    \begin{column}{0.45\textwidth}
      \begin{itemize}
        \item global
          \begin{itemize}
            \item corr w/ global climate
            \item multiple regions corr
          \end{itemize}
        \item regional
          \begin{itemize}
            \item \(\downarrow E\), \(\uparrow Occ\), \\\(\uparrow BC\), \(\uparrow\) code
          \end{itemize}
        \item local
          \begin{itemize}
            \item \(\uparrow E\), \(\downarrow Occ\), \\\(\downarrow BC\), \(\downarrow\) code
          \end{itemize}
        \item \alert{not mutually exclusive}
      \end{itemize}
    \end{column}
  \end{columns}
\end{frame}


\section{Brachiopods}

\begin{frame}
  \frametitle{Brachiopods, environmental preference, and extinction}

  \begin{block}{Questions}
    \begin{itemize}
      \item Are traits related to environmental preference correlated with survival?
      \item Which trait(s) best model extinction? One or more?
      \item Is global climate important to extinction?
    \end{itemize}
  \end{block}
\end{frame}

\begin{frame}
  \frametitle{Substrate affinity}

  \begin{columns}
    \begin{column}{0.5\textwidth}
      \begin{center}
        \includegraphics[height = 0.4\textheight, keepaspectratio = true]{figure/miller}

        \tiny{\attrib{Miller and Connoly 2001 \textit{Paleobio.}}}

        \includegraphics[height = 0.4\textheight, keepaspectratio = true]{figure/foote}

        \tiny{\attrib{Foote 2006 \textit{Paleobio.}}}
      \end{center}
    \end{column}
    \begin{column}{0.5\textwidth}
      \begin{itemize}
        \item carbonates, clastics, mixed
          \begin{itemize}
            \item physio-chemical 
            \item availability
          \end{itemize}
        \item Pharenozoic decrease carbonates:clastics
          \begin{itemize}
            \item predicted longevity: \\clastics \(>\) carbonates
          \end{itemize}
      \end{itemize}
    \end{column}
  \end{columns}
\end{frame}

\begin{frame}
  \frametitle{Habitat preference}

  \begin{columns}
    \begin{column}{0.5\textwidth}
      \begin{center}
        \includegraphics[height = 0.8\textheight, width = \textwidth, keepaspectratio = true]{figure/onoff}

        \tiny{\attrib{Jablonski \textit{et al.} 1983 \textit{Science}}}
      \end{center}
    \end{column}
    \begin{column}{0.5\textwidth}
      \begin{itemize}
        \item on-shore, off-shore, none
          \begin{itemize}
            \item above/below \\storm wave base
            \item energetics, availability
          \end{itemize}
        \item offshore \(>\) onshore
      \end{itemize}
    \end{column}
  \end{columns}
\end{frame}

\begin{frame}
  \frametitle{Affixing strategy}

  \begin{columns}
    \begin{column}{0.5\textwidth}
      \begin{center}
        \includegraphics[height = 0.7\textheight, width = \textwidth, keepaspectratio = true]{figure/methods}

        \tiny{\attrib{Johansen 1989 \textit{Paleo\(^3\)}}}
      \end{center}
    \end{column}
    \begin{column}{0.5\textwidth}
      \begin{itemize}
        \item environmental energetics \\and material (mud)
        \item pedunculate, reclining, cementing
          \begin{itemize}
            \item endemics duration: \\reclining \(>\) others
            \item cosmopolitan duration: \\ped./cement \(>\) others
          \end{itemize}
        \item pedunculate:on-shore, reclining:off-shore
      \end{itemize}
    \end{column}
  \end{columns}
\end{frame}

\begin{frame}
  \frametitle{Permian climate}
  \begin{center}
    \includegraphics[height = 0.8\textheight, width = \textwidth, keepaspectratio = true]{figure/glacial}

    \tiny{\attrib{Birgenheier \textit{et al.} 2010 \textit{Paleo\(^3\)}}}
  \end{center}
\end{frame}

\begin{frame}
  \frametitle{Assigning substrate and habitat}

  \begin{block}{Probability of assignment}
    \begin{align*}
      P(H_{1}|E) &= \frac{P(E|H_{1})P(H_{1})}{P(E|H_{1})P(H_{1}) + P(E|H_{2})P(H_{2})} \\
      P(E|H) &= \binom{n}{k} p^{k}(1 - p)^{n - k}
    \end{align*}

    \begin{itemize}
      \item \(p\): proportion of all collections (e.g) carbonate
      \item \(n\): total \# taxon occurrences
      \item \(k\): of \(n\), \# (e.g.) carbonate occurrences
    \end{itemize}

    \tiny{\attrib{Simpson and Harnik 2009 \textit{Paleobiology}}}
  \end{block}
\end{frame}

\begin{frame}
  \frametitle{Analysis}

  \begin{columns}
    \begin{column}{0.5\textwidth}
      \begin{center}
        \includegraphics[height = 0.8\textheight, width = \textwidth, keepaspectratio = true]{figure/australia}

        \tiny{\attrib{Clapham and James 2008 \textit{Palaios}}}
      \end{center}
    \end{column}
    \begin{column}{0.5\textwidth}
      \begin{itemize}
        \item genus FAD--LAD
        \item time-independent traits
          \begin{itemize}
            \item preliminary substrate following Foote 2006 \textit{Paleobio.}
            \item preliminary habitat following Kiessling \textit{et al.} 2007 \textit{Paleo\(^3\)}
          \end{itemize}
        \item time-dependent climate 
      \end{itemize}
    \end{column}
  \end{columns}
\end{frame}

\begin{frame}
  \frametitle{K-M curve substrate}
  \begin{center}
    \includegraphics[height = 0.8\textheight, width = \textwidth, keepaspectratio = true]{figure/km_aff}
  \end{center}
\end{frame}

\begin{frame}
  \frametitle{K-M curve habitat}
  \begin{center}
    \includegraphics[height = 0.8\textheight, width = \textwidth, keepaspectratio = true]{figure/km_hab}
  \end{center}
\end{frame}

\begin{frame}
  \frametitle{Preliminary results: model comparison}

  \input{model_tabs.tex}
\end{frame}


\section{Mammals}

\begin{frame}
  \frametitle{Ecology and survival in Cenozoic mammals}

  \begin{block}{Questions}
    \begin{itemize}
      \item How do traits related to range size relate to survival?
      \item Which trait(s) best model survival? One or more?
      \item Is climatic change important for modeling survival?
      \item Are patterns of survival different between genera and species?
    \end{itemize}
  \end{block}

\end{frame}

\begin{frame}
  \frametitle{Dietary category}
  \begin{columns}
    \begin{column}{0.5\textwidth}
      \begin{center}
        \includegraphics[height=0.4\textheight, width=\textwidth, keepaspectratio=true]{figure/dietdiv}

        \tiny{\attrib{Price \textit{et al.} 2012 \textit{PNAS}}}

        \includegraphics[height=0.4\textheight,width=\textwidth,keepaspectratio=true]{figure/jernvall}

        \tiny{\attrib{Jernvall and Fortelius 2004 \textit{Am. Nat.}}}
      \end{center}
    \end{column}
    \begin{column}{0.5\textwidth}
      \begin{itemize}
        \item trophic hierarchy \\(stability \(\to\) duration)
          \begin{itemize}
            \item herb: most stable, \\longest duration
            \item carni: least stable, \\shortest duration
            \item omni: avg. stability, \\avg. duration
          \end{itemize}
      \end{itemize}
    \end{column}
  \end{columns}
\end{frame}

\begin{frame}
  \frametitle{Locomotor category}

  \begin{columns}
    \begin{column}{0.5\textwidth}
      \includegraphics[height=0.8\textheight, width=\textwidth, keepaspectratio=true]{figure/scherler}

      \tiny{\attrib{Scherler \textit{et al.} 2013 \textit{Swiss. J. Geosci.}}}
    \end{column}
    \begin{column}{0.5\textwidth}
      \begin{itemize}
        \item Paleogene \(\to\) Neogene
          \begin{itemize}
            \item open \(\to\) closed environment
          \end{itemize}
        \item predictions
          \begin{itemize}
            \item arboreal: \\Paleogene \(>\) Neogene
            \item ground dwelling: \\Paleogene \(<\) Neogene
            \item scansorial: \\Paleogene \(\approx\) Neogene
          \end{itemize}
      \end{itemize}
    \end{column}
  \end{columns}
\end{frame}

\begin{frame}
  \frametitle{Body size}

  \begin{columns}
    \begin{column}{0.5\textwidth}
      \includegraphics[height=0.4\textheight, width=\textwidth, keepaspectratio=true]{figure/liowmam}

      \tiny{\attrib{Liow \textit{et al.} 2008 \textit{PNAS}}}

      \includegraphics[height=0.4\textheight, width=\textwidth, keepaspectratio=true]{figure/susumu}

      \tiny{\attrib{Tomiya 2013 \textit{Am. Nat.}}}
    \end{column}
    \begin{column}{0.5\textwidth}
      \begin{itemize}
        \item \(\uparrow\) body size, \(\uparrow\) energy req, \\\(\uparrow\) range size, \(\downarrow\) extinction
        \item \(\uparrow\) body size, \(\downarrow\) rep. rate, \\\(\uparrow\) extinction 
      \end{itemize}
    \end{column}
  \end{columns}
\end{frame}

\begin{frame}
  \frametitle{Analysis}

  \begin{itemize}
    \item data: genus, species FAD--LAD
      \begin{itemize}
        \item NA: PBDB
        \item Europe: PBDB, NOW
        \item SA: collections, compilations
      \end{itemize}
    \item traits: time-indep. covariates
    \item climate: time-dep. covariate
    \item Paleogene versus Neogene
  \end{itemize}
\end{frame}


%\section{Mammal community connectedness}

\begin{frame}
  \frametitle{Community connectedness in Cenozoic mammals}

  \begin{block}{Questions}
    \begin{itemize}
      \item How does the ratio between endemic and cosmopolitan taxa change over time?
      \item Is there a single global pattern or does each regions have a different patterns?
      \item Do these patterns differ between ecological categories?
      \item Is global climate change an important predictor of these patterns?
    \end{itemize}
  \end{block}
\end{frame}

\begin{frame}
  \frametitle{Global expectations: locomotor category}

  \begin{columns}
    \begin{column}{0.6\textwidth}
      \includegraphics[height=0.8\textheight,width=\textwidth,keepaspectratio=true]{figure/stromberg}

      \tiny{\attrib{Str\"{o}mberg \textit{et al.} 2013 \textit{Nature Com.}}}
    \end{column}
    \begin{column}{0.4\textwidth}
      \begin{itemize}
        \item arboreal
          \begin{itemize}
            \item \(\uparrow E\), \(\uparrow\) code
            \item \(\downarrow BC\), \(\downarrow Occ\)
          \end{itemize}
        \item ground dwelling
          \begin{itemize}
            \item \(\downarrow E\), \(\downarrow\) code
            \item \(\uparrow BC\), \(\uparrow Occ\)
          \end{itemize}
        \item scansorial
          \begin{itemize}
            \item constant \(\lor\) random
          \end{itemize}
      \end{itemize}
    \end{column}
  \end{columns}
\end{frame}

\begin{frame}
  \frametitle{Global expectations: dietary category}

  \begin{columns}
    \begin{column}{0.6\textwidth}
      \includegraphics[height=0.7\textheight,width=\textwidth,keepaspectratio=true]{figure/jernvall}

      \tiny{\attrib{Jernvall and Fortelius 2004 \textit{Am. Nat.}}}
    \end{column}
    \begin{column}{0.4\textwidth}
      \begin{itemize}
        \item herbivore
          \begin{itemize}
            \item most like all taxa
          \end{itemize}
        \item carnivore
          \begin{itemize}
            \item constant \(\lor\) corr w/ herbivores
          \end{itemize}
        \item omnivore
          \begin{itemize}
            \item constant \(\lor\) random
          \end{itemize}
      \end{itemize}
    \end{column}
  \end{columns}
\end{frame}

\begin{frame}
  \frametitle{Preliminary results: NA, Eur}

  \begin{center}
    \includegraphics[height = 0.8\textheight, width = \textwidth, keepaspectratio = true]{figure/gen_bin}
  \end{center}
\end{frame}

\begin{frame}
  \frametitle{Preliminary results: locomotor category NA}

  \begin{center}
    \includegraphics[height = 0.8\textheight, width = \textwidth, keepaspectratio = true]{figure/na_lf}
  \end{center}
\end{frame}

\begin{frame}
  \frametitle{preliminary results: locomotor category Eur}

  \begin{center}
    \includegraphics[height = 0.8\textheight, width = \textwidth, keepaspectratio = true]{figure/er_lf}
  \end{center}
\end{frame}

\begin{frame}
  \frametitle{Preliminary results: dietary category NA}

  \begin{center}
    \includegraphics[height = 0.8\textheight, width = \textwidth, keepaspectratio = true]{figure/na_dt}
  \end{center}
\end{frame}

\begin{frame}
  \frametitle{Preliminary results: dietary category Eur}

  \begin{center}
    \includegraphics[height = 0.8\textheight, width = \textwidth, keepaspectratio = true]{figure/er_dt}
  \end{center}
\end{frame}


\section{Summary}

\begin{frame}
  \frametitle{Fundamental}

  \begin{alertblock}{Question}
    Why do some taxa go extinct while others do not?
  \end{alertblock}
\end{frame}

\begin{frame}
  \frametitle{Evolutionary paleoecological rephrasing}

  \begin{block}{Question}
    How does a taxon's adaptive zone affect extinction risk?
  \end{block}
\end{frame}

\begin{frame}
  \frametitle{``Testing'' the Law of Constant Extinction}

  \begin{block}{Liow et al. 2011 \textit{TREE}}
    Only applies during periods of relatively \alert{constant} environment.
  \end{block}

  \vspace{1cm}

  \begin{center}
    Measure, analyze, model changing environment.
  \end{center}
\end{frame}


\begin{frame}
  \frametitle{Ask the following\dots}

  \begin{center}
    Is there a \alert{general pattern} of extinction?

    \vspace{0.75cm}

    \alert{What} traits matter for extinction and \alert{when}?

    \vspace{0.75cm}

    \alert{How} do they matter?
  \end{center}
\end{frame}

% need quotation to close the talk


\begin{frame}
  \frametitle{Acknowledgements}
  \begin{columns}
    \begin{column}{0.5\textwidth}
      \begin{itemize}
        \item \textbf{Committee}
          \begin{itemize}
            \item Kenneth D. Angielczyk (co-advisor)
            \item Michael J. Foote (co-advisor)
            \item P. David Polly
            \item Richard H. Ree
          \end{itemize}
        \item Discussion
          \begin{itemize}
            \item David Bapst, Megan Boatright, Ben Frable, Colin Kyle, Darcy Ross, Liz Sander
            \item John Alroy, Graeme Lloyd, Carl Simpson, Graham Slater
          \end{itemize}
      \end{itemize}
    \end{column}
    \begin{column}{0.5\textwidth}
      \includegraphics[height = 0.3\textheight, keepaspectratio = true]{figure/chicago} \\
      \includegraphics[height = 0.3\textheight, width = 0.5\textwidth, keepaspectratio = true]{figure/field} \\
    \end{column}
  \end{columns}
\end{frame}



\appendix
\section{Further concerns}

\begin{frame}
  \frametitle{Compressing a network}

  \begin{block}{Map equation \tiny{\attrib{Rosvall and Bargstrom 2008 \textit{PNAS}}}}
    \begin{align*}
      L(\textbf{M}) &= q_{\curvearrowright}H(\mathcal{Q}) + \sum^{m}_{i = 1} p^{i}_{\circlearrowright}H(\mathcal{P}^{i})
    \end{align*}

    \begin{itemize}
      \item \(\textbf{M}\): module partion of \textit{n} nodes in \textit{m} partitions
      \item \(L(\textbf{M})\): network code length 
      \item \(q_{\curvearrowright}\): P(walk switches modules)
      \item \(H(\mathcal{Q})\): entropy module codewords
      \item \(H(\mathcal{P}^{i})\): entropy within--module
      \item \(p^{i}_{\circlearrowright}\): rate within--module use
    \end{itemize}
  \end{block}
\end{frame}

\begin{frame}
  \frametitle{Effect of differential preservation on comparisons of survival}

  \begin{columns}
    \begin{column}{0.5\textwidth}
      \begin{center}
        \includegraphics[height = 0.4\textheight, width = \textwidth, keepaspectratio = true]{figure/raup}

        \tiny{\attrib{Raup 1975 \textit{Paleobio.}}}

        \includegraphics[height = 0.4\textheight, width = \textwidth, keepaspectratio = true]{figure/sepkoski}

        \tiny{\attrib{Sepkoski 1975 \textit{Paleobio.}}}
      \end{center}
    \end{column}
    \begin{column}{0.5\textwidth}
      two groups in four scenarios
      \begin{itemize}
        \item \(=\) birth, death; \\\(=\)preservation
        \item \(=\) birth, death; \\\(!=\)preservation
        \item \(!=\) birth, death; \\\(=\) preservation
        \item \(!=\) birth, death; \\\(!=\)preservation
      \end{itemize}
    \end{column}
  \end{columns}
\end{frame}

\begin{frame}
  \frametitle{Phylogeneic similarity of communities}

  \begin{columns}
    \begin{column}{0.5\textwidth}
      \includegraphics[height = 0.8\textheight, width = \textwidth,  keepaspectratio = true]{figure/webb}

      \tiny{\attrib{Webb \textit{et al.} 2002 \textit{Ann. Rev. Ecol. Syst.}}}
    \end{column}
    \begin{column}{0.5\textwidth}
      \begin{itemize}
        \item informal time scaled phylogeny \\(taxonomy tree)
        \item measures
          \begin{itemize}
            \item relative mean pairwise distance between taxa at locality
            \item mean locality phylogenetic species variability (Helmus \textit{et al.} 2007 \textit{Am. Nat})
          \end{itemize}
      \end{itemize}
    \end{column}
  \end{columns}
\end{frame}


\end{document}
