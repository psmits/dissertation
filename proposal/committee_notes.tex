\documentclass[12pt,letterpaper]{article}

\usepackage{amsmath, amsthm}
\usepackage{graphicx}
\usepackage{microtype, parskip}
\usepackage{caption, subcaption, multirow, morefloats, rotating, longtable}
\usepackage{hyperref}
\usepackage[numbers,sort&compress]{natbib}
\usepackage{authblk, attrib, fullpage}
\usepackage{lineno}

\begin{document}

\section*{Michael}
The question of relative importance of different determinants of taxonomic longevity is very important.  

An obvious challenge in assessing relative importance is that factors that are known with more uncertainty can spuriously appear to be less important.

Not sure I understand the suggestion that macroevolution is a metaphor.

What exactly do you mean by strong and weak definitions?  And since definition, metaphor, and metonymy are clearly different concepts, why do we need to pose the question, in the context of a scientific discussion, as to how they differ?

The 1975 Sepkoski paper on survivorship curves is important.  You might also want to consider my 1996, 1997, and 2001 papers on the topic. 

I'd like to see more technical detail on connectivity, complexity, and code length.  How are these measured, and what is the rationale behind the measures?

I think you'll be happier in the long run if you develop your own code rather than using the paleotree package.

Weakness of the Alroy, Koch, and Zachos paper was using a climate proxy that was spatially far removed from the data.  You risk falling into the same trap.  Are there alternatives?

For what you're doing, I can't help wondering whether the distinctions between carbonate vs. clastic and onshore vs. offshore are too coarse.

Discussion seems to imply equivalence between epicontinental and onshore.

Why should environment, habitat, and fixation strategy affect taxonomic duration?

Maybe variance in habitat etc. rather than habitat per se is more important.  (See Foote \& Miller 2013, Paleobiology.)


\section*{General}
link traits more strongly. what is the expected effect on survivorship? biological relevance of traits?

emphasize testing aspects, such as effect of climate change.

natural scale of locality size. Miller et al. 2009 Paleobiology. what is the natural locality size? in event of emergency try 2 by 2 then 5 by 5.

temporal decay of community connectedness. Faunmap to PBDB.

maybe not keep the South American mammals.

brachiopod traits too coarse (environement, lithology). better resolution. Foote implies consider dropping them.

premable is unintreresting. why should metaphor be interesting? no contensions or stakes. just definitions.

weave theory/philosophy into the projects better. how and why do they address these issues/concerns?

polished draft for next meeting.

\end{document}
