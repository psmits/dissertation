\documentclass{article}
\usepackage{parskip}
\frenchspacing

\begin{document}

What you mean by evolutionary ecology (opening paragraph) is fine for your purposes, but if you turn this into a DDIG proposal you will want to be less restrictive.

This is highly debatable.  What is the evidence that most extinction is biologically selective?

What do you think about equating FAD with origin and LAD with extinction?  (Oh, I see you get back to this later.)

Why model with Weibull, in which the parameters don't have an obvious biological interpretation, instead of with a birth-death model, in which genus survivorship emerges from species-level branching and extinction rates?

Explain how you will use the simulation results to assess whether an observed difference in survival is believable.  Don't you need an empirical estimate of sampling rates?

Actually, the level of distortion depends not only on the sampling/preservation rate but also on the true extinction rate.  So two groups with equal sampling and different extinction rates will, I think, be distorted differently.

You will want to explore other schemes for assigning affinities.

I am still concerned that you are not looking directly at the possible proximal causes of survival.  For example, you will test whether substrate affinity correlates with survival, the rationale being that genera that prefer more widespread substrates should be favored.  But if that is the causal link, don't you need to test directly whether those genera are more widespread?

I'm still not sure I see what's at stake with the community connectedness work---what the intellectual tension is that motivates it.  But we can talk about that.  For the DDIG, you really should be less jargony about this and present everything in terms that area easily interpreted biologically.   For example, don't say ``Mammals increase in code length from time 1 to time 2,'' but instead, ``Provinciality increases from time 1 to time 2.''

\end{document}
