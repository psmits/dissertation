\documentclass[12pt,letterpaper]{article}
\usepackage{amsmath, amsthm}
\usepackage{parskip, graphicx, microtype}
\usepackage{caption, subcaption, multirow}
\usepackage{morefloats, hyperref}
\usepackage{rotating, longtable}
\usepackage[sort&compress]{natbib}
\usepackage{authblk}
\usepackage{fullpage}

\usepackage[]{lineno}

\frenchspacing

\title{Evolution of life history traits and evolutionary paleoecology}
\author[1]{Peter D Smits}
\affil[1]{\footnotesize{\href{mailto:psmits@uchicago.edu}{psmits@uchicago.edu}, Committee on Evolutionary Biology, University of Chicago}}

\begin{document}
\maketitle

\section{Theoretical framework}
Evolutionary paleoecology has been defined as the study of the effect of ecological characters expressed at any level on the macroevolutionary process \citep{Kitchell1985a}. While macroevolution is defined as the long term speciation (\(\lambda\)) and extinction (\(\mu\)) dynamics \citep{Jablonski2008a}, this does not remain the sole manner of discussing macroevolutionary dynamics \citep{Kitchell1985a,Kitchell1990}. Instead, macroevolution can be discussed in terms of the dynamics of differential fitness. Here fitness is defined as the expected time till extinction \citep{Cooper1984} which can be considered a universal statement of fitness.

Expected time till extinction is defined
\begin{equation}
  E[t_{ext}] = \sum p t
  \label{eq:ete}
\end{equation}
where \(p\) is the probability that the subject of interest goes extinct and \(t\) is time \citep{Cooper1984}.


\section{Effect of life history on survival in Permian brachiopods}

\section{Evolution of correlation in life history traits in brachiopods}

\section{Cosmopolitan versus endemic dynamics in Cenozoic terrestrial mammals}

\section{UNNAMED CHAPTER}

\bibliographystyle{abbrvnat}
\bibliography{proposal}

\end{document}
