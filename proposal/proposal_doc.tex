\documentclass[12pt,letterpaper]{article}

\usepackage{amsmath, amsthm}
\usepackage{graphicx}
\usepackage{microtype, parskip}
\usepackage{caption, subcaption, multirow, morefloats, rotating, longtable}
\usepackage{hyperref}
\usepackage[numbers,sort&compress]{natbib}
\usepackage{authblk, attrib, fullpage}
\usepackage{lineno}


\begin{document}
\setcounter{secnumdepth}{0}

\begin{titlepage}
  \begin{center}
    \huge{Evolutionary paleoecology and the biology of extinction}

    \vspace{1.5cm}

    \large{Peter D. Smits \\}
    \footnotesize{\href{mailto:psmits@uchicago.edu}{psmits@uchicago.edu}}

    \vspace{1.5cm}

    Dissertation Proposal Hearing \\
    \today \\
    Commmittee on Evolutionary Biology \\
    The University of Chicago

    \vspace{1.5cm}

    \textit{Committee} \\
    Dr. Michael J. Foote (co-advisor) \\
    Dr. Kenneth D. Angielczyk (co-advisor) \\
    Dr. Richard H. Ree \\
    Dr. P. David Polly
  \end{center}
\end{titlepage}

\linenumbers
\modulolinenumbers[2]


\section{Introduction and theoretical framework}

\subsection{Evolutionary paleoecology}
Evolutionary paleoecology is defined as the study of the effects of ecological traits and factors on differential rate dynamics, particularly rates of faunal turnover and diversification \citep{Kitchell1985a}. Ecological traits and factors are any and all traits expressed by a taxon, at any level, that are involved with biotic--biotic or biotic--abiotic interactions. Diversification is the difference between origination and extinction and is the net pattern of macroevolution. The study of evolutionary paleoecology is then the link between environmental interactions and macroevolution. As a corrolary to \citet{Kitchell1985a}'s definition, \citet{Allmon1994} states that in order to correctly link ecological interactions to macroevolution, one must focus on the specific traits and factors that may affect the speciation process. Tacitly included in this is then the study of the biology of extinction and how it relates to ecological properties and interactions \citep{Kitchell1990}.

\citet{Simpson1944} defined also defined the environment braodly as to include all possible biotic and abiotic interactions as well as the organism itself. This usage of ``environment'' is frequently misinterpreted or unknown and has lead to a great deal of confusion thus it is important to note here \citep{VanValen1973,Barnosky2001a,Liow2011a}. Related to this, \citet{Simpson1944} defined the set of all biotic and abiotic interactions that a lineage experiences as the ``adaptive zone.'' The adaptive zone represents one of the fundamental metaphors in paleobiology and can be considered the adaptive landscape through time \citep{Simpson1944,Simpson1953}. 

It is under this framework that I purpose to study how ecological traits associated with range size have affected both the availability of biotic interactions and differential survivorship. I will be studying two very distantly related and biotically different groups: Cenozoic mammals and Permian brachiopods. Both of these groups are considered to have very good fossil records able to reflect massive long term evolutionary patterns \citep{Mark1977}. These two groups were also chosen because they experienced opposite climatic shifts (cooling and warming respectively) and represent a terrestrial and marine system respectively. Additionally, the ecological traits associated with range size (described below) are fundamentally very different. For example, the mammalian traits being motile throughout life and involve biotic--biotic interactions while the brachiopod traits are principally associated environmental preference and biotic--abiotic interactions.

% Jurassic of Europe Kiessling Aberhaan
% John Hunter OHIO STATE
% -Watson Seminar


\subsection{Paleobiological theory}
Extinction, when not during periods of ``mass extinction,'' is widely considered non-random with respect to biology \citep{Jablonski1986}. Additionally, times of ``background extinction'' represent the vast majority of geological time compared to periods of ``mass extinction.'' The exact definition and distinction between periods of background and mass extinction is based on the relative intensity of extinction at a particular time versus times preceding and following. Simply put, the major indicator of a mass extinction is an event that is across taxonomic categories and the biotic interactors before and after the event are dramatically different \citep{Jablonski1986,Jablonski2005,Kitchell1986,Kitchell1991}.

Paleobiology is the study of life over time and the processes that generate the observed patterns in diversity and disparity. Intimately related to paleobiology is the concept of macroevolution here defined as the pattern of speciation and extinction over time \citep{Jablonski2008a}. The study of macroevolution is the estimation of the processes underlying these observed patterns. The term origination is frequently used in place of speciation because it includes both speciation and migration because, depending on both the spatial scale and quality of the fossil record, it may be impossible to distinguish between the two.

Macroevolution, as metaphor, subsumes both speciation/extinction (phyletic) and anagenetic/lineage disparity (anagenetic) dynamics \citep{Foote2007b}. This usage has also been termed the tempo and mode of evolution \citep{Simpson1944}. This metaphoric usage more closely links paleobiology and macroevolution. This raises an important question: can we have paleobiology without macroevolution?

The contrast to macroevolution is microevolution \citep{Simpson1944,Foote2007b} which is strictly defined as change in allele frequency in a population from one generation to the next. It is important to note that changes in allele frequency affects phenotype frequency and expression and thus, by corollary, describes phenotypic change. Just as \citet{Simpson1944} argued, this link can explain rates of evolution and (most) patterns of disparity without invoking any ``macroevolutionary'' phenomenon.

What makes a definition strong or weak? Why would we prefer strong definitions to weak ones? What is the difference between definition, metaphor and metonymy?

Of major concern with the metaphoric use of macroevolution is that this concept subsumes all aspects of anagenetic change making the difference between microevolution and macrevolution unclear. Interestingly, the link between broad sense macroevolution and Simpson's tempo and mode of evolution is that Simpson's statement assigns no hierarchical level to these patterns. The pervasiveness of the use of broad sense macroevolution then is because this usage is explicitly because it is not the actual definition of macroevolution. Effectively, the metaphoric use removes any distinction between the terms and makes the terms useless.

An important theoretical construct in paleobiology is that of emergent traits or properties which are defined as traits that are not reducible to a lower level \citep{Grantham1995,Vrba1984,Jablonski2008a,Lloyd1993}. An emergent property is effectively a trait that is the product of multiple traits expressed at one or more hierarchical levels in concert and is not reducible \citep{Vrba1984,Jablonski2008a}. It is extremely important to recognize a fundamental hierarchy in biology in order to allow for emergent traits \citep{Vrba1984}. In paleobiology, an emergent property is normally one ascribed to the species or genus level as opposed to the organism level and is the root of ``species selection'' \citep{Jablonski2008a,Vrba1984,Lloyd1993,Grantham1995}. Range size is considered an emergent property that has continually been demonstrated to play a crucial role in extinction selectivity with species with larger geographic ranges having lower extinction rates than species with smaller geographic ranges \citep{Jablonski1986,Harnik2013,Nurnberg2013a,Jablonski2003,Roy2009c}. Range size is emergent because no one property of a single organism can explain this trait and instead it is a combination of multiple properties which determines global range size.

Survival can be considered the fundamental measure of fitness or evolution success \citep{Cooper1984,Palmer2012} because ultimately a long-term successful lineage is not one that speciated greatly but one that never went extinct \citep{Palmer2012}. Because during periods of background extinction patterns of taxonomic extinction are expected to be non-random with respect to biology \citep{Jablonski1986}. It should be possible to effectively measure the relative fitness of various ecological traits \citep{Kitchell1990,Kitchell1985a}. 

Survivorship analysis has a long history in paleobiology \citep{Simpson1944,VanValen1979,Foote1988,Baumiller1993,Kitchell1987b,Kitchell1990,Simpson1953,Raup1991a,Simpson2006}. The hazard/extinction rate from a survivorship curve is a statement of the rate at which organisms go extinction following origination. Hazard rates are comparable to the diversification rates estimated via phylogenetic comparative methods \citep{Fitzjohn2010,Maddison2007,Rabosky2013,Nee2001,Nee1994d,Nee1992}, however hazard rates are an estimate of extinction rate which is considered difficult if not impossible to estimate in this fashion \citep{Rabosky2010a}.

Here I propose to study the individual and combined effects of organismal traits related to emergent range size on extinction and, by extension, fitness.


\section{Cosmopolitan and endemic mammal dynamics of Cenozoic mammals}

\textit{Questions:} How do patterns of community connectedness change over time, specifically in Cenozoic terrestrial mammals? Do these patterns differ between continents? Are these patterns different between groups ecologically different organisms? Are these patterns related to changes in global temperature?

\textit{Background and Predictions:}
During the Cenozoic, there was a global shift from predominately closed, forested habitats to more open, savanna habitats. This pattern leads to the expectation that there would be a relative decrease in arboreal taxa as well as a relative increase in terrestrial taxa. Additionally, an increase in the relative endemism of arboreal taxa over time and an decrease in the relative endemism of terrestrial taxa might have accompanied this. The timing of this shift was different between continents \citep{Stromberg2005,Stromberg2013}, meaning that the patterns reflecting this environmental shift may be globally non-uniform and changes in community structures could reflect regional scale changes instead of global scale changes. Shifts in distribution of taxa according to locomotor category were not necessarily accompanied by shifts in distribution of dietary categories.

The majority of previous research has focused on mammalian faunal dynamics has focused on the North American fossil record \citep{Alroy2000g,Alroy1996a,Alroy1998,Barnosky2001a,Simpson1944,Simpson1953,Badgley2013,Blois2009,Figueirido2012,Gunnell1995,Hadly2001}. The major focuses have been on the effect of climate change on diversity and distributions between different higher taxonomic levels. The long term effects of climate change on North American mammalian diversity dynamics and community connectedness and similarly remains unresolved and controversial \citep{Alroy2000g,Blois2009,Figueirido2012,Barnosky2001a}. The basic predictions are that over the Cenozoic there would be a relative increase in endemism in arboreal taxa versus a relative decrease in ground dwelling endemism. Because of the vast amount of prior work on North American mammalian faunal dynamics, this forms the basis for the global predictions made above. The North American record then inadvertently becomes the baseline comparison for regional differences.

In comparison to North America, the European mammalian fossil is less studied. Importantly, research has focused primarily on faunal dynamics in the Neogene \citep{Jernvall2002,Jernvall2004,Liow2008,Raia2006,Raia2005,Raia2011c}. One of the major findings is that, during the Neogene, there was very little shift in relative trophic level abundance \citep{Jernvall2004} while the patterns in dietary shifts were mostly driven by abundance and cosmopolitan herbivores \citep{Jernvall2002}. Because of this, the major predictions for the European record is that occupancy will increase for herbivorous taxa, while increasing or remaining identical in carnivores, and remaining relatively constant for omnivores. 

The South American mammalian faunal record appears to reflect two distinct biotic provinces between the North and the South \citep{Macfadden1997,Macfadden2006,Flynn1998a,Patterson1968}. Because of this, I predict the South American record to have a very different pattern of biogeographic connectedness than either North America or Europe. Namely, the expectation would be a high or progressively increasing degree of endemism along with low connectivity. Also, an expected increase in land-dwelling herbiovres relative to arboreal (at least in the south).

A global trend during the Cenzoic was the shift from a ``hot house'' environment with no polar ice caps to an ``ice house'' environment with polar ice caps \citep{Zachos2008,Zachos2001}. This transition was known to have caused major shifts in the global climatic envelopes and the reorganization of communities along with it \citep{Janis1993a,Fortelius2002,Blois2009,Alroy2000g,Figueirido2012}. For global mammalian community connectedness and trophic structure there are two possible scenarios. First, it could be possible that while the environment might be shifting, lineages may adapt in place and overall trophic structure and biogeographic structure remaining rather constant through time \citep{Jernvall2004}. Alternatively, species may shift ranges and thus change the set of possible interacting taxa which would be associated with changes in trophic structure as well as biogeographic connectedness.

\textit{Proposed research:}
Using methods first proposed by \citet{Sidor2013} and \citet{Vilhena2013}, I propose to construct bipartite biogeographic networks between taxa and localities. A link between a taxon and a locality is formed when that is present at that locality. Here taxa are defined as species and localities are defined as formations. Biogeographic networks will be constructed for every 2 million year bin of the Cenozoic. This bin width is chosen to have minimum 2 formations to be present in the same bin. Additionally, networks will be constructed for each dietary category and each locomotor category. 

Network complexity and connectedness is measured using four previously used summary statistics \citep{Sidor2013}: average number of endemics, average occupancy, biogeographic connectedness, and code length. Biogeographic connectedness is effectively the relationship between the number of endemic taxa and the average occupancy. Code length is a measurement of the complexity and clustering of the graph \citep{Rosvall2008,Rosvall2010b}. A low code length indicates that a graph is compressible into a greater number of subunits without information loss than a graph with a high code length. This means that a low code length indicates lower overall locality similarity than a high code length.

In order to compare whether patterns observed on different continents are similar or different, as well as compare patterns between different categories of ecological traits, HOW DO I DO THIS?

Taxonomic occurrence data will be collected through a combination of the Paleobiology Database (PBDB; \url{http://fossilworks.org}), Neogene Old World Database (NOW; \url{http://www.helsink.fi/science/now/}), and museum collections. North American fossil mammal data is very well represented and vetted in the PBDB because of the extensive work by John Alroy \citep{Alroy1996a,Alroy1998,Alroy2000g}. European fossil mammal data is also well represented between the PBDB and NOW. South American fossil mammal data is available through the PBDB, but is not particularly well vetted and poorly covered. Because of this, South American fossil mammal data will be gathered via various museums such as the Field Museum of Natural History and the American Museum of Natural History. 


\section{Fitness and extinction risk related to ecology in Cenozoic mammals}

\textit{Questions:} How do ecological characters affect survivorship in mammals? Is any single trait the best predictor of mammalian survivorship, or do multiple traits together predict survivorship better?

\textit{Background and Predictions:} 
In mammals, three of the arguably most important ecological traits in determining range size are dietary category, locomotor category, and body size \citep{Jernvall2004,Smith2008b,Smith2004,Lyons2005,Lyons2010}. As discussed above, as the Cenozoic progressed on all Continents there was a shift from closed habitat to more open habitat. In the intermediate, one would expect some degree of patchiness in the landscape. Expectedly, there would be a increase followed by a decrease in both speciation rate and extinction rate, with the peak being during the intermediate period.

According to \citet{Price2012}, herbivores and carnivores should have a greater diversification rate than omnivores. This analysis was global in scope, and purely extant taxa in a comparative phylogenetic context. Diversification rate can increase via either an increase in origination relative to extinction or a decrease in extinction relative to origination. Which of these two processes is occurring is impossible to determine from a phylogeny of only extant organisms \citep{Rabosky2010a} which means that only via the analysis of the fossil record is it possible to estimate which process is more likely. 

Depending on the continent, body size has been demonstrated to be related to extinction rate \citep{Tomiya2013,Liow2008,Liow2009}. By expanding to include a third continent, South America, I hope to elucidate how differences in taxonomic diversity at a continental level might affect body size mediated extinction rate. Additionally, I will be using alternative methods to better understand the dynamics governing trait based extinction probability.

\textit{Proposed research:}
To investigate the effect of ecological traits and climate change on survivorship, I plan to compare different models of survival in order to best understand what are the most important factors in estimating survival probability.

Survivorship analysis is the analysis of time-till-event data. In a paleontological context this is the time from origination (first appearance date; FAD) till extinction (last appearance date; LAD). Dietary category, locomotor category, and body size will be modeled as time-independent covariates of survival. The climate proxy \(\delta O^{18}\) oxygen curve from \citet{Zachos2008} will be modeled as an ancillary time-dependent covariate. Also, constant versus accelerating, decelerating, or time variant extinction rate will be estimated using different fundamental hazard models by comparing the fit various probability distributions to survival.

%There are several commonly used probability distributions for modeling survivorship data CITATION. The Law of Constant extinction states that the shape of a survival curve is linear on a semi-log scale for any taxon in a particular adaptive zone, meaning that extinction rate is constant with respect to time \citep{VanValen1973}. This means that survival should be best modeled using the exponential distribution because has a single parameter, meaning that the logarithm of survival is constant with respect to time. Other probability distributions, such as the Weibull distribution, are used to better model accelerating, decelerating, or otherwise time varying extinction rate CITATION.

%ADD IN MODELS. NEED TO READ MORE SURVIVAL BOOK.

While many analyses of survivorship are done using generic data \citep{Tomiya2013,Liow2008,Harnik2013}, there are potential biases in accurately modeling specific level process from generic level data \citep{Raup1975,Sepkoski1975,Simpson2006,Raup1991a,VanValen1979}. There are important concerns regarding anagenesis, hierarchical selection, and extant taxa or taxa that did not go extinct in the time frame of interest \citep{Raup1975,VanValen1979,Simpson2006,Raup1991a}. Interestingly, the effect of incomplete sampling on estimation of survivorship curves appears rather minimal and uniform \citep{Sepkoski1975}. The problems involving extant taxa and taxa that did not go extinct have mostly been dealt with following advances of how to model right-censored data CITATION. 

In order to asses potential specific versus generic effects I will estimate differences in estimated survival between specific and generic level survivorship models. Using an approach based on previous work to estimate specific level survival from generic level survival curves \citep{Foote1988}, or a variant there of, I will measure the deviance between extinction rate estimated from the specific survivorship and the specific level extinction rates estimated from the analysis of the generic survivorship data. 

In addition to the above study of mammalian survivorship, I also propose a simulation study to analyze effect of varying sampling probability and/or anagenesis rate on estimating various models of survivorship using \texttt{paleotree} \citep{Bapst2012a}. Principally, I am interested in the effect of these paleontological realities on estimation of the hazard function of the survivorship data and in particular departures from a constant, or exponential, hazard function. Alternatives are, for example, models of accelerating or decelerating extinction rate. I intend to revise the previously analyzed effect of sampling on estimation survivorship in this new context \citep{Sepkoski1975}.

The data necessary to complete the empirical aspectes of this study will be the same as described above for analysis of dynamics of mammalian biogeographic connectedness.


\section{Extinction risk related to traits affecting habitat selection in Permian brachiopods}

\textit{Questions:} In Permian brachiopods in Australasia, do traits directly related habitat selection and range size relate to differential survivorship? Are certain traits more explanatory of survival than others? Does changing climate, and environmental and/or substrate availability affect survival?

\textit{Background and Predictions:}
In brachiopods, three extremely important ecological traits involved in determining possible range size are affixing strategy, substrate preference, and habitat preference. While larval biology is also considered extremely important for determining range size in marine invertebrates \citep{Jablonski2006a,Jablonski1983}. However, larval ecology does not preserve in brachiopods and thus cannot be used to model survivorship \citep{Jablonski1983}.

The three principle ways of classifying brachiopod affixing strategies are pedunculate, reclining, and cementing. While these classifications can be further subdivided \citep{Alexander1977}, this is most likely impractical or unnecessary for studies of differential survivorship \citep{Clapham2007}. During the Permian, pedunculate taxa tend to be associated with shallow on-shore environments while reclining taxa are associated with deep or off-shore environments \citep{Clapham2007}. However, this association is weak as most assemblages are composed of a heterogeneous mix of taxa \citep{Clapham2007}. Previous analysis of brachiopod taxonomic durations indicated affixing strategy is associated with differential longevity \citep{Alexander1977}. Among endemic taxa, reclining taxa have longer durations than all other affixing strategies. In contrast, among cosmopolitan taxa, pedunculate and cementing taxa had longer durations than all other taxa. 

The three principle categories of substrate affinity are carbonate, clastic, or mixed which describes the lithology of the sites at which the taxa are predominately found \citep{Foote2006} OTHER CITATIONS. The Pharenozoic is characterized by an overall decline in carbonate lithologies relative to clastic lithologies \citep{Foote2006,Miller2001}. Because of this, it is expected that taxa with clasitic or mixed affinities will have greater durations than taxa associated with carbonate substrates. % Brachiopods are mixed slash switchers CARL AND MELANIE'S UNPUBLISHED WORK.

The primary ways of classifying habitat preference are on-shore, off-shore, or mixed. Habitat preference has been the focus of a great deal of research in terms of explaining global diversity dynamics \citep{Sepkoski1991}. On-shore environments are represented in part by epicontinental seas which declined in areal extent over the Pharenozoic CITATIONS. Because of this decrease in areal extent, the expectation would be that taxa predominately associated with on-shore habitats would have overall lower durations (fitness) than taxa associated with off-shore habitats or mixed preference.

During the Permian there was a shift from an ``ice house'' world to a ``hot house'' world \citep{Fielding2006,Birgenheier2010,Jones2006,Powell2007} which could be expected to have some major effects on brachiopod survivorship. In particular, taxa in Australia would be of particular interest because of the proximity of Australia to the south pole during the Permian and the repeated glacial activity in the region \citep{Fielding2006,Birgenheier2010,Jones2006}. According to \citet{Olszewski2004}, sea-level and climate change do not wholly explain the ecological dynamics experienced by brachiopods in the Permian of Texas. The prediction then is that the best model of brachiopod survivorship will have to have some biotic component such as affixing strategy or substrate preference. If climate or environmental information, such as habitat preference, is a predictor in the best model of survivorship is less clear cut and necessary to determine empirically.


\textit{Proposed research:}
I propose to use a survival analysis approach similar to the one previously described to estimate the differential survivorship of brachiopods leading up to the Permo-Triassic boundary. I restric the analysis to Australiasia because it represents a relatively continually sampled and well worked area that preserves the majority of the entire Permian \citep{Clapham2012,Clapham2008a,Waterhouse1987,Archbold1995}. In this case, the time-independent covariates are substrate preference, affixing strategy, and habitat preference. Climate will be modeled as either an ancillary heavyside function or a time-varying covariate depending on the quality of the Permian isotope record. Additionally, as in the mammalian survivorship analysis described above, the time dependence of brachiopod extinction rates will be estimated using different fundamental hazard models by comparing the fit various probability distributions to survival.

Permian brachiopod occurrence information is available via the PBDB and is primarily based on the work of M. Clapham \citep{Clapham2006,Clapham2008a,Clapham2007a,Clapham2012,Clapham2007} and \citet{Waterhouse1987}.


\section{Summary of proposed research}
Fundamental questions in evolution and paleobiology. 

What do certain organisms go extinct while others do not?

How does community similarity change over time?

What role does ecology play in survival?

Is climate an important factor in survival?


\clearpage
\section{Timeline}

Spring/Summer 2014
\begin{itemize}
  \item Evolution Meeting: preliminary brachiopod survival results
  \item South American fossil mammal data from Field Museum of Natural History collections
\end{itemize}

Fall 2014/Winter 2015
\begin{itemize}
  \item GSA: survivorship simulation for anagenesis and sampling
\end{itemize}

Spring/Summer 2015
\begin{itemize}
  \item Evolution Meeting: mammalian survivorship analysis or biogeographic connectedness for North America and Europe
  \item South American fossil mammal data from American Museum of Natural History collections
\end{itemize}

Fall 2015/Winter 2016
\begin{itemize}
  \item SVP or GSA: mammalian survivorship analysis or biogeographic connectedness for North America and Europe
\end{itemize}

Spring/Summer 2016
\begin{itemize}
  \item Evolution Meeting: brachiopod survival analysis results
\end{itemize}

Fall 2016/Winter 2017
\begin{itemize}
  \item SVP or GSA: mammalian survivorship analysis or biogeographic connectedness for North America, Europe and South America
\end{itemize}

Spring/Summer 2017
\begin{itemize}
  \item Evolution Meeting
  \item \textbf{Defend}
\end{itemize}



\clearpage
\bibliographystyle{abbrvnat}
\bibliography{proposal}

\end{document}
