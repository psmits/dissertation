\documentclass[12pt,letterpaper]{article}

\usepackage{amsmath, amsthm}
\usepackage{graphicx}
\usepackage{microtype, parskip}
\usepackage{caption, subcaption, multirow, morefloats, rotating, longtable}
\usepackage{hyperref}
\usepackage[numbers,sort&compress]{natbib}
\usepackage{authblk, attrib, fullpage}
\usepackage{lineno}


\begin{document}
\setcounter{secnumdepth}{0}

\begin{titlepage}
  \begin{center}
    \huge{Evolutionary paleoecology and the biology of extinction}

    \vspace{1.5cm}

    \large{Peter D. Smits \\}
    \footnotesize{\href{mailto:psmits@uchicago.edu}{psmits@uchicago.edu}}

    \vspace{1.5cm}

    Dissertation Proposal Hearing \\
    \today \\
    Commmittee on Evolutionary Biology \\
    The University of Chicago

    \vspace{1.5cm}

    \textit{Committee} \\
    Dr. Michael J. Foote (co-advisor) \\
    Dr. Kenneth D. Angielczyk (co-advisor) \\
    Dr. Richard H. Ree \\
    Dr. P. David Polly
  \end{center}
\end{titlepage}

\linenumbers
\modulolinenumbers[2]


\section{Introduction and Theory}

\subsection{Evolutionary paleoecology}
Paleobiology is the study of life over time and the processes that generate the observed patterns in diversity and disparity. Intimately related to paleobiology is the concept of macroevolution here defined as the pattern of speciation and extinction over time \citep{Jablonski2008a}. The study of macroevolution is the estimation of the processes underlying these observed patterns. The term origination is frequently used in place of speciation because it includes both speciation and migration and depending on both the spatial scale and quality of the fossil record it may be impossible to distinguish between the two.

Evolutionary paleoecology is defined as the study of the effects of ecological traits and factors on differential rate dynamics, particularly rates of faunal turnover and diversification \citep{Kitchell1985a}. Ecological traits and factors are traits expressed by a taxon, at any level, that are involved with biotic--biotic or biotic--abiotic interactions. Diversification is the difference between origination and extinction and is the net pattern of macroevolution. The study of evolutionary paleoecology is therefore the link between environmental interactions and macroevolution. As a corollary to \citet{Kitchell1985a}'s definition, \citet{Allmon1994} states that in order to correctly link ecological interactions to macroevolution, one must focus on the specific traits and factors that may affect the speciation process. Tacitly included in this is the study of extinction and how it relates to ecological traits \citep{Kitchell1990}.

%\citet{Simpson1944} defined environment broadly as the set of all possible biotic and abiotic interactions as well as the organism itself. Related to this, \citet{Simpson1944} defined the ``adaptive zone'' as the set of all biotic and abiotic interactions that a lineage experiences. The adaptive zone represents one of the fundamental metaphors in paleobiology and can be considered the adaptive landscape through time \citep{Simpson1944,Simpson1953}. This usage of environment is frequently misinterpreted or unknown and has lead to a great deal of confusion of the validity and meaning of, for example, the Red Queen hypothesis and thus is important to note here \citep{VanValen1973,Barnosky2001a,Liow2011a}. 

It is under this framework that I propose to study how ecological traits associated with range size have affected both the availability of biotic interactions and differential survival. I will be studying two distantly related and biotically different groups: Cenozoic mammals and Permian brachiopods. Both of these groups are considered to have very good fossil records able to reflect massive long term evolutionary patterns \citep{Mark1977}. These two time periods were chosen because they represent periods of climatic change, global cooling and global warming respectively. Also, these two groups are a terrestrial and marine system respectively and the ecological traits associated with range size (described below) are fundamentally very different. 


%\subsection{Semantics of paleobiology}
%When used as a metaphor, macroevolution subsumes both speciation/extinction (phyletic) and anagenetic/lineage disparity (anagenetic) dynamics \citep{Foote2007b}. \citet{Simpson1944} termed this usage the tempo and mode of evolution \citep{Simpson1944}. This metaphoric usage of macroevolution more closely links it with paleobiology. This raises an important question: can we have paleobiology without ``macroevolution?''

%The historical contrast to macroevolution is microevolution \citep{Simpson1944,Foote2007b} which is here defined as change in allele frequency in a population from one generation to the next. It is important to note that changes in allele frequency affects phenotypic expression and thus, by corollary, describes phenotypic change. Just as \citet{Simpson1944} argued, this link can explain rates of evolution and patterns of disparity without invoking any ``macroevolutionary'' phenomenon. Macroevolution is most frequently used in this metaphoric context and used to describe an important phenomenon or something different from ``normal'' biotic processes. The metaphoric usage of macroevolution, however, is stuck as a pattern based concept and is divorced from the important processes based definition(s) of (paleo)biology. 

%An important theoretical construct in paleobiology is emergent traits or properties which are defined as traits that are not reducible to a lower level \citep{Grantham1995,Vrba1984,Jablonski2008a,Lloyd1993}. An emergent property is effectively a trait that is the product of multiple traits expressed at one or more lower hierarchical levels in concert and is not reducible to any one single constituent trait \citep{Vrba1984,Jablonski2008a}. It is extremely important to recognize a fundamental hierarchy in biology in order to allow for emergent traits \citep{Vrba1984}. In paleobiology, an emergent property is normally ascribed to the species or genus level as opposed to the organism level and is the root of ``species selection'' \citep{Jablonski2008a,Vrba1984,Lloyd1993,Grantham1995}. Importantly, phenomenological ``macroevolution'' does not necessitate species selection, such as in the effect hypothesis \citep{Vrba1983,Jablonski2008a,Kitchell1985a}, though it is frequently invoked as an integral part of what makes macroevolution different from microevolution \citep{Grantham1995,Jablonski2008}.

%However, because by definition emergent properties can exist at any hierarchical level, ``species selection'' is no different from selection at any level. While microevolution and macroevolution were originally coined to distinguish different ``types'' of evolution \citep{Goldschmidt1940}, because species selection is not necessarily different selection at any other level then there is nothing ``special'' about macroevolution and thus this distinction can be considered greatly flawed \citep{Simpson1953}. Instead, accepting the fundamental hierarchy of biology \citep{Vrba1984} we can say that, following \citet{Simpson1953}, it is not that there is different ``types'' of evolution but instead different ``levels'' of evolution which reflect the degree of complexity of processes that can be acting at once. This is just (paleo)biology and does not require invoking any kind of special ``macroevolutionary'' phenomena. The flawed distinction between macroevolution and microevolution only comes into play when the here described metaphoric macroevolution is used as the defintion. By making a clear and distinct defintion of macroevolution, the metaphor can be more freely used because it is precisely not the definition of the term and allows for the accustomed freedom of usage without creating a distinction in ``types'' of evolution. This opening of semantic space is thus an act unification instead of one of division. 


\subsection{Theoretical framework for proposed study}
For the majority of geological time, extinction can be considered non-random with respect to biology \citep{Jablonski1986}. These times of ``background extinction'' are in comparison to the alternative periods of ``mass extinction.'' The distinction between background and mass extinction is based on the relative intensity of extinction at a particular time versus times preceding and following. Simply put, the major indicator of a mass extinction is an event that is across higher taxonomic levels and the biotic interactors before and after the event are dramatically different \citep{Jablonski1986,Jablonski2005,Kitchell1986,Kitchell1991}.

Survival can be considered the fundamental measure of fitness or evolution success because ultimately a successful lineage is not one that speciated greatly but one that never went extinct \citep{Cooper1984,Palmer2012}. Because during periods of background extinction extinction is non-random with respect to biology \citep{Jablonski1986}, it should be possible to estimate the relative fitness effects of various ecological traits \citep{Kitchell1990,Kitchell1985a}. 

%Survivorship analysis has a long history in paleobiology \citep{Simpson1944,VanValen1979,Foote1988,Baumiller1993,Kitchell1987b,Kitchell1990,Simpson1953,Raup1991a,Simpson2006}. The hazard/extinction rate from a survivorship curve is a statement of the rate at which organisms go extinction following origination. Hazard rates are comparable to the diversification rates estimated via phylogenetic comparative methods \citep{Fitzjohn2010,Maddison2007,Rabosky2013,Nee2001,Nee1994d,Nee1992}, however they are an estimate of extinction rate which is considered difficult if not impossible to estimate in this fashion \citep{Rabosky2010a}.

%Importantly, extinction is function not necessarily a constant rate. This is a very important distinction that needs to be tested. A single rate implies an extinction is an exponential function that is time invariant. This is central to the Law of Constant extinction. Multiple extinction functions, namely functions that are time varying, should be compared. For example, Wiebull distribution has rate and shape parameters. What then is extinction ``rate'' when it is best represented as a function?

Species geographic range size has continually been demonstrated to play a crucial role in extinction selectivity. Species with larger geographic ranges tend to have lower extinction rates than species with smaller geographic ranges \citep{Jablonski1986,Harnik2013,Nurnberg2013a,Jablonski2003,Roy2009c}. Range size is considered emergent because no one property of an organism can explain this trait and instead it is a combination of multiple properties which determines range size. Here I propose to study the individual and combined effects of organismal traits related to range size on extinction and, by extension, fitness. Additionally, the ecological adaptations to variable environments has been shown to be important in predicting species longevity \citep{Foote2013,Liow2007b}.


\section{Dynamics of community connectedness in Cenozoic mammals}

\textit{Questions:} 
How does the average set of possible biotic--biotic interactions in mammalian communities change over time? Is there a single global pattern, or do different continents have different patterns? Do patterns among ecological categories differ? Is global climate change an important predictor of these patterns?

\textit{Background and Predictions:}
In mammals, three of the most important ecological traits in determining range size are dietary category, locomotor category, and body size \citep{Jernvall2004,Smith2008b,Smith2004,Lyons2005,Lyons2010}. Different dietary categories acts as a limit on the possible abundance in a location because of the available environmental energy. It has been found that abundance is correlated with occupancy, or the number of unique localities at which a taxon is found \citep{Jernvall2002,Fortelius2002,Brown1984}. It follows then that limits imposed trophic level by avaliable environmental energy would then effect the (possible) range size of a taxon. Locomotor category describes the motility of a taxon and the plausibility of environmental occurrence. Locomotor category also limits the dispersal ability of a taxon. For example, an obligate arboreal taxon can only occur in locations with a minimum of tree cover and can most likely only disperse to other nearby environments with suitable tree cover. Dispersal ability is considered important in determining the extent of a taxon's geographic range \citep{Birand2012,Jablonski2006a,Gaston2009} and thus any trait that would limit the ability for an organism to disperse would most likely limit the range size of that organism.

During the Cenozoic there was a global shift from predominately closed, forested habitat to more open, savanna-like habitat. It is then expected that there would have been a relative increase in endemism of arboreal taxa over time and a decrease in relative endemism of terrestrial taxa. The timing of environmental shift was different between continents \citep{Stromberg2005,Stromberg2013}, so the patterns of community connectedness may not be globally uniform and could reflect regional differences. Shifts in distribution of taxa by locomotor category would not necessarily accompanied by shifts in distribution related to dietary category.

A global trend during the Cenzoic was the shift from a ``hot house'' environment with no polar ice caps to an ``ice house'' environment with polar ice caps \citep{Zachos2008,Zachos2001}. This transition was known to have caused major shifts in the global climatic envelopes and the reorganization of communities along with it \citep{Janis1993a,Fortelius2002,Blois2009,Alroy2000g,Figueirido2012}. For mammalian community connectedness there are two possible scenarios. First, it could be possible that while the environment might be shifting, lineages may adapt in place and overall trophic structure and biogeographic structure remaining rather constant through time \citep{Jernvall2004}. Alternatively, species may shift ranges and thus change the set of possible interacting taxa which would be associated with changes in trophic structure as well as biogeographic connectedness.

The majority of previous research has focused on mammalian faunal dynamics has focused on the North American fossil record and the effects of climate change on the diversity and distributions \citep{Alroy2000g,Alroy1996a,Alroy1998,Barnosky2001a,Simpson1944,Simpson1953,Badgley2013,Blois2009,Figueirido2012,Gunnell1995,Hadly2001}. The long term effects of climate change on North American mammalian diversity dynamics and community connectedness and similarly remains unresolved and controversial \citep{Alroy2000g,Blois2009,Figueirido2012,Barnosky2001a}. The basic predictions are that over the Cenozoic there would be a relative increase in endemism in arboreal taxa versus a relative decrease in ground dwelling endemism. Because of the vast amount of prior work on North American mammalian faunal dynamics, this forms the basis for the global predictions made above and the North American record inadvertently becomes the baseline for comparison with other regions.

The European mammalian fossil is less studied compared to North America and research has focused primarily on faunal dynamics in the Neogene \citep{Jernvall2002,Jernvall2004,Liow2008,Raia2006,Raia2005,Raia2011c}. One of the major findings is that there was very little shift in relative trophic level abundance \citep{Jernvall2004} while the patterns in dietary shifts were mostly driven by abundant, cosmopolitan herbivores \citep{Jernvall2002}. Because of this, the major predictions for the European record is that occupancy will increase for herbivorous taxa, while increasing or remaining constant in carnivores, and remaining relatively constant for omnivores. 

The South American mammalian faunal record reflects two distinct biotic provinces between the North and the South \citep{Macfadden1997,Macfadden2006,Flynn1998a,Patterson1968}. Because of this, I predict the South American record to have a very different pattern of biogeographic connectedness than either North America or Europe. Also, there is an expected increase in land-dwelling herbiovres relative to arboreal (at least in the south).


\textit{Proposed research:}
Using methods first proposed by \citet{Sidor2013} and \citet{Vilhena2013}, I propose to construct bipartite biogeographic networks between taxa and localities. Here taxa are defined as species and localities are defined as formations. Networks will be made for every 2 million year bin of the Cenozoic. This bin width is chosen to have minimum 2 localities be present in every bin. Additionally, networks will be constructed for each dietary category and each locomotor category. Previous studies of mammalian occurrence patters have restricted analysis to major orders, such as Primates and Artiodactyls, in order to account for apparent sampling and taxonomic biases. Here, analysis will be done using all available taxa and with a restricted sample of just major orders in order to observe any differences in community connectedness.

Network complexity and connectedness is measured using four previously used summary statistics \citep{Sidor2013}: average relative number of endemics, average relative occupancy, biogeographic connectedness, and code length. The average number of endemics is defined as 
%\begin{equation}
\(
  E = \frac{\sum_{i = 1}^{L} \frac{u_{i}}{n_{i}}}{L}
\)
%  \label{eq:end}
%\end{equation}
where \(L\) is as the number of localities, \(u\) is the number of taxa unique to a locality, and \(n\) is the number of taxa present at a locality. This is a measure of how unique localities are. Average occupancy is the number of localities a taxon is, on average, found at. It is defined as 
%\begin{equation}
\(
  Occ = \frac{\sum_{i = 1}^{N} \frac{l_{i}}{L}}{N}
\)
%  \label{eq:occ}
%\end{equation}
where \(N\) is as the number of taxa present in the biogeographic network and \(l\) is the number of localities a taxon occurred in. Biogeographic connectedness is effectively the relationship between the number of endemic taxa and occupancy and is a measurement of the shared taxa between localities. It is defined as 
%\begin{equation}
\(
  BC = \frac{O - N}{LN - N}
\)
%  \label{eq:bc}
%\end{equation}
where \(O\) is the total number of taxonomic occurrences. BC ranges from 0 to 1, with 0 meaning that each locality completely disconnected from all other localities and 1 indicating all that taxa shared between all localities. Importantly, \(BC\) is infinite when there is only one locality.

Code length is a measurement of the complexity and clustering of the graph \citep{Rosvall2008,Rosvall2010b}. This value is calculated as the length of the Huffman code CONTINUE. A low code length indicates that a graph is compressible into a greater number of subunits without information loss than a graph with a high code length. This means that a low code length indicates lower overall locality connectedness than a high code length. 

In order to include phylogenetic similarity between localities, I will rely on taxonomy as a informal proxy of phylogeny. There are two possible approaches for measuring the similiarty in taxonomic composition between two localities: a variant of the Robinson-Foulds metric \citep{Robinson1981} can be used that accounts for unshared taxa or the two localities are combined to form one tree and the average pairwise patristic distance \citep{Sokal1962} is calculated. The average of either of these two measures between all localities then forms another times series of change in average phylogenetic similarity which can then be used in partial correlation tests or modeling questions.

In order to compare whether patterns observed on different continents are similar or different, as well as compare patterns between different categories of ecological traits, HOW DO I DO THIS?

Taxonomic occurrence data will be collected through a combination of the Paleobiology Database (PBDB; \url{http://fossilworks.org}), Neogene Old World Database (NOW; \url{http://www.helsink.fi/science/now/}), and museum collections. North American fossil mammal data is very well represented and vetted in the PBDB because of the extensive work by John Alroy \citep{Alroy1996a,Alroy1998,Alroy2000g}. European fossil mammal data is also well represented between the PBDB and NOW. South American fossil mammal data is available through the PBDB, but is not particularly well vetted and has poor overall coverage. Because of this, South American fossil mammal data will be gathered via various museums such as the Field Museum of Natural History and the American Museum of Natural History. With the South American taxa, taxonomy and sampling may not be as well resolved as for North and South America and it may be necessary to restrict analysis to the most taxonomically resolved and sampled groups such as Notoungulates, Marsupials, Carnivores, and Primates.


\section{Ecology, survivorship, and fitness in Cenozoic mammals}

\textit{Questions:} 
How do ecological traits related to range size affect time from origination to extinction in mammals? Is any single trait the best predictor of mammalian survivorship, or do multiple constituent traits best model time from origination to extinction? Does climate change matter when modeling time from origination to extinction?

\textit{Background and Predictions:} 
As discussed above, dietary and locomotor categories are strong candidate constituent traits of range size. Additionally, body size is a classically cited constituent trait of range size. An organism of a certain body size has associated energetic cost in order to maintain homeostasis, which in turn necessitates the availability of the appropriate prey items. Because of this, it is expected that larger organisms have higher energetic costs and thus a greater range size in order to obtain necessary resources. 

As discussed above, as the Cenozoic progressed on all continents there was a shift from closed habitat to open habitat. In the intermediate, one would expect some degree of patchiness. It follows that there would be an increase followed by a decrease in both speciation and extinction, with the peak being during the intermediate period.

Herbivores and carnivores have been found to a greater diversification rate than omnivores \citep{Price2012}. This analysis was global in scope, and based on purely extant taxa in a comparative phylogenetic context. Diversification rate can increase via either an increase in origination relative to extinction or a decrease in extinction relative to origination. Which of these two processes is occurring is impossible to determine from a phylogeny of only extant organisms \citep{Rabosky2010a} which means that analysis of the fossil record is necessary to estimate which scenario is most likely to have occurred. 

Depending on the continent, body size has been demonstrated to be related to extinction rate \citep{Tomiya2013,Liow2008,Liow2009}. By expanding to include a third continent, South America, I hope to elucidate how differences in taxonomic diversity at a continental level might affect body size mediated extinction rate. Additionally, I will be using alternative methods to better understand the dynamics governing trait based extinction probability.

\textit{Proposed research:}
To investigate the effect of ecological traits and climate change on survivorship, I plan to compare different models of trait based survival in order to best understand which facgtors are the most important.

Survivorship analysis is the analysis of time-till-event data. In a paleontological context this is the time from origination (first appearance date; FAD) till extinction (last appearance date; LAD). Dietary category, locomotor category, and body size will be modeled as time-independent covariates of survival. The climate proxy \(\delta O^{18}\) oxygen curve \citep{Zachos2008} will be modeled as an ancillary time-dependent covariate. Also, constant versus accelerating, decelerating, or time variant extinction rate will be estimated using different fundamental hazard models by comparing the fit various probability distributions to survival.

%There are several commonly used probability distributions for modeling survivorship data CITATION. The Law of Constant extinction states that the shape of a survival curve is linear on a semi-log scale for any taxon in a particular adaptive zone, meaning that extinction rate is constant with respect to time \citep{VanValen1973}. This means that survival should be best modeled using the exponential distribution because has a single parameter, meaning that the logarithm of survival is constant with respect to time. Other probability distributions, such as the Weibull distribution, are used to better model accelerating, decelerating, or otherwise time varying extinction rate CITATION.

%ADD IN MODELS. NEED TO READ MORE SURVIVAL BOOK.

While many analyses of survivorship are done using generic data \citep{Tomiya2013,Liow2008,Harnik2013}, there are potential biases in accurately modeling specific level process from generic level data \citep{Raup1975,Sepkoski1975,Simpson2006,Raup1991a,VanValen1979}. There are important concerns regarding anagenesis, hierarchical selection, and extant taxa or taxa that did not go extinct in the time frame of interest \citep{Raup1975,VanValen1979,Simpson2006,Raup1991a}. Interestingly, the effect of incomplete sampling on estimation of survivorship curves appears rather minimal and uniform \citep{Sepkoski1975}. The problems involving extant taxa and taxa that did not go extinct have mostly been dealt with following advances of how to model right-censored data \citep{Kleinbaum2005}.

In order to asses potential specific versus generic effects I will estimate differences in estimated survival between specific and generic level survivorship models. Using an approach based on previous work to estimate specific level survival from generic level survival curves \citep{Foote1988}, or a variant there of, I will measure the deviance between extinction rate estimated from the specific survivorship and the specific level extinction rates estimated from the analysis of the generic survivorship data. 

In addition to the above study of mammalian survivorship, I also propose a simulation study to analyze effect of varying sampling probability and/or anagenesis rate on estimating various models of survivorship using \texttt{paleotree} \citep{Bapst2012a}. Principally, I am interested in the effect of these paleontological realities on estimation of the survival function and in particular departures from a constant, or exponential, hazard function. 

The data necessary to complete the empirical aspects of this study will be the same as described above for analysis of dynamics of mammalian biogeographic connectedness.


\section{Permian brachiopods, extinction and environmental preference}

\textit{Questions:} In Permian brachiopods in Australasia, do traits directly related to environmental selection relate to differential survivorship? Are certain traits more explanatory of survival than others? Does changing climate, and habitat and/or substrate availability affect survival?

\textit{Background and Predictions:}
In brachiopods, three extremely important ecological traits involved in determining possible range size are affixing strategy, substrate preference, and habitat preference. While larval biology is also considered extremely important for determining range size in marine invertebrates \citep{Jablonski2006a,Jablonski1983} larval state does not preserve in brachiopods and thus cannot be used to model survivorship \citep{Jablonski1983}. Substrate preference is statement of the chemical and physical processes affecting the environment and acts as a limiting factor on the range of possible environments in which an organism can optimally survive. This then limits the total possible geographic range of a taxon. Substrate selection is mitigated via larval chemosensory abilities and is thus also a weak proxy for larval dispersal ability \citep{Jablonski2006a,Jablonski1983}. 

Affixing strategy and habitat preference relate to range size also by means of limiting the possible total geographic range of a taxon. Affixing strategy is the manner by with an individual directly interfaces with the substrate. Different strategies are optimal for certain environmental conditions such as flow speed or mud depth \citep{Alexander1977,LaBarbera1978,LaBarbera1981}. Because all brachiopods are obligate filter feeders, flow speed and environmental energetics is important in prey capture and survival. Thus, the availability of optimal environments becomes a limiting factor on the possible total geographic range of a taxon. Habitat preference is a statement of the suitability of an environment and the accompanying environmental energy level that an organism is most suited for. The availability of this habitat acts as a limiting factor on the taxon total possible geographic range of a taxon. 

The three principle ways of classifying affixing strategies are pedunculate, reclining, and cementing. During the Permian, pedunculate taxa tend to be associated with shallow on-shore environments while reclining taxa are associated with deep or off-shore environments \citep{Clapham2007}. However, this association is weak as most assemblages are composed of a heterogeneous mix of taxa \citep{Clapham2007}. Previous analysis of brachiopod durations indicated affixing strategy is associated with differential longevity \citep{Alexander1977}. Among endemic taxa, reclining taxa have longer durations than all other affixing strategies. In contrast, among cosmopolitan taxa, pedunculate and cementing taxa had longer durations than all other taxa. 

The three principle categories of substrate affinity are carbonate, clastic, or mixed and are descriptions of the lithology of the sites at which the taxa are predominately found \citep{Foote2006,Anderson2011a,Nurnberg2013a,Kiessling2007a,Miller2001}. The Pharenozoic is characterized by an overall decline in carbonates relative to clastics \citep{Foote2006,Miller2001}. Because of this, it is expected that taxa with clasitic or mixed affinities will have greater durations than taxa associated with carbonate substrates. % Brachiopods are mixed slash switchers CARL AND MELANIE'S UNPUBLISHED WORK.

The primary ways of classifying habitat preference are on-shore, off-shore, or mixed. Habitat preference has been the focus of a great deal of research in terms of explaining global diversity dynamics \citep{Sepkoski1991}. On-shore environments, and in particular epicontinental seas, have declined in areal extent over the Pharenozoic \citep{Peters2008}. Because of this decrease in areal extent, the expectation would be that taxa predominately associated with on-shore habitats would have overall lower durations than taxa associated with off-shore habitats or mixed preference.

An important consideration is that taxonomic survival might not be linked to environment \textit{per se}, but the variability of environment \citep{Foote2013,Heim2011,Liow2007b}. This adaptation to variable environments has been found to relate strongly with survival past inception \citep{Foote2013}. In this case, it would be expected then that taxa with mixed preferences for both substrate and habitat would have potentially longer taxa with single preferences in substrate and/or habitat. This makes logical sense as it would mean that a taxon's potential geographic range is not expressly limited these two important ecological traits. 

During the Permian there was a shift from an ``ice house'' to a ``hot house'' world \citep{Fielding2006,Birgenheier2010,Jones2006,Powell2007} which could be expected to have some major effects on brachiopod survivorship. In particular, taxa in Australia would be of particular interest because of the proximity of Australia to the south pole during the Permian and the repeated glacial activity in the region \citep{Fielding2006,Birgenheier2010,Jones2006}. According to \citet{Olszewski2004}, sea-level and climate change do not wholly explain the ecological dynamics experienced by brachiopods in the Permian of Texas. The prediction then is that the best model of brachiopod survivorship will have to have some biotic component such as affixing strategy or substrate preference. If climate is a predictor in the best model of survivorship is less clear cut and necessary to determine empirically.


\textit{Proposed research:}
I propose to use an approach similar to that previously described to estimate the differential survivorship of brachiopods leading up to the Permo-Triassic boundary. I restrict the analysis to Australiasia because it represents a relatively continually sampled and well worked area that preserves the majority of the entire Permian \citep{Clapham2012,Clapham2008a,Waterhouse1987,Archbold1995}. In this analysis, the time-independent covariates are substrate preference, affixing strategy, and habitat preference. Climate will be modeled as either an ancillary heavyside function or a time-dependent covariate depending on the quality of the Permian isotope record. Additionally, as in the mammalian survivorship analysis described above, the age dependence of brachiopod extinction rates will be estimated using different fundamental hazard models by comparing the fit various probability distributions to survival.

Permian brachiopod occurrence information is available via the PBDB and is primarily based on the work of Clapham \citep{Clapham2006,Clapham2008a,Clapham2007a,Clapham2012,Clapham2007} and \citet{Waterhouse1987}.


\section{Summary of proposed research}
% Fundamental questions in evolution and paleobiology. 
% What do certain organisms go extinct while others do not?
% How does community similarity change over time?
% What role does ecology play in survival?
% Is climate an important factor in survival?
% evolve, move, or die
One of the most important questions in (paleo)biology is why do certain species go extinct while others do not? %Related to this is what makes some species have long durations and some have short? 
Elucidating what interactions, or traits governing interactions, are important when estimating survival is then extremely important and a fundamental concern of evolutionary paleoecology. While the species level property of range size is continually found to be an extremely vital for both origination and extinction \citep{Roy2009c,Foote2013,Jablonski2003,Jablonski1987,Harnik2013}, which of the candidate constituent lower level traits are necessary to ``form'' range size remains more nebulous and is frequently framed as which traits in addition to range size \citep{Foote2013,Harnik2011,Nurnberg2013a}. Relating to this is the general question of how environmental and climatic change impact diversity dynamics \citep{Barnosky2001a,Alroy2000g,Figueirido2012,Olszewski2004}.
Here I compare the impact of various probable constituent traits of emergent range size on both community connectedness and survival. The former is a discussion of the range and limitations of possible biotic--biotic interactors and the later is a discussion of what ecological traits, either alone or in concert, best model time from origination till extinction. By comparing two distantly related clades, mammals and brachiopods, the hope is to determine how necessary discussion of ``emergent'' properties is and how lower level traits may interact to increase the survival, success, and fitness of a taxon.


\clearpage
\section{Timeline}

Spring/Summer 2014
\begin{itemize}
  \item Evolution Meeting: preliminary brachiopod survival results
  \item South American fossil mammal data from Field Museum of Natural History collections
\end{itemize}

Fall 2014/Winter 2015
\begin{itemize}
  \item GSA: survivorship simulation for anagenesis and sampling
  \item Doctoral Dissertation Improvement Grant
\end{itemize}

Spring/Summer 2015
\begin{itemize}
  \item Evolution Meeting: mammalian survivorship analysis for North America and Europe
  \item South American fossil mammal data from American Museum of Natural History collections
  \item write and submit survivorship simulation paper (target: \textit{Paleobiology})
\end{itemize}

Fall 2015/Winter 2016
\begin{itemize}
  \item SVP or GSA: mammalian biogeographic connectedness
  \item write and submit mammal connectedness paper (target: \textit{American Naturalist})
\end{itemize}

Spring/Summer 2016
\begin{itemize}
  \item Evolution Meeting: brachiopod survival analysis results
  \item write and submit brachiopod survival paper (target: unknown)
\end{itemize}

Fall 2016/Winter 2017
\begin{itemize}
  \item SVP or GSA: mammalian survivorship analysis
  \item write and submit mammal survival paper (target: unknown)
\end{itemize}

Spring/Summer 2017
\begin{itemize}
  \item Evolution Meeting
  \item write and submit review/philosophy paper (target: unknown)
  \item \textbf{Defend}
\end{itemize}



\clearpage
\bibliographystyle{abbrvnat}
\bibliography{proposal}

\end{document}
