\documentclass[12pt,letterpaper]{article}

\usepackage{amsmath, amsthm}
\usepackage{graphicx}
\usepackage{microtype, parskip}
\usepackage{caption, subcaption, multirow, morefloats, rotating, longtable}
\usepackage{hyperref}
\usepackage[numbers,sort&compress]{natbib}
\usepackage{authblk, attrib, fullpage}
\usepackage{lineno}


\begin{document}
\setcounter{secnumdepth}{0}

\begin{titlepage}
  \begin{center}
    \huge{Evolutionary paleoecology and the biology of extinction}

    \vspace{1.5cm}

    \large{Peter D. Smits \\}
    \footnotesize{\href{mailto:psmits@uchicago.edu}{psmits@uchicago.edu}}

    \vspace{1.5cm}

    Dissertation Proposal Hearing \\
    \today \\
    Commmittee on Evolutionary Biology \\
    The University of Chicago

    \vspace{1.5cm}

    \textit{Committee} \\
    Dr. Michael J. Foote (co-advisor) \\
    Dr. Kenneth D. Angielczky (co-advisor) \\
    Dr. Richard H. Ree \\
    Dr. P. David Polly
  \end{center}
\end{titlepage}

\linenumbers
\modulolinenumbers[2]


\section{Introduction and theoretical framework}

\subsection{Evolutionary paleoecology}
Evolutionary paleoecology is defined as the study of the effects of ecological traits and factors on differential rate dynamics, particularly rates of faunal turnover and diversification \citep{Kitchell1985a}.
Ecological traits and factors are any and all traits expressed by a taxon, at any level, that are involved with biotic--biotic or biotic--abiotic interactions. These interactions are between the taxon and a factor, which as stated may be either biotic or abiotic.
Diversification is the difference between origination and extinction, and is thus the net product of pattern of macroevolution.
The study of evolutionary paleoecology is then the link between interactions and macroevolution. Namely, it is the study of the ecological processes that may or may not generate the patterns of macroevolution.
\citet{Allmon1994} amends Kitchell's definition by stating that in order to correctly link ecological processes to macroevolution, one must focus on the specific traits and factors that affect the speciation process. Tacitly included in this is the study of the biology of extinction \citep{Kitchell1990}.


\subsection{Paleobiological theory}
% definition
% metaphor
% metonymy

Paleobiology is the study of life over time and in particular the processes that generate the observed patterns in diversity and disparity and how these may have changed.  % look up in the foote book
Intimately related to this is the concept of macroevolution. Macroevolution, \textit{sensu stricto}, is the pattern of speciation and extinction dynamics over time \citep{Jablonski2008a}. The study of macroevolution, thus, is the method by which the processes underlying these patterns are delineated. The term origination is frequently used in place of speciation because it is considered impossible to observe speciation in the fossil record and instead we only observe the sudden appearance of a new taxon \citep{Coyne2004}.

Macroevolution, \textit{sensu lato}, is both phyletic and anagenetic evolutionary dynamics \citep{Foote2007b}. Phyletic means speciation/extinction dynamics and anagenetic means within lineage disparity dynamics. This concept has also been termed the tempo and mode of evolution \citep{Simpson1944}. This broader definition more closely links paleobiology and macroevolution.

In contrast to macroevolution is microevolution \citep{Simpson1944,Foote2007b} which is defined strictly as change in allele frequency in a population from one generation to the next. A weaker definition is that microevolution is change below the species level \citep{Foote2007b} though there is no qualifier on what this change is defined as. It is important to note that changes in allele frequency affect phenotype frequency and expression.

% why this definition isn't as useful
%   conflation of two different patterns
%   makes the word too broad
%   this is macroevolution as metaphor

Of concern with the broader definition of macroevolution is that this concept subsumes all aspects of anagenetic change. The difference between microevolution versus macrevolution is unclear. Interestingly, the link between macroevolution \textit{sensu lato} and Simpson's tempo and mode of evolution is that Simpson's statement assigns no hierarchical level to these patterns. The pervasiveness of the use of macroevolution \textit{sensu lato} then is because this usage is metaphoric and explicitly because it is not the actual definition of macroevolution.



Traits are properties that are expressed at some level. Emergent traits are defined as traits that are not reducible to a lower level \citep{Jablonski2008a}OTHER CITATIONS. An emergent trait is thus a relative concept which must be defined in relation to a specific organizational level (e.g. species, organism, etc.). In paleobiology, emergent traits are frequently defined as those properties not reducible to the organismal level. If a trait is not reducible, it is then considered species, or genus, level traits.

Range size is continually cited as an emergent, species-level trait CITATIONS. 

The studies I am and will be undertaking are related to organismal traits such as dietary category and substrate affinity. Each of these traits can be considered to be related to the emergence of range-size. However, the relative importance of these different traits and their interactions in terms of fitness and extinction risk SOMETHING


\section{Cosmopolitan and endemic mammal dynamics of Cenozoic mammals}

\textit{Questions:} Do different continental populations of terrestrial mammals demonstrate different patterns of community similarity and change over the Cenozoic? Are these patterns related to ecological characters, such as dietary category and locomotor category? Are these patterns related to changes in global temperature?

\textit{Hypotheses and Predictions:}

\textit{Reasoning:}


\section{Fitness and extinction risk related to ecology in Cenozoic mammals}

\textit{Questions:} How do ecological characters, such as dietary category and locomotor category, affect extinction risk in mammals? Does this relationship change over time? Is anyone trait the best predictor of extinction risk, or do multiple traits together or interacting better explain extinction?

\textit{Hypotheses and Predictions:}

\textit{Reasoning:}


\section{Extinction risk related to traits affecting habitat selection in Permian brachiopods}

\textit{Questions:} How do traits directly related habitat selection and range size relate to extinction risk? Are certain traits more explanatory of extinction risk than others? Does changing environmental and substrate availability affect trait-based extinction risk?

\textit{Hypotheses and Predictions:}

\textit{Reasoning:}

\section{FOURTH UNKNOWN CHAPTER}

\textit{Questions:}

\textit{Hypotheses and Predictions:}

\textit{Reasoning:}


\section{Importance}


\clearpage
\bibliographystyle{abbrvnat}
\bibliography{proposal}

\end{document}
