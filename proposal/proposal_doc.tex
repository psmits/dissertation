\documentclass[12pt,letterpaper]{article}

\usepackage{amsmath, amsthm}
\usepackage{graphicx}
\usepackage{microtype, parskip}
\usepackage{caption, subcaption, multirow, morefloats, rotating, longtable}
\usepackage{hyperref}
\usepackage[numbers,sort&compress]{natbib}
\usepackage{authblk, attrib, fullpage}
\usepackage{lineno}


\begin{document}
\setcounter{secnumdepth}{0}

\begin{titlepage}
  \begin{center}
    \huge{Evolutionary paleoecology and the biology of extinction}

    \vspace{1.5cm}

    \large{Peter D. Smits \\}
    \footnotesize{\href{mailto:psmits@uchicago.edu}{psmits@uchicago.edu}}

    \vspace{1.5cm}

    Dissertation Proposal Hearing \\
    \today \\
    Commmittee on Evolutionary Biology \\
    The University of Chicago

    \vspace{1.5cm}

    \textit{Committee} \\
    Dr. Michael J. Foote (co-advisor) \\
    Dr. Kenneth D. Angielczky (co-advisor) \\
    Dr. Richard H. Ree \\
    Dr. P. David Polly
  \end{center}
\end{titlepage}

\linenumbers
\modulolinenumbers[2]


\section{Introduction and theoretical framework}

\subsection{Evolutionary paleoecology}
Evolutionary paleoecology is defined as the study of the effects of ecological traits and factors on differential rate dynamics, particularly rates of faunal turnover and diversification \citep{Kitchell1985a}. Ecological traits and factors are any and all traits expressed by a taxon, at any level, that are involved with biotic--biotic or biotic--abiotic interactions. Diversification is the difference between origination and extinction, and is thus the net pattern of macroevolution. The study of evolutionary paleoecology is then the link between interactions and macroevolution. Namely, it is the study of the ecological interactions that  may or may not generate the patterns of macroevolution. As a corrolary to \citet{Kitchell1985a}'s definition, \citet{Allmon1994} states that in order to correctly link ecological interactions to macroevolution, one must focus on the specific traits and factors that may affect the speciation process. Tacitly included in this is then the study of the biology of extinction and how it relates to ecological properties and interactions \citep{Kitchell1990}.

Importantly, \citet{Simpson1944} defined the set of all biotic and abiotic interactions that a lineage may experience as the ``adaptive zone.'' The also termed this the ``environment'' which the dynamics of were later expanded upon by \citet{VanValen1973}. This usage of ``environment'' is frequently misinterpreted thus it is important to note it here. The adaptive zone represents one of the fundamental metaphors in paleobiology \citep{Simpson1944,Simpson1953}. 

It is under this framework that I purpose to study how ecological traits associated with range size have affected both the availability of biotic interactions and differential survivorship. I will be comparing survivorship in two very distantly related and biotically very different groups: mammals and brachiopods. Both of these groups are considered to have very good fossil records able to reflect massive long term evolutionary patterns \citep{Mark1977}. RELATIVE MERITS

%anagenisis
%taxonomy
%Jurassic of Europe Kiessling Aberhaan
%John Hunter OHIO STATE
%-Watson Seminar
%Marie Hoerner


\subsection{Paleobiological theory}
% definition
% metaphor
% metonymy

Extinction, when not during periods of mass extinction, is widely considered non-random with respect to biology \citep{Jablonski1986}. Additionally, times of ``background extinction'' represent the vast majority of geological time compared to periods of ``mass extinction.'' The exact definition and distinction between periods of background and mass extinction is based on the relative intensity of extinction at a particular time versus times preceding and following. Simply put, the major indicator of a mass extinction is an event that is across taxonomic categories and the biotic interactors before and after the event are dramatically different \citet{Jablonski1986,Jablonski2005,Kitchell1986,Kitchell1991}.

Paleobiology is the study of life over time and the processes that generate the observed patterns in diversity and disparity. Intimately related to paleobiology is the concept of macroevolution which is here defined as the pattern of speciation and extinction over time \citep{Jablonski2008a}. The study of macroevolution, thus, is the estimation of the processes underlying these observed patterns. The term origination is frequently used in place of speciation because it is considered impossible to observe speciation in the fossil record and instead we only observe the sudden appearance or origination of a new taxon \citep{Coyne2004}. Additionally, origination includes both speciation and migration because, depending on both the spatial scale and quality of the fossil record, it may be impossible to distinguish between the two.

Macroevolution, as metaphor, is both phyletic and anagenetic evolutionary dynamics \citep{Foote2007b}. Phyletic means speciation/extinction dynamics and anagenetic means within lineage disparity dynamics. This concept has also been termed the tempo and mode of evolution \citep{Simpson1944}. This broader definition more closely links paleobiology and macroevolution. This raises an important question: can we have paleobiology without macroevolution?

In contrast to macroevolution there is microevolution \citep{Simpson1944,Foote2007b} which is strictly defined as change in allele frequency in a population from one generation to the next. A weaker definition is that microevolution is change below the species level \citep{Foote2007b} though there is no qualifier on what this change is defined as. It is important to note that changes in allele frequency affect phenotype frequency and expression. This definition, by corollary, describes phenotypic change. Just as \citet{Simpson1944} described, this link can explain rates of evolution and (most) patterns of disparity without invoking any macroevolutionary phenomenon.

What makes a definition strong or weak? Why would we prefer one over the other? What is the difference between definition, metaphor and metonomy?

Of major concern with the broader definition of macroevolution is that this concept subsumes all aspects of anagenetic change. The difference between microevolution versus macrevolution is unclear. Interestingly, the link between broad sense macroevolution and Simpson's tempo and mode of evolution is that Simpson's statement assigns no hierarchical level to these patterns. The pervasiveness of the use of broad sense macroevolution then is because this usage is metaphoric and explicitly because it is not the actual definition of macroevolution.

An important theoretical construct in paleobiology is that of emergent traits or properties \citep{Jablonski2008a,Vrba1984}. Emergent traits are defined as traits that are not reducible to a lower level \citep{Grantham1995,Vrba1984,Jablonski2008a,Lloyd1993}. An emergent trait is thus a relative concept which must be defined in relation to a specific organizational level (e.g. species, organism, etc.). An emergent property is effectively a trait that is the product of multiple traits expressed at one or more hierarchical levels in concert and is not reducible to a lower level \citep{Vrba1984,Jablonski2008a}. It is extremely important to recognize a fundamental hierarchy in biology in order to allow for emergent traits \citet{Vrba1984}. In paleobiology, an emergent property is normally one ascribed to the species or genus level as opposed to the organism level \citep{Grantham1995,Jablonski2008a,Vrba1984b,Lloyd1993}, and thus the root of the term ``species selection'' \citep{Jablonski2008a,Vrba1984,Lloyd1993}. Range size is considered an emergent property that has continually be demonstrated to play a crucial role in extinction selectivity with species with larger geographic ranges having lower extinction rates than species with smaller geographic ranges \citep{Jablonski1986,Harnik2013,Nurnberg2013a}. Range size is emergent because no one property of a single organism can explain this trait, instead it is a combination of multiple properties in addition to multiple members of the species which helps to determine global range size.

Survival can be considered the ultimate measure of fitness or evolution success \citep{Cooper1984,Palmer2012} because ultimately a long-term successful lineage is not one that speciated greatly but one that never went extinct \citep{Palmer2012}. Because during periods of background extinction, patterns of taxonomic extinction are non-random with respect to biology \citep{Jablonski1986}. Survival is effectively the opposite of extinction, thus during periods of background extinction we should be able to effectively measure the relative fitness of various ecologies or adaptive zones \citep{Simpson1944,Kitchell1990,Kitchell1985a,VanValen1973}. 

Survivorship is a statement of a sample of organisms and is a summary over all of them. This approach to studying extinction and selection has a long history in paleobiology \citep{Simpson1944,VanValen1979,Foote1988,Baumiller1993,Kitchell1987b,Kitchell1990,Simpson1953,Raup1991a,Simpson2006}. This is not the same as temporal extinction rate \citep{Foote2000,Foote2000a,Alroy2010b,Alroy2010,Alroy2010c}. The hazard/extinction rate from a survivorship curve is a statement of the rate at which organisms go extinction following origination. Temporal extinction rates are statements of how extinction intensity has varied across lineages over time. Hazard rates are potentially more comparable to the diversification rates estimated via phylogenetic comparative methods \citep{Fitzjohn2010,Maddison2007,Rabosky2013,Nee2001,Nee1994d,Nee1992}, however hazard rates are an estimate of extinction rate which is considered difficult if not impossible to estimate in this fashion \citep{Rabosky2010a}.

Here I propose to study the individual and combined effects of organismal traits related to emergent range size on extinction and, by extension, fitness.


\section{Cosmopolitan and endemic mammal dynamics of Cenozoic mammals}

\textit{Questions:} How do patterns of community connectedness change over time, specifically in Cenozoic terrestrial mammals? Do these patterns differ between continents? Are these patterns related to ecological traits, such as dietary category and locomotor category? Are these patterns related to changes in global temperature?

\textit{Background and Predictions:}
During the Cenozoic, there was a global shift from predominately closed, forested habitats to more open, savanna habitats. This pattern leads to the expectation that there would be a relative decrease in arboreal taxa as well as a relative decrease in terrestrial taxa. Additionally, an increased in the relative endemism of arboreal taxa over time and an decrease in the relative endemism of terrestrial taxa might have accompanied this. The timing of this shift was different between continents \citep{Stromberg2005,Stromberg2013}, meaning that the patterns reflecting this environmental shift may be globally non-uniform and any progressive changes in community structures would reflect regional scale changes instead of any global trend.

However, this expected shift in distribution of taxa according to locomotor category is not necessarily accompanied by broad shifts in distribution of (coarse) dietary categories. The majority of previous research has focused on mammalian faunal dynamics has focused on the North American fossil record \citep{Alroy2000g,Alroy1996a,Alroy1998,Barnosky2001a,Simpson1944,Simpson1953,Badgley2013,Blois2009,Figueirido2012,Gunnell1995,Hadly2001}. The major focuses have been on the effect of climate change on diversity and distributions between different higher taxonomic levels. The long term effects of climate change on North American mammalian diversity dynamics and community connectedness and similarly remains unresolved and controversial \citep{Alroy2000g,Blois2009,Figueirido2012,Barnosky2001a}. The basic predictions that can be made are coherent with the general predictions made above. Namely, that over the Cenozoic there would be a relative increase in endemism in arboreal taxa versus a relative decrease in ground dwelling endemism. Because of the vast amount of prior work on North American mammalian faunal dynamics, this forms the basis for the global predictions made above. The North American record then inadvertently becomes the baseline comparison for regional differences.

In comparison to North America, the European mammalian fossil is less studied. Importantly, a great deal of work has focused on faunal dynamics in the Neogene \citep{Jernvall2002,Jernvall2004,Liow2008,Raia2006,Raia2005,Raia2011c}. One of the major findings is that, during the Neogene, there was very little shift in relative trophic level abundance \citep{Jernvall2004} while the patterns in dietary shifts were mostly driven by abundance and cosmopolitan herbivores \citep{Jernvall2002}. Because of this, the major predictions for the European record is that occupancy will increase for herbivorous taxa, while increasing or remaining identical in carnivores, and remaining relatively constant for omnivores. These are the expectations for at least the Neogene.

The record of the South American mammalian fauna appears to reflect two distinct biotic provinces between the North and the South \citep{Macfadden1997,Macfadden2006,Flynn1998a,Patterson1968}. Because of this, I predict the South American record to have a very different pattern of biogeographic connectedness than either North America or Europe, specifically in terms of endemism. Namely, the expectation would be a high or progressively increasing degree of endemism along with low connectivity. Also, an increase in land-dwelling herbiovres relative to arboreal (at least in the south) would be expected.

A global trend during the Cenzoic was the shift from a ``hot house'' environment with no polar ice caps to an ``ice house'' environment with polar ice caps \citep{Zachos2008,Zachos2001}. NEED SOME LITERATURE ON RANGE SIZE SHIFTS AND CLIMATE SEE MACRO NOTES. This transition is predicted to cause major shifts in biomes, causing reorganization of communities CITATIONS. For global mammalian community connectedness and trophic structure, it is predicted that while the environment might be shifting, lineages may adapt in place and overall structure will remain rather constant through time \citep{Jernvall2004}.

\textit{Proposed research:}
Using methods first proposed by \citet{Sidor2013} and \citet{Vilhena2013}, I propose to construct bipartite biogeographic networks between taxa and localities. A link between a taxon and a locality is formed when that is present at that locality. Because the network is bipartite, by definition there are no links between a taxon and another taxon or a locality and another locality. Here taxa are defined as species and localities are defined as formations. Biogeographic networks will be constructed for every 2 million year bin of the Cenozoic. This bin width is chosen to have minimum 2 formations to be present in the same bin, even though it has been found that at least the North American fossil record is resolvable at the 1 My level \citep{Alroy2000a,Alroy1996a,Alroy1998}. Additionally, while the European fossil record may be resolvable at comparable levels to the North American record \citep{Jernvall2002,Jernvall2004} OTHER CITATIONS, it is expected that the South American fossil record may not be accurately resolvable at the 1 My level.

Additionally, networks will be constructed for each dietary category and each locomotor category. These ecology specific networks will also be made for every 2 million year bin of the Cenozoic.

Network complexity and connectedness is measured using the four summary statistics previously used \citep{Sidor2013}: average number of endemics, average occupancy, biogeographic connectedness, and code length. Biogeographic connectedness is effectively the relationship between the number of endemic taxa and the average occupancy. Code length is a measurement of the complexity and clustering of the graph and is measured using the map equation \citep{Rosvall2008,Rosvall2010b}. A low code length indicates that a graph is compressible into a greater number of subunits without information loss than a graph with a high code length. Effectively this means that a low code length indicates lower overall locality similarity than a high code length.

In order to compare whether patterns observed on different continents are similar or different, as well as compare patterns between different categories of ecological traits, HOW DO I DO THIS?

Taxonomic occurrence data will be collected through a combination of the Paleobiological Database (PBDB; \url{http://fossilworks.org}), Neogene Old World Database (NOW; \url{http://www.helsink.fi/science/now/}), and museum collections. North American fossil mammal data is very well represented and vetted in the PBDB because of the extensive work by John Alroy \citep{Alroy1996a,Alroy1998,Alroy2000g}. European fossil mammal data is also well represented between the PBDB and NOW. The major concern of data duplication is not an issue for this study because only presence--absence information will use used to create the biogeographic networks. South American fossil mammal data is available through the PBDB, but is not particularly well vetted and poorly covered. Because of this, South American fossil mammal data will be gathered via various museums such as the Field Museum of Natural History and the American Museum of Natural History both of which are known to house excellent South American fossil mammal collections.


\section{Fitness and extinction risk related to ecology in Cenozoic mammals}

\textit{Questions:} How do ecological characters, such as dietary category and locomotor category, affect survivorship in mammals? Is the hazard (instantaneous extinction rate) in adaptive zones constant with respect to taxon age, or does instantaneous extinction rate change with taxon age? Is any single trait the best predictor of survivorship, or do multiple traits together predict survivorship better?

\textit{Background and Predictions:} 

In mammals, three of the arguably most important ecological traits are dietary category, locomotor category, and body size CITATIONS. Additionally, each of these traits are important in determining the emergent species level property of range size. 

%While diet and locomotor categories are known to evolve in context amongst terrestrial mammals, they are not the same trait and thus it is very possible that there is not necessarily any direct causal connection between the evolution of these traits. However, it may be quite possible that there is a correlation between them. This is similar, also, with correlations between either of these traits and body size.

As discussed above, as the Cenozoic progressed on all Continents there was a shift from closed habitat to more open habitat. In the intermediate, one would expect some degree of patchiness in the landscape. Expectedly, there would be a increase followed by a decrease in both speciation rate and extinction rate, with the peak being during the intermediate period.

According to \citet{Price2012}, herbivores and carnivores should have a greater diversification rate than omnivores. This analysis was global in scope, and purely extant taxa in a comparative phylogenetic context. Importantly, diversification rate is the difference between origination and extinction rate. Diversification rate can increase via either an increase in origination relative to extinction or a decrease in extinction relative to origination. Which of these two processes is occurring is impossible to determine from a phylogeny of only extant organisms \citep{Rabosky2010a} which means that only via the analysis of the mammalian fossil record is it possible to estimate which process is more likely. 

Depending on the continent, body size has been demonstrated to play either some or no roll in extinction selectivity during the Neogene \citep{Tomiya2013,Liow2008,Liow2009}. By expanding to include a third continent, South America, I hope to elucidate how differences in taxonomic diversity at a continental level might affect body size mediated extinction risk. Additionally, I will be using alternative methods to better understand the dynamics governing trait based extinction probability.


Given \citep{Jernvall2002}, it might be expected that the pattern for all mammals will be most similar to the pattern from (common) herbivores. However, I am unsure if this is as reasonable prediction for survivorship as opposed to biogeographic connectedness.



\textit{Proposed research:}
To investigate the effect of ecological traits and climate change on survivorship, I plan to analyze differential survivorship and compare models of hazard in order to best understand what are the most important factors in estimating survival probability.

Survivorship analysis is the analysis of time-till-event data. In a paleontological context this means the time from origination till extinction. Time of origination is measured as the first appearance date (FAD) and time of extinction is measured as the last appearance date (LAD). The following time-independent covariates will be modeled as predictors of survival: dietary category, locomotor category, body size. The climate proxy \(\delta O^{18}\) oxygen curve from \citet{Zachos2008} will be modeled as an ancillary time-dependent covariate.

There are four basic probability distributions used to model survival: exponential, Weibull, lognormal, and log logistic CITATIONS. The Law of Constant extinction states that the shape of a survival curve is linear on a semi-log scale for any taxon in a particular adaptive zone, meaning that extinction rate is constant with respect to time \citep{VanValen1973}. This means that survival should be best modeled using the exponential distribution because has a single parameter, meaning that the logarithm of survival is constant with respect to time.

ADD IN MODELS. NEED TO READ MORE SURVIVAL BOOK.

While many analyses of extinction risk and survivorship are done using generic level data \citep{Tomiya2013,Liow2008,Harnik2013}, there is a known biasing effect in survivorship analysis of paleontological data where the hazard function describing the survivorship curve is biased away uniformity \citep{Raup1975,Sepkoski1975} which can cause potentially false results in relation to the Van Valen's Law of Constant Extinction \citep{VanValen1973}. There are also important concerns regarding anagenetic lineages, hierarchical selection, extant taxa or taxa that did not go extinct in the time frame of interest \citep{Raup1975,VanValen1979,Simpson2006,Raup1991a} PROBABLY MORE. Interestingly, the effect of incomplete sampling on estimation of survivorship curves appears rather minimal and uniform \citep{Sepkoski1975}. However, the problems involving extant taxa and taxa that did not go extinct have mostly been dealt with following advances of how to describe right-censored survivorship data and estimate the likelihood of various parametric models of hazard and survivorship CITATION TEXTBOOK. 

Simulation study to analyze effect of varying sampling probability and/or anagenesis rate on estimating various models of survivorship using \texttt{paleotree} \citep{Bapst2012a}. Principally, I am interested in the effect of these paleontological realities on estimation of the hazard function of the survivorship data and in particular departures from a constant, or exponential, hazard function. Alternatives are, for example, models of accelerating or decelerating extinction rate. I intend to revise the previously analyzed effect of sampling on estimation survivorhip in this new context \citep{Sepkoski1975}.

Do not use generic data? Or figure out a correction factor? Do both generic and specific level analysis and then use \citep{Foote1988} to get deviance/concordance and see what potential biasing effect might be occurring?


\section{Extinction risk related to traits affecting habitat selection in Permian brachiopods}

\textit{Questions:} How do traits directly related habitat selection and range size relate to extinction risk? Are certain traits more explanatory of extinction risk than others? Does changing environmental and substrate availability affect trait-based extinction risk?

\textit{Background and Predictions:}
Affixing strat: pedicle, reclining, cementing. Pedicles are associated with shallow, on-shore environments while reclining on associated with deep, off-shore environments. Note, global Permian pattern is rather heterogeneous in terms of relative abundance between the three affixing strategies CLAPHAM. According to \citet{Alexander1977}, endemic unattached/reclining taxa have longer durations than all other affixing strategies of other endemic taxa. on the other hand, cosmopolitan pedunculate and cementing taxa have longer durations than unattached/reclining taxa. in terms of regional scales, it will be necessary to correct for relative abundance to actally measure survivability/fitness.

Substrate: carbonate, clastic, mixed. Overall decline in carbonates relative to clastics, decreased fitness for carbonate lovers relative to clastic lovers. Best to be mixed? Brachiopods are mixed slash switchers CARL AND MELANIE'S UNPUBLISHED WORK.

Habitat: on-shore vs off-shore / epicontinental vs oceanic. How possible is this? Very difficult to resolve accurately. Progressive loss of epicontinental seas during the Pharenozoic which would predict over all lower fitness in taxa which occur solely in epicontinental environments. On-shore off-shore dynamics of Sepkoski and Raup.

During the Permian there was a shift from an ``ice house'' world to a ``hot house world'' CITATIONS which could be expected to have some major effects on brachiopod survivorship. In particular, taxa in modern Australia would be of particular interest because of the proximity of Australia to the south pole during the Permian and the repeated glacial activity in the region CITATIONS. According to \citet{Olszewski2004}, sea-level and climate change do not wholly explain the ecological dynamics experienced by brachiopods in the Permian of Texas. The prediction then is that the best model of brachiopod survivorship will have to have some biotic component such as affixing strategy or substrate preference. If climate or environmental information, such as habitat preference, is a predictor in the best model of survivorship is less clear cut and necessary to determine empirically.


\textit{Proposed research:}



%\section{Importance}



\clearpage
\section{Timeline}

Spring/Summer 2014
\begin{itemize}
  \item Evolution Meeting: brachiopods
  %\item cosmo prov
  %\item survivor
\end{itemize}

Fall 2014/Winter 2015
\begin{itemize}
  \item GSA: simulation
  %\item cosmo prov
  %\item survivor
  %\item mammal risk
\end{itemize}

Spring/Summer 2015
\begin{itemize}
  \item Evolution Meeting: mammals
  %\item survivor
  %\item mammal risk
  %\item review 1
\end{itemize}

Fall 2015/Winter 2016
\begin{itemize}
  \item SVP or GSA
  %\item mammal risk
  %\item review 1
  %\item review 2
\end{itemize}

Spring/Summer 2016
\begin{itemize}
  \item Evolution Meeting
  %\item review 1
  %\item review 2
\end{itemize}

Fall 2016/Winter 2017
\begin{itemize}
  \item SVP or GSA
  %\item review 2
\end{itemize}

Spring/Summer 2017
\begin{itemize}
  \item Evolution Meeting
  \item \textbf{Defend}
\end{itemize}



\clearpage
\bibliographystyle{abbrvnat}
\bibliography{proposal}

\end{document}
