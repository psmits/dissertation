\documentclass[12pt,letterpaper]{article}

\usepackage{amsmath, amsthm}
\usepackage{graphicx}
\usepackage{microtype, parskip}
\usepackage{caption, subcaption, multirow, morefloats, rotating, longtable}
\usepackage{hyperref}
\usepackage[numbers,sort&compress]{natbib}
\usepackage{authblk, attrib, fullpage}
\usepackage{lineno}

\begin{document}
\setcounter{secnumdepth}{0}

\begin{Large}
  \textbf{Pre-proposal defense committee meeting agenda}
\end{Large}

\section{Comments on current proposal draft}
\begin{itemize}
  \item Michael's concerns
    \begin{itemize}
      \item Trait assignments
        \begin{itemize}
          \item Current substrate and habitat assignments are preliminary
            \begin{itemize}
              \item Review literature for improved site/formation descriptions
              \item Increase assignment resolution
            \end{itemize}
          \item Currently, there are no reef(-like) assignments for habitat in the PBDB data
        \end{itemize}
      \item Scaling up of traits
        \begin{itemize}
          \item Proposal has shifted from testing ``emergence'' directly
          \item Precise language in proposal and talk necessary
        \end{itemize}
      \item Preliminary results
        \begin{itemize}
          \item Very poor/rough right censoring criterion
          \item To be improved with more conservative ``study end'' date
          \item LAD uncertainty via interval censoring
          \item No climatic context
        \end{itemize}
      \item Regional/local climate proxies instead of purely global
        \begin{itemize}
          \item Mammals: unknown, must research
          \item Brachiopods: well resolved glacial patterns for Australia thanks to Tracy Frank and Christopher Fielding
        \end{itemize}
    \end{itemize}
\end{itemize}

\section{Proposal talk}
\subsection{Current outline}
\begin{itemize}
  \item Introduction to theory
    \begin{itemize}
      \item Law of Constant Extinction and survival analysis
    \end{itemize}
  \item Brachiopod survival
    \begin{itemize}
      \item Biotic traits and climatic context
      \item Proposed analysis
      \item Preliminary results without conclusions
    \end{itemize}
  \item Mammal survival
    \begin{itemize}
      \item Biotic traits and climatic context
      \item Proposed analysis
    \end{itemize}
  \item Mammal community connectedness
    \begin{itemize}
      \item Bipartite biogeographic networks
      \item Community measures
      \item Hypotheses/predictions
      \item Preliminary results without conclusions
    \end{itemize}
  \item Synthesis
\end{itemize}


\section{Future concerns and directions}
\subsection{Projects}
\begin{itemize}
  \item Preliminary results
    \begin{itemize}
      \item Improve censoring (combination right, left and interval)
      \item More conservative ``thresholds'' near boundaries
      \item Execute phylogenetic measures for community connectedness
    \end{itemize}
  \item Survival analysis
    \begin{itemize}
      \item In context of phylogeny? Some work has been done recently to include pedigrees and such. 
      \item Develop (new) PCM method (with Rick)?
      \item Survival analysis is similar to a regression problem, though likelihood is calculated differently to allow for different censoring methods.
    \end{itemize}
  \item Brachiopods
    \begin{itemize}
      \item Assign affixing strategies
      \item Improve on current coarse assignments
      \item Site/formation descriptions from literature
      \item Include Chinese record for latitudinal comparison
        \begin{itemize}
          \item Latitude/temperature has not been found to limit dispersal/range in modern Brachiopods
        \end{itemize}
    \end{itemize}
  \item Mammals
    \begin{itemize}
      \item Hind's Fund: obtained
      \item South American mammal taxonomy congruence between FMNH and AMNH
      \item Review catalogue information from FMNH collection
      \item Body size from literature search
    \end{itemize}
\end{itemize}

\subsection{Conferences}
\begin{itemize}
  \item Preliminary brachiopod results at Evolution 2014
    \begin{itemize}
      \item currently have substrate preference and habitat affinity assigned
    \end{itemize}
  \item Simulation of effects of differential preservation on survival at GSA 2014
    \begin{itemize}
      \item time-homogeneous birth--death simulation implemented
      \item Poisson process fossilization implemented
    \end{itemize}
\end{itemize}

\end{document}
